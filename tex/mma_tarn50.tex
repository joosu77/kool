\documentclass{article}
\usepackage{amsfonts}
\usepackage{amsmath}
\usepackage{mathtools}
\usepackage{systeme}
\usepackage{polynom}
\usepackage{pgfplots}
\everymath{\displaystyle}
\DeclarePairedDelimiter\ceil{\lceil}{\rceil}
\begin{document}
\begin{center}
\Large\textbf{Tärnülesanne 50}\\
\small{Joosep Näks}
\end{center}
Olgu antud kahe muutuja funktsioon $f:D\to\mathbb{R}$ ning $A\in D^\circ$. Ülesande 45 põhjal on teada, et kehtivad implikatsioonid
\begin{gather*}
f\text{ on diferentseeruv punktis }A\quad\Rightarrow\quad \forall \vec{s}=(s_1,s_2)\neq \vec0\quad \exists\frac{\partial f}{\partial\vec s}(A)\in\mathbb{R}\\
f\text{ on diferentseeruv punktis }A,\quad\frac{\partial f}{\partial x}(A)=\frac{\partial f}{\partial y}(A)=0\quad\Rightarrow\quad \forall \vec{s}=(s_1,s_2)\neq \vec0\quad \exists\frac{\partial f}{\partial\vec s}(A)=0.
\end{gather*}
Kas emb-kumb neist implikatsioonidest on üldiselt pööratav?\\\\
\textbf{Lahendus:}
Kumbki implikatsioon ei ole pööratav. Näitan seda vastunäitega. Vaatlen funktisooni $f(x,y)=\frac{x^3}{x^2+y^2}$ diferentseeruvust punktis $(0,0)$. Sellel leidub tuletis igas suunas:
\begin{gather*}
\forall\vec{s}=(s_1,s_2)\neq\vec0\quad \frac{\partial f}{\partial\vec s}(0,0)=\lim_{t\to0}\frac{\frac{t^3s_1^3}{t^2s_1^2+t^2s_2^2}}{t}=\frac{s_1^3}{s_1^2+s_2^2}\in\mathbb{R}
\end{gather*}
\end{document}