\documentclass{article}
\usepackage{amsfonts}
\usepackage{amsmath}
\usepackage{systeme}
\usepackage{mathtools}
\usepackage{forest}
\usepackage{xcolor}
\usepackage{tikz}
\usetikzlibrary{positioning}
\DeclarePairedDelimiter\ceil{\lceil}{\rceil}
\addtolength{\oddsidemargin}{-1in}
\addtolength{\evensidemargin}{-1in}
\addtolength{\textwidth}{1.75in}
\begin{document}
\begin{center}
\Large\textbf{Kodut\"o\"o nr. 3}\\
Joosep N\"aks
\end{center}
\textbf{1.} T\~oestage, et naturaalarvudel on j\"argmine omadus:\\
\begin{equation*}
\forall x \forall y (x'\cdot y=x\cdot y+y).
\end{equation*}
\textbf{T\~oestus:} T\~oestan induktsiooniga, et j\"argnev valem kehtib iga $y$ korral:\\
\begin{equation*}
\begin{aligned}
\mathcal{F}(y) = \forall x (x'\cdot y=x\cdot y+y).
\end{aligned}
\end{equation*}
\textbf{Alus:}
\begin{equation*}
\begin{aligned}
x'\cdot0&=0\quad(P5)\\
&=x\cdot0\quad(P5)\\
&=x\cdot0+0\quad(P3)
\end{aligned}
\end{equation*}
\textbf{Samm:} Eeldan et kehtib $\forall x (x'\cdot y=x\cdot y+y)$, n\"aitan, et sel juhul kehtib $\forall x (x'\cdot y'=x\cdot y'+y')$:
\begin{equation*}
\begin{aligned}
x'\cdot y'&=x'\cdot y+x'\quad(P6)\\
&=(x\cdot y+ y)+x'\quad(induktsioonieeldus)\\
&=x\cdot y+ (y+x')\quad(liitmise assotsiatiivsus,\ teoreem\ 1.114)\\
&=x\cdot y+ (y+x)'\quad(P4)\\
&=x\cdot y+ (x+y)'\quad(liitmise kommutatiivsus,\ teoreem\ 1.115)\\
&=x\cdot y+ (x+y')\quad(P4)\\
&=(x\cdot y+ x)+y'\quad(liitmise assotsiatiivsus,\ teoreem\ 1.114)\\
&=x\cdot y'+y'\quad(P6)\\
\end{aligned}
\end{equation*}
Seega kehtib $\forall x \forall y (x'\cdot y=x\cdot y+y)$ induktsiooniaksioomi P7 p\~ohjal.
\pagebreak\\
\textbf{2.} On antud 9 tipu ja 17 servaga graaf. Seejuures on teada, et graafis iga tipu aste on 3 v\~oi 4. Leidke astmega 3 tippude arv.\\
\textbf{Lahendus:} V\~otan astmega 3 tippude koguseks q ja astmega 4 tippude koguseks $\varpi$. On teada, et kokku on tippe 9 ehk kehtib $q+\varpi=9$. On teada ka, et k\~oigi tippude astmete summa on v\~ordne kahekordse graafi servade arvuga (loengukonspekti teoreem 2.20), seega kehtib ka $3q+4\varpi=2\cdot17=34$. Lahendan v\~orrandis\"usteemi:
\begin{equation*}
\begin{aligned}
\systeme{
	q+\varpi=9,
	3q+4\varpi=34
}\\
q=9-\varpi\\
3(9-\varpi)+4\varpi=34\\
27-3\varpi+4\varpi=34\\
\varpi=7\\
q=2
\end{aligned}
\end{equation*}
Seega on tippude arv, mille aste on 3, 2.
\pagebreak\\
\textbf{3.} Tooge n\"aide v\"ahemalt kahe tipuga graafist, mis on isomorfne oma t\"aiendgraafiga. Kirjutage v\"alja nend vaheline isomorfism.\\
\textbf{Lahendus:} Minu n\"aide on graaf $G=(\{A,B,C,D\},\{\{A,B\},\{B,C\},\{C,D\}\})$, mille t\"aiendgraaf on $G'=(\{A,B,C,D\},\{\{B,D\},\{D,A\},\{A,C\}\})$
\begin{align*}
\begin{gathered}
\textbf{G:}
\end{gathered}
&&
\begin{gathered}
\textbf{G':}
\end{gathered}
\end{align*}
\begin {center}
\begin {tikzpicture}[auto, node distance = 2cm and 2cm, on grid, thick, main node/.style ={circle, draw, minimum width = 1cm}]
        \node[main node] (A) {A};
        \node[main node] (B) [left = of A] {B};
        \node[main node] (C) [below = of A] {C};
        \node[main node] (D) [below= of B] {D};
        \path (C) edge (D);
        \path (B) edge (A);
        \path (C) edge (B);
\end{tikzpicture}
\qquad\qquad\qquad\quad
\begin {tikzpicture}[auto, node distance = 2cm and 2cm, on grid, thick, main node/.style ={circle, draw, minimum width = 1cm}]
        \node[main node] (A) {A};
        \node[main node] (B) [left = of A] {B};
        \node[main node] (C) [below = of A] {C};
        \node[main node] (D) [below= of B] {D};
        \path (A) edge (D);
        \path (A) edge (C);
        \path (D) edge (B);
\end{tikzpicture}
\end{center}
Graafist G saab v\~otta isomorfismi $A\to B, B\to D, C\to A, D\to C$, siis saame graafi \\$V=(\{A,B,C,D\},\{\{B,D\},\{D,A\},\{A,C\}\})$, mis ongi $G$ t\"aiendgraaf.
\pagebreak\\
\textbf{4.} Olgu antud sidus graaf, milles iga tipu aste on 3. On teada, et $u$ on eraldav tipp. T\~oestagem et leidub serv $uv$ nii, et $uv$ on sild.\\
\textbf{T\~oestus:} Definitsiooni kohaselt on $u$ eraldav tipp parajasti siis, kui tipu $u$ ja temaga indentsete servade eemaldamisel graafi sidususkomponentide arv suureneb. Seega kuna on teada, et iga tipu aste on 3, v\"aljub tipust t\"apselt 3 serva $ua$, $ub$ ja $uc$. Kuna $u$ on eraldav tipp, peavad tippudest $a$, $b$ ja $c$ 2 t\"ukki olema p\"arast $u$ eemaldamist erinevates sidusates komponentides, kuna muidu ei suureneks sidusate komponentide kogus. V\~otan nendeks tippudeks $a$ ja $b$. \\
Kui tipp $c$ on p\"arast $u$ eemaldamist samas sidusas komponendis nagu $b$, on serv $ua$ sild, kuna $a$ ja $b$ on \"uhendatud vaid tipu $u$ kaudu. \\
Samal p\~ohjusel, kui tipp $c$ on p\"arast $u$ eemaldamist samas sidusas kpmponendis nagu $a$, on serv $ub$ sild.\\
 Kui p\"arast $u$ eemaldamist pole $c$ samas sidusas komponendis nagu $a$ ega ka $b$, siis on k\~oik servad $ua$, $ub$ ja $uc$ sillad, kuna nad on \"uhendatud vaid tipu $u$ kaudu. Seega on igal juhul v\"ahemalt \"uks tipuga $u$ indentsetest servadest sild ehk leidub sild $uv$.
\end{document}