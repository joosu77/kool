% --- LaTeX CV Template - S. Venkatraman ---

% --- Set document class and font size ---

\documentclass[letterpaper, 11pt]{article}

% --- Package imports ---

\usepackage{hyperref, enumitem, longtable, amsmath, array}

% --- Page layout settings ---

% Set page margins
\usepackage[left=0.7in, right=0.8in, bottom=.8in, top=0.8in, headsep=0in, footskip=.2in]{geometry}

% Set line spacing
\renewcommand{\baselinestretch}{1.2}

% --- Page formatting settings ---

% Set link colors
\usepackage[dvipsnames]{xcolor}
\hypersetup{colorlinks=true, linkcolor=MidnightBlue, urlcolor=MidnightBlue}

% Set font to Libertine, including math support
\usepackage{libertine}
\usepackage[libertine]{newtxmath}

% Remove page numbering
\pagenumbering{gobble}

% Define font size and color for section headings
\newcommand{\headingfont}{\Large\color{OliveGreen}}

% --- CV section settings ---

% Note: each section of this table (Education, Awards, Publications etc.) is 
% stored in a two-column table. The left-hand column is narrow (1 inch) and is 
% meant to store dates. The right-hand column is wide (5.2 inches) and stores 
% the main text.  Sections in which each entry might have multiple lines 
% (e.g., Education) are stored in a 'SectionTable' environment). Sections in 
% which each entry might just have one line are stored in a 'SectionTableSingleSpace'
% environment. The only difference between the two environments is the line 
% spacing between each entry. Both environments take one argument, which is the
% title of the section. See main document for how these environments are used.

% Define settings for left-hand column in which dates are printed
\newcolumntype{R}{>{\raggedleft}p{1in}}

% Define 'SectionTable' environment
\newenvironment{SectionTable}[1]{
	\renewcommand*{\arraystretch}{1.7}
	\setlength{\tabcolsep}{10pt}
	\begin{longtable}{Rp{5.0in}} & #1 \\}
{\end{longtable}\vspace{-.3cm}}

% Define 'SectionTableSingleSpace' environment
\newenvironment{SectionTableSingleSpace}[1]{
	\renewcommand*{\arraystretch}{1.2}
	\setlength{\tabcolsep}{10pt}
	\begin{longtable}{Rp{5.2in}} & #1 \\[0.6em]}
{\end{longtable}\vspace{-.3cm}}

% --- Document starts here ---

\begin{document}

% --- Name and contact information ---

\begin{SectionTable}{\Huge Joosep Näks} & 
Meil: joosep.naks@gmail.com   $\;\boldsymbol{\cdot}\;$ 
Tel: +372 56915051 \newline
Sünniaeg: 04.12.1999
\end{SectionTable}

% --- Section: Education ---

\begin{SectionTable}{\headingfont Hariduskäik}
2019 -- ... & 
\textbf{Tartu Ülikool}\newline
Bakalauruseõpe informaatika erialal matemaatika kõrvalerialaga\\

2016 -- 2019 & 
\textbf{Reaalkool}\newline
Gümnaasiumiharidus\\

\end{SectionTable}

% --- Section: Oskused

\begin{SectionTable}{\headingfont Oskused}
& \textbf{Programmeerimiskeeled} \newline
Kasutan igapäevaselt: C++, Python, Java, Bash\newline
Olen tuttav: C, JavaScript, PHP, SQL\\
& \textbf{Keeled} \newline
Eesti keel (emakeelena)\newline
Inglise keel (C1 tase)
\end{SectionTable}

% --- Section: Other interests/hobbies ---

\begin{SectionTable}{\headingfont Veel tõike}

&\vspace{-3em} \begin{itemize}
\item Olen osalenud programmeerimisvõistlustel:
\begin{itemize}
\item Balti informaatika olümpiaad 2019
\item NWERC 2019
\item NWERC 2020
\item IEEEXtreme 13.0
\item IEEEXtreme 14.0
\end{itemize}

\item Olen osalenud gümnaasiumis olümpiaadide riigivoorudes ainetes: matemaatika, informaatika, füüsika, astronoomia
\item Olen tegelenud robootikaga põhikoolis alates 5. klassist
\end{itemize}
\end{SectionTable}

% --- End of CV! ---

\end{document}
