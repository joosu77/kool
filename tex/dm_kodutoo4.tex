\documentclass{article}
\usepackage{amsfonts}
\usepackage{amsmath}
\usepackage{systeme}
\usepackage{mathtools}
\usepackage{forest}
\usepackage{xcolor}
\usepackage{tikz}
\usetikzlibrary{positioning}
\DeclarePairedDelimiter\ceil{\lceil}{\rceil}
\addtolength{\oddsidemargin}{-1in}
\addtolength{\evensidemargin}{-1in}
\addtolength{\textwidth}{2in}
\begin{document}
\begin{center}
\Large\textbf{Kodut\"o\"o nr. 4}\\
Joosep N\"aks
\end{center}
\textbf{1.} On teada, et puus on $k$ sisetippu, sealjuures iga sisetipu aste on samuti $k$. Leida selle puu tippude arv, kui puus on 290 lehttippu.\\
\textbf{Lahendus:} Kuna iga puu tipp on kas sisetipp või lehttipp, siis on puu tippude arv 290+k. Kuna lehttipud on tipud, mille aste on 1, on puu tippude astmete summa $\displaystyle\sum_{v\in V} d(v) = k*k+290*1=k^2+290$. Tipuastmeteoreemi kohaselt on graafi servade arv pool tippude astmete summast ehk $|E|=\displaystyle\frac{1}{2}\sum_{v\in V} d(v)=\frac{k^2}{2}+145$. Loengukonspekti teoreem 2.73 j\"argi on igal n-tipulisel puul n-1 serva, ehk kehtib $290+k-1=\frac{k^2}{2}+145 \Leftrightarrow k^2-2k-288=0 \Leftrightarrow k_1=18, k_2=-16$. Kuna sisetippude kogus peab olema positiivne arv, ei sobi $k_2$ ehk sisetippe on 18. Seega on kokku tippe $|V|=290+18=308$.\\\\
\textbf{2.} Olgu graaf $G$ puu ja leidugu selles tipp $u$ nii, et iga tipu kaugus tipust $u$ on \"ulimalt $r$. T\~oesta, et iga lihtahela pikkus selles puus on \"ulimalt $2r$.\\
\textbf{Lahendus:} Valin kaks suvalist tippu $p$ ja $q$. Moodustan $G$ alamgraafi $G'$, mis koosneb $p$ ja $u$ vahelisest lihtahelast, mille pikkus on \"ulimalt $r$ ning $q$ ja $u$ vahelisest lihtahelast, mille pikkus on \"ulimalt $r$ (need lihtahelad leiduvad kuna on teada, et iga tipu kaugus tipust $u$ on \"ulimalt $r$). Loengukonspekti teoreemi 2.76 kohaselt on puus iga kahe tipu vahel parajasti 1 lihtahel ehk kuna saadud puus $G'$ leiduvad tipud $p$ ja $q$, siis leidub ka nende vahel lihtahel. Kuna puu $G'$ moodustati kahest \"ulimalt $r$ pikkusest ahelast, on puus $G'$ \"ulimalt $2r$ serva. Seega on ka $p$ ja $q$ vahelises lihtahelast \"ulimalt $2r$ serva. Kuna $G'$ on graafi $G$ alamgraaf, sisaldub leitud $p$ ja $q$ vaheline lihtahel ka graafis $G$. Sama teoreemi 2.76 kohaselt kuna ka $G$ on puu, siis on see ainus lihtahel graafis $G$ punktide $p$ ja $q$ vahel. Seega olen n\"aidanud, et iga kahe tipu vahel graafis $G$ leidub parajasti 1 lihtahel pikkusega \"ulimalt $2r$.\\\\
\textbf{3.} Tuua n\"aide lihtgraafist ja selle kahest toesepuust, mis ei ole isomorfsed. P\~ohjendada.\\
\textbf{Lahendus:} Toon n\"aiteks graafi $G=(\{a,b,c,d\},\{\{a,b\},\{a,c\},\{a,d\},\{b,d\}\})$. Selle toesuspuudeks on nii $P_1=(\{a,b,c,d\},\{\{a,b\},\{a,c\},\{a,d\}\}$ kui ka $P_2=(\{a,b,c,d\},\{\{a,b\},\{a,c\},\{b,d\}\}$. Need on toesepuud kuna m\~olemad on $G$ alamgraafid, sisaldavad sama palju tippe nagu $G$ ning on puud. Need ei ole isomorfsed, kuna graafis $P_1$ on \"uks tipp astmega 3 (tipp $a$) ja kolm tippu astmetega 1 kuid graafis $P_2$ on kaks tippus astmega 2 (tipud $a$ j $b$) ja kaks tippu astmega 1 (tipud $c$ ja $d$).\\\\
\textbf{4.} Olgu $G$ sidus kaalutud graaf ja serva $e$ kaal rangelt suurem iga teise serva kaalust. T\~oestada, et serv $e$ kuulub graafi $G$ mingisse minimaalse kaaluga toesepuusse parajasti siis, kui serv $e$ on sild.\\
\textbf{Lahendus:}\\
Kuna serv $e$ ei ole sild, sisaldub $e$ mingis ts\"uklis, mis koosneb servadest $u_1, u_2, ... , u_n, e$. Toesepuu moodustamiseks peab nendest v\"ahemalt \"uhe serva eemaldama, kuna toesepuud ei ole ts\"ukleid. Juhul kui k\~oik servad $u_1, u_2, ... , u_n, e$ sisalduvad vaid \"uhes ts\"uklis, peab neist t\"apselt 1 eemaldama, kuna esimese eemaldamisel muutuvad k\~oik teised neist definitsiooni j\"argi sildadeks (kuna nad ei sisaldu \"uheski ts\"uklis) ning kui veel m\~oni eemaldada, tekib mitu siduskomponenti ehk tegemist poleks enam puuga. Seega tuleks nendest eemaldada suurima kaaluga serv ehk serv $e$. juhul aga kui mingi serv $u_i$ asub kahes ts\"uklis, tuleb m\~olemast ts\"uklist m\~oni serv eemaldada, kuna kui nendest kahest ts\"uklist m\~oni serv eemaldatakse, j\"a\"ab ts\"uklitest teine alles v\~oi kui kustutada serv, mis sisaldub m\~olemas ts\"uklis, j\"ab alles ts\"ukkel, mis kasutab m\~olemast ts\"uklist neid servu, mis ei kattu teise ts\"ukli servadega. See t\"ahendab aga et ka sel juhul on vaja algsest ts\"uklist $u_1, u_2, ... , u_n, e$ kustutada v\"ahemalt \"uks serv ning kuna tahetakse saavutada v\"ahimat k\~oigi servade kaalude summat, tuleks eemaldada v\"ahima kaaluga serv ehk serv $e$. Kui serv $u_i$ sisaldub rohkemates ts\"uklites, tuleb igast ts\"uklist v\"ahemalt 1 serv kustutada ning selliseid servi $u_i$, mis sisalduvad mitmes ts\"uklis saab ka mitu olla.
\\\\
\textbf{5.} Olgu suunatud graafis $G$ t\"apselt kaks tugevalt sidusat komponenti. On teada, et graafi $G$ alusgraaf on sidus ja selles ei leidu \"uhtegi silda. T\~oestada, et leidub serv, mille suuna vahetamisel muutub graaf $G$ tugevalt sidusaks.\\
\textbf{Lahendus:}\\
Kuna on teada, et graafis $G$ sisaldub kaks tugevalt sidusat komponenti, saame \"oelda, et graafil $G$ on kaks ilma kattuvusteta alamgraafi $U$ ja $V$, mis kumbki on tugevalt sidusad ehk kahe tipu vahel leiduvad m\~olemas suunas suunatud ahelad. Kuna graafis $G$ ei leidu sildu, tähendab see definitsiooni kohaselt, et iga serv sisaldub mingis ts\"uklis ehk iga kahe tipu vahel leidub kaks mitte kattuvat lihtahelat. Valin graafist $U$ tipu $u$ ja graafist $V$ tipu $v$. Kuna graafis $G$ peab nende vahel leiduma kaks mitte kattuvat lihtahelat, peab seal leiduma v\"ahemalt kaks serva $\{a,b\}$ ja $\{c,d\}$ nii, et tipud $a$ ja $c$ sisalduvad graafis $U$ ning tipus $b$ ja $d$ sisalduvad graafis $V$. Kui serv $\{a,b\}$ on suunaga $(a,b)$ ja $\{c,d\}$ on suunaga $(d,c)$, siis on graaf $G$ juba tugevalt sidus. Selle n\"aitamiseks moodustan suvalise graafi $U$ tipu $u$ ja graafi $V$ tipu $v$ vahele m\~olemat pidi suunatud lihtahelad graafis $G$. K\~oigepealt moodustan lihtahela tipust $u$ tippu $a$ (kuna nad asuvad m\~olemad graafis $U$, mis on tugevalt sidus), siis kasutada serva $(a,b)$ ning seej\"arel moodustada lihtahel tipust $b$ tippu $v$ (see leidub kuna nad m\~olemad asuvad graafis $V$, mis on tugevalt sidus). Seega olen leidnud suunatud lihtahela tipust $u$ tippu $v$. Analoogselt saab alustades tipust $v$ moodustada suunatud lihtahela tippu $d$, kasutada serva $(d,c)$ ning seej\"arel moodustada lihtahel tipust $c$ tippu $u$. \"Ulesande p\"ustituse kohaselt aga ei ole graaf $G$ tugevalt sidus, seega on servad $\{a,b\}$ ja $\{c,d\}$ samat pidi ehk on hoopis servad $(a,b)$ ja $(c,d)$ ning ka iga teine serv graafis $G$ mis \"uhendab m\~onda tippu graafis $U$ tipuga graafis $V$ on suunatud graafist $U$ graafi $V$. Sel juhul pole graaf $G$ tugevalt sidus, kuna \"uhestki graafi $V$ tipus pole graafis $G$ v\~oimalik moodustada suunatud lihtahelat tipuni, mis asuks graafis $U$. kui aga serv $(a,b)$ v\~oi $(c,d)$ \"umber p\"o\"orata, on varem seletatud p\~ohjustel $G$ tugevalt sidus.
\end{document}