\documentclass[a4paper, 10pt]{article}
%\usepackage[estonian]{babel}
\usepackage{t1enc}
\usepackage{amsthm}
\usepackage{amscd}
\usepackage{amssymb}
\usepackage{lscape}
\usepackage{amsfonts}
\usepackage{amsmath}
\usepackage{diagbox}
\usepackage[official]{eurosym}
\usepackage{mathtools}
\usepackage{systeme}
\usepackage{polynom}
\usepackage{xcolor}
\usepackage[shortlabels]{enumitem}
\usepackage[a4paper,margin=1in,footskip=0.25in]{geometry}
\usepackage{pgffor}
\everymath{\displaystyle}
\DeclarePairedDelimiter\ceil{\lceil}{\rceil}
\newcommand{\p}[1]{\frac{\partial}{\partial #1}}
\newcommand{\Z}{\mathbb{Z}}
\newcommand{\N}{\mathbb{N}}
\newcommand{\B}{\mathbb{P}}
\newcommand{\w}{\overline}
\newcommand{\ind}{\mathrm{ind}}
\newcommand{\db}[2]{\slashbox{#1}{#2}}
\newcommand{\leg}[2]{\left(\frac{#1}{#2}\right)}
\topmargin-3em
\oddsidemargin0cm
\textwidth16cm
%\textheight27cm
\evensidemargin-2cm
\usepackage{amsmath}
\begin{document}
\begin{center}
\Large\textbf{Kodutöö nr. 16}\\
\small{Joosep Näks ja Uku Hannes Arismaa}
\end{center}


\bigskip

\noindent 1. Tõestada, et kui $n\in\N$, siis $[1,2,\ldots,n]\geq\left(\sqrt{n}\right)^{\pi(n)}.$

\bigskip
Võtame võrratuse mõlemad pooled ruutu ning parema poole jagame algteguriteks. Algtegureid on sama palju, kui on $n$-st väiksemaid algarve, kuna need kõik peavad jagama VÜKi. Iga algarvu astmete puhul peab vähiamt ühiskordset jagama mingi algarvu nii suur aste, kui ta maksimaalselt arvude 1 kuni $n$ hulgas esineb. Seega iga $p^k$ puhul $p^k\leq n$ ning $p^{k+1}>n$, kusjuures $k>0$, kuna vastasel juhul esineks vähimas ühiskordses vähmalt üks $p$ aste, mis ei jaga lõpptulemust. Seega võrratuses vastab igale $n$-le paremal pool algarvu aste vaakul pool, kusjuures algarvu aste on kujul $p^{2k}$, seega kuna $p^{2k}\geq p^{k+1}>n$, saame, et iga vasaku poole tegur on vastavast parema poole tegurist suurem, seeega on seda ka korrutis vasakul pool.

\bigskip

\noindent 2. Tõestada, et iga $m,n\in\N$ korral $(m!)^n\mid (mn)!$. 

\bigskip
Parema poole saab ümber kirjutada kui $(mn)!=\prod_{i=0}^{n-1}\prod_{j=1}^{m}(j+mi)$ ehk on kokku korrutatud $m$ järjestikust arvu $n$ korda. Iga $m$ järjestikuse arvu kokku korrutis jagub arvuga $m!$, sest selles on esimesed $m$ arvu kokku korrutatud ning kui võtta näiteks selle tegur $m$, siis iga $m$ järjestiku arvu hulgas leidub täpselt üks $m$ kordne arv, samuti iga $m-1$ järjestikuse arvu hulgas leidub üks $m-1$ kordne arv ja nii edasi ehk iga $m$ järjestikuse arvu hulgas leiduvad kõigi esimese $m$ arvu kordsed arvud. Kattuvusi tekib vaid juhul kui mingi arv $a$ on mingi teise arvu $b$ kordne, kuid siis leidub $b$ järjestikuse arvu hulgas üks $b$ kordne ning $\frac ba>1$ tükki $a$ kordseid arve ehk saab $a$ kordseks arvuks valida mõne sellise, mis ei ole $b$ kordne. Seega iga $m$ järjestikuse arvu korrutis jagub arvuga $m!$ ehk kuna $(mn)!$ sisaldab $n$ sellist korrutist, jagub see arvuga $(m!)^n$.
\bigskip

\noindent 3. Leida diofantilise võrrandi $4x+12y+16z = 2024$ positiivsete lahendite arv. 

\bigskip
Ülesanne on samaväärne ülesandega $x+3y+4z=506$. Saame lahendeid hakata loendama, kui alustame seisust, kus $x=y=z=1$, ning uurime, kui mitu väärtust saab olla $y$-l iga võimaliku z väärtuse korral. $x$ väärtus on alati eelmise kahe põhjal üheselt määratud. Kui $z=1$, siis on $y$ jaoks $\frac{506-4-3-1}{3}+1=167$ väärtust. Suurendades $z$ 1 võrra saame $\frac{498-4}{3}+1=165$ väärtust. Veel 1 võrra $z$ suurendades saame 164 väärtust. Kokku on juba 493 väärtust. Seega saime, et kui meil on $z$ vahemikus 1-4, siis on meil 496 väärtustust, mille korral saame soovitud tulemuse. suurendades $z$ võiamlike väärtuste piire 3 võrra (5-7), saame $y$ võiamlike väärtuste arvuks samad tulemused, lihtsalt 4 võrra väiksemad. Nii saame konstrueerida aritmeetilise jada, kus alguses on 1 võimalus, kui $z=124$ ning iga järgmine elemet kuni $496=4+12*41$ on eelmisest 12 võrra suurem. Matemaatika järgi on selle jada summa 10500.

\bigskip

\noindent 4. Leida kõik algarvud $p$, mille korral $p^2\mid 5^{p^2}+1$.

\bigskip
Vaatan arvu $5^{p^2}+1$ mooduli $p$ järgi. Et see saaks jaguda arvuga $p^2$, peab see olema kongruentne arvuga 0. Kui $p=5$, siis $5^{p^2}+1\equiv1\not\equiv0\pmod p$. Muudel juhtudel saab FVT põhjal \mbox{$5^{p^2}+1\equiv5^{p^2-(p-1)(p+1)}+1=5^1+1=6\pmod p$}. Ainsad algarvulised moodulid, mille järgi 6 saab olla kongruentne nulliga on 2 ja 3. Proovin need läbi: $5^{2^2}+1=626\equiv2\pmod{2^2}$ ja $5^{3^2}+1=1953126\equiv0\pmod{3^2}$ ehk ainus algarv, mille puhul jaguvus kehtib, on $p=3$.
\bigskip

\pagebreak

\noindent 5. Leida ringi $\Z_{40336800}$ kõigi selliste elementide $\overline{x}$ arv, mille korral $x^2\equiv 0\pmod{40336800}$. 

\bigskip
Peam lihtsalt näitama, kui mitme $x^2$ korral see 40336800-ga jagub. Selleks peavad $x^2$ algtegurduses leiduma need algarvud, mis on 40336800 algtegurduses (2,3,5,7), ning need peavad leiduma vähemalt selles astmes, mis need leiduvad 40336800 algetegurduses (vastavalt 5,1,2,5) ühtlasi, peavad need seega ka leiduma $x$ algtegurduses, kusjuures väiksemas astmes kui 40336800 omas. Nii saame , et $x$ algtegurduses peab olema minimaalselt $2^3$, $3$, $5$, $7^3$. Kõik selle arvu kordsed sobivad samuti $x$-ks. Neid on 40336800-st väiksemaid $\frac{40336800}{2^3\cdot5\cdot3\cdot7^3}=980$. Saimegi vastuse. 

\bigskip

\noindent 6. Lahendada diofantiline võrrand $x^3+y^4=2100$.

\bigskip
Vaatlen võrrandit mooduli 13 järgi. Kuupide ja neljandate astmete tabelid mooduli 13 järgi:\\
\begin{tabular}{c|c|c|c|c|c|c|c|c|c|c|c|c|c}
$n$&0&1&2&3&4&5&6&7&8&9&10&11&12\\
\hline
$n^3$&0&1&8&1&12&8&8&5&5&1&12&5&12\\
\hline
$n^4$&0&1&3&3&9&1&9&9&1&9&3&3&1
\end{tabular}\\
Ehk $x^3$ võimalikud väärtused on 0, 1, 3, 5, 8, 12 ning $y^4$ võimalikud väärtused on 0, 1, 3 ja 9. Arv 2100 on kongruentne arvuga 7 mooduli 13 järgi, kuid leitud $x^3$ ja $y^4$ väärtuste summana ei ole võimalik arvu 7 ega ka 7+13 saavutada (ning suurim võimalik summa on 12+9 mis on väiksem kui järgmine arvuga 7 kongruentne arv) ehk võrrandil puuduvad lahendid.
\bigskip

\noindent 7. Olgu $n\equiv -1\pmod{8}$. Tõestada, et $\sigma(n)\equiv 0 \pmod{8}$. 

\bigskip

\pagebreak

\noindent 8. Lahendada kongruents $$x^4-6x^3-7x^2+96x+6\equiv 0 \pmod{1125}.$$

\bigskip
Tegurdades saab $1125=3^2\cdot5^3$ seega vaatlen kõigepealt võrrandit moodulite 3 ja 5 järgi. $$x^4-6x^3-7x^2+96x+6\equiv 1-0-1+0+0\equiv0  \pmod{3}.$$ Ehk mooduli 3 järgi sobivad kõik lahendid. Järgmise sammu jaoks võtan algsest funktsioonist tuletise: $$f'(x)=4x^3-18x^2-14x+96$$ Otsin lahendit kujul $x=1+3y$. Kuna $f(1)=90$ ja $f'(1)=68\equiv-1\pmod3$, tuleb lahendada lineaarkongruents $$-1\cdot y+\frac{90}{3}\equiv0\pmod3$$ Selle lahendiks on $y\equiv0\pmod3$ ehk siit saab lahendi $x\equiv1\pmod9$. Järgmiseks otsin lahendit kujul $x=2+3y$. Kuna $f(2)=138$ ja $f'(1)=28\equiv1\pmod3$, tuleb lahendada lineaarkongruents $$1\cdot y+\frac{138}{3}\equiv0\pmod3$$ Selle lahendiks on $y\equiv2\pmod3$ ehk siit saab lahendi $x\equiv2+3\cdot2=8\pmod9$. Viimaks otsin lahendit kujul $x=0+3y$. Kuna $f(0)=6$ ja $f'(0)=96\equiv0\pmod3$, tuleb lahendada lineaarkongruents $$0\cdot y+\frac{6}{3}\equiv0\pmod3$$ Sellel puuduvad lahendid ehk siit lahendeid juurde ei tule.

Nüüd vaatan võrrandit mooduli 5 järgi:$$x^4-6x^3-7x^2+96x+6\equiv -x^3-2x^2+x+2  \pmod{5}.$$ Läbi proovides saab, et selle lahendid on 1, 3 ja 4. Leian järgmiseks lahendid mooduli 25 järgi. Otsin lahendit kujul $x=1+5y$. Kuna $f(1)=90$ ja $f'(1)=68\equiv3\pmod5$, tuleb lahendada lineaarkongruents $$3\cdot y+\frac{90}{5}\equiv0\pmod5$$ Selle lahendiks on $y\equiv4\pmod5$ ehk siit saab lahendi $x\equiv1+4\cdot5=21\pmod{25}$.

Järgmiseks otsin lahendit kujul $x=3+5y$. Kuna $f(3)=150$ ja $f'(3)=0$, tuleb lahendada lineaarkongruents $$0\cdot y+\frac{150}{5}\equiv0\pmod5$$ See võrrand kehtib $y$ väärtusest sõltumatult ehk kõik lahendid kujul $z\in\{3,8,13,18,23\}, x\equiv z\pmod{25}$ kehtivad.

Viimaks otsin lahendit kujul $x=4+5y$. Kuna $f(4)=150$ ja $f'(4)=8\equiv3\pmod5$, tuleb lahendada lineaarkongruents $$3\cdot y+\frac{150}{5}\equiv0\pmod5$$ Selle ainsaks lahendiks on $y\equiv0\pmod5$ ehk siit saab lahendi $x\equiv4\pmod{25}$.

Seega tulid mooduli 25 järgi lahendid $21,3,8,13,18,23,4$. Vaatan ka lahendeid mooduli 125 järgi. Kõigepealt otsin lahendit kujul $x=21+25y$. Kuna $\frac{f(21)}{5^2}=5514\equiv4\pmod5$ ja $f'(21)=28908\equiv3\pmod5$, tuleb lahendada kongruents $3y+4\equiv0\pmod5$, mille ainsaks lahendiks on $y\equiv2\pmod5$ ehk $x\equiv71\pmod{125}$.

Järgmiseks otsin lahendit kujul $x=z+25y$, kus $z\in\{3,8,13,18,23\}$. Kuna iga $z$ väärtuse puhul $f'(z)\equiv 0\pmod5$, saab siin olla lahendeid vaid juhul, kui kehtib $\frac{f(z)}{5^2}\equiv0\pmod5$. Vaatlen neid:

\begin{tabular}{c|c|c|c|c|c}
$z$&3&8&13&18&23\\
\hline
$\frac{f(z)}{25}$&6&54&618&2778&238214\\
\end{tabular}

Ükski saadud arvudest ei jagu arvuga 5 ehk siit lahendeid ei tule. Otsin veel lahendit kujul $x=4+25y$. Kuna $\frac{f(4)}{5^2}=6\equiv1\pmod5$ ja $f'(4)=8\equiv3\pmod5$, tuleb lahendada kongruents $3y+1\equiv0\pmod5$, mille ainsaks lahendiks on $y\equiv3\pmod5$ ehk $x\equiv79\pmod{125}$.

Seega olen leidnud lahendid $x\equiv8\pmod9$ ja $x\equiv1\pmod9$ ning $x\equiv71\pmod{125}$ ja $x\equiv79\pmod{125}$. Kuna ringis $\Z_9$ on $\w{125}^{-1}=\w8$ ja ringis $\Z_{125}$ on $\w9^{-1}=14$, saab HJT põhjal lahendid kombineerida pannes need sisse valemisse $x=a\cdot125\cdot8+b\cdot9\cdot14$, kus $a$ on lahend mooduli 9 järgi ning $b$ on lahend mooduli 125 järgi. Kui nii lahendid välja arvutada, saab et mooduli 1125 järgi on lahenditeks 71, 829, 946 ja 1079.
\bigskip

\noindent 9. Teha kindlaks, kas mooduli $n$ j\"argi leidub algjuuri ning 
kui leidub, siis leida nende arv ja \"uks algjuur, kui \\
{${}_{}$}\hskip1.5cm
 a) $n=2661$, \hskip0.5cm b) $n=2662$, \hskip0.5cm c) $n=2663$, \hskip0.5cm d) $n=2664$.
 
\bigskip
a) $2661=3\cdot887$, seega algjuuri ei leidu.

b) $2662=2\cdot11^3$, seega algjuuri leidub. 

c) 2663 on algarv ehk algjuuri leidub.

d) $2664=4\cdot 666$, seega algjuuri ei leidu.


\bigskip
\noindent 10. Leida kõik algarvud $p$, mille järgi eksisteerib täpselt 32 algjuurt.

\bigskip
Algarvul $p$ on $\varphi(\varphi(p))=\varphi(p-1)$ algjuurt ehk leian kõik arvud $n$, mille puhul $\varphi(n)=32$ ning kontrollin kas $n+1$ on algarv. Kui arvu $n$ algtegurdus on $n=q_1^{k_1}q_2^{k_2}..q_s^{k_s}$, siis $\varphi(n)=q_1^{k_1-1}(q_1-1)\cdot q_2^{k_2-1}(q_2-1)\cdot ..\cdot q_s^{k_s-1}(q_s-1)$. Kuna $32=2^5$, ei jaga ükski kahest suurem algarv arvu 32 ehk arvu $n$ tegurites saavad kahest suuremad arvud olla ülimalt astendajaga 1. Kuna $32$ tegurid on $2,4,8,16,32$, saaksid $n$ algtegurite hulgas olla arvud $2+1=3, 4+1=5, 8+1=9, 16+1=17, 32+1=33$, kuid nendest sobivad vaid $3,5,17$ kuna teised pole algarvud. Leian kõik võimalikud $n$ väärtused (2 on alati tegurite hulgas kuna $n+1$ peab lõpuks algarv olema ehk $n$ on kas paarisarv või 1 ning $\varphi(1)\neq32$):\\
\begin{tabular}{cccc|c|c}
\multicolumn{4}{c|}{Binaararv näitamaks, kas arv on tegurite hulgas} & $\varphi(n)$ & $n$\\
\hline
2&3&5&17&\\
\hline
\hline
1&0&0&0&$2^{k-1}=32$&$64$\\
\hline
1&0&0&1&$2^{k-1}(17-1)=32$&$68$\\
\hline
1&0&1&0&$2^{k-1}(5-1)=32$&$80$\\
\hline
1&0&1&1&$2^{k-1}(5-1)(17-1)=32$&ei leidu\\
\hline
1&1&0&0&$2^{k-1}(3-1)=32$&$96$\\
\hline
1&1&0&1&$2^{k-1}(3-1)(17-1)=32$&$102$\\
\hline
1&1&1&0&$2^{k-1}(3-1)(5-1)=32$&$120$\\
\hline
1&1&1&1&$2^{k-1}(3-1)(5-1)(17-1)=32$&ei leidu\\
\end{tabular}\\
Leitud $n$ väärtustest on vaid 96 ja 102 sellised, et $n+1$ oleks algarv ehk sobivad algarvud on 97 ja 103.
\bigskip

\noindent 11. Tõestada, et iga naturaalarvude paari $(a,b)$ korral leidub naturaalarv $n$ nii, et arvul $n^2+an+b$ on vähemalt 2021 jagajat. (Vihje: $n^2+an+b \equiv 0\pmod{p}\Longleftrightarrow a^2-4b$ on ruutjääk mooduli $p$ järgi.)

\bigskip
 Kui $ a^2-4b$ on ruutjääk mooduli $p$ järgi, ning $p>2$, siis saame leida $n=(-a\pm\sqrt{a^2-4b} )2^{-1}$, mille asendades valemisee $n^2+an+b$, saame, et $n^2+an+b\equiv0\pmod{p}$. Nüüd märkame ,et tähistades ümber $x:=a^2-4b$, siis teame, et algarve kujul $1+4x$ on lõpmata palju. Siit teame, et need algarvud annavad nii $x$ kui ka 4 järgi jäägi 1. Lahutades $x$ algteguriteks saame $(\frac{x}{p})=(\frac{q_1}{p})\cdot...\cdot(\frac{q_s}{p})$. Kui mõni neist algteguritest on $p$, siis saame korrutiseks 0, mis tähendaks, et $\sqrt{x}\equiv0\pmod{p}$, mis $n$ leidmiseks mooduli $p$ järgi on piisav. Kuna $p\equiv1\pmod{4}$, saame need legendre'i sümbolid ümber pöörata, saades korrutise $(\frac{p}{q_!})\cdot...\cdot(\frac{p}{q_s})$. Kuna $p\equiv 1\pmod{x}$, siis peab ka iga $x$ algteguri korral kehtima, et  $p\equiv 1\pmod{q}$. Seega saame enda korrutises kõik Legendre'i sümbolid väärtustada ühtedega saades korrutiseks ühe. Seega on $x$ ruutjääk mooduli $p$ järgi siis, kui $p$ on kujul $1+4x$. Nüüd peame lihtsalt valima 2021 sellist $p$, leidma iga ühe järgi  ühe sobiva $n$ ning need HJT järgi kokku panema üheks suureks $n$-ks, mille järgi saabki arvutada sobiva $n^2+an+b$.

\bigskip
\pagebreak


\noindent 12. Kasutades loengukonspekti näites 9.8 toodud skeemi kaheksatäheliste (st. kuueteist\-numbriliste) blokkide jaoks ja mooduli väiksust, dekodeerida avaliku võtmega $(9591149766518863,2021)$ kodeeritud RSA sõnum $$3358877268877051 8580786695769838 8157730396723143.$$ 

\bigskip
Avalikus võtmes oleva $n$ lahti tegurdades saan $n=9591149766518863=97435691\cdot98435693$ ehk $\varphi(n)=(97435691-1)\cdot(98435693-1)=9591149570647480$. Kuna salajane astendaja on avaliku astendaja pöördarv mooduli $\varphi(n)$, saan leida salajase astendaja: $d\equiv2021^{-1}\equiv2026432888009141\pmod{9591149570647480}$. Lõpuks jagan sõnumi blokkideks ja astendan need salajase astendajaga mooduli $n$ järgi:
\begin{gather*}
3358877268877051^{2026432888009141}\equiv12\ 15\ 16\ 16\ 00\ 08\ 05\ 01\pmod{9591149766518863}\\
8580786695769838^{2026432888009141}\equiv00\ 11\ 15\ 09\ 11\ 00\ 08\ 05\pmod{9591149766518863}\\
8157730396723143^{2026432888009141}\equiv01\ 00\ 05\ 11\ 19\ 15\ 12\ 05\pmod{9591149766518863}\\
\end{gather*}
Ehk sõnumi on "lopp hea koik hea eksole".
\bigskip

\end{document}