\documentclass[a4paper, 10pt]{article}
\usepackage[estonian]{babel}
\usepackage{t1enc}
\usepackage{amsthm}
\usepackage{amscd}
\usepackage{amssymb}
\usepackage{lscape}
\usepackage{amsfonts}
\usepackage{amsmath}
\usepackage{mathtools}
\usepackage{systeme}
\usepackage{polynom}
\usepackage{pgfplots}
\usepackage[shortlabels]{enumitem}
\usepackage[a4paper,margin=1in,footskip=0.25in]{geometry}
\everymath{\displaystyle}
\DeclarePairedDelimiter\ceil{\lceil}{\rceil}
\newcommand{\p}[1]{\frac{\partial}{\partial #1}}
\newcommand{\Z}{\mathbb{Z}}
\newcommand{\N}{\mathbb{N}}
\topmargin-3em
\oddsidemargin0cm
\textwidth16cm
%\textheight27cm
\evensidemargin-2cm
\begin{document}
\begin{center}
\Large\textbf{Kodutöö nr. 3}\\
\small{Joosep Näks ja Uku Hannes Arismaa}
\end{center}

\bigskip

\noindent 1. Leida k\~oik algarvud lõigus $[360, 460]$. Põhjendada, et saadud arvud on tõesti algarvud ja ükski ei ole puudu. Kas selles lõigus esineb alg\-arvukaksikuid?

\bigskip

\noindent 2. Olgu $n\in\N$ selline, et ülimalt üks selle {\bf erinevatest} algteguritest $p$ rahuldab tingimust $p\leq \sqrt[4]{n}$. (St võib olla, et ükski algtegur seda tingimust ei rahulda, või ongi ainult üks, aga mitmekordne algtegur, mis seda rahuldab.) Leida kõik erinevad võimalused arvu $n$ algteguriteks lahutada (st üldkujud \'a la $p^kq$) ja tuua iga juhu kohta arvuline näide. 

\bigskip

On ilmne, et kõik üldkujud $n=p^k$ (näide: $n=32=2^5$) ja $n=1$ sobivad, kuna neis ei ole rohkem kui üks erinev algtegur. Sobivad ka kindlad arvud üldkujuga $n=p^k\cdot q$ (näide: $n=48=2^4\cdot3$, kus $2^4=16<48<81=3^4$), arvud $n=p^k\cdot q\cdot w$ (näiteks: $n=60=2^2\cdot3\cdot5$, kus $2^4=16<60<81=3^4$) ja $n=p^k\cdot q\cdot w\cdot r$ (näiteks: $n=490=2\cdot5\cdot7\cdot7$, kus $2^4=16<490<625=5^4$). (Mainitud kujudes $q$, $w$ ja $r$ võivad ka võrdsed olla)\\
Üle kolme algteguri, mis oleks suurem kui $\sqrt[4]n$, ei saa arvul $n$ olla, kuna sel juhul kehtiks $n=p^k\cdot q\cdot w\cdot r\cdot m>p^k\cdot\sqrt[4]n\cdot\sqrt[4]n\cdot\sqrt[4]n\cdot\sqrt[4]n=p^k\cdot n$ ehk isegi kui $k=0$, siis jääb võrratuseks $n>n$, mis ei ole tõene.

\bigskip

\noindent 3. Tõestada, et kui $n!+n^2+1$ on algarv, siis ka $n^2+1$ on algarv. 

\bigskip

Kui $n^2+1$ on kordarv, saab teda tegurdada $n^2+1=p\cdot a$, kus $p$ on algarv, nii et $p\leq\sqrt{n^2+1}$ (kui arvul oleks mitu algarvulist tegurit, mis oleks suurem kui arvu juur, tuleks nende korrutis arvust suurem, ning kui arvul on vaid üks algarvuline tegur, on ta algarv). Kuna $n^2$ on täisruut ja ühegi kahe täisruudu vahe ei ole 1 ega vähem, kehtib ka $p\leq n$ ehk $p|n!$. Seega kuna $p$ jagab nii $n!$ kui ka $n^2+1$, jagab ta ka $n!+n^2+1$, ehk $n!+n^2+1$ ei saa olla algarv.\\
Seega kui $n!+n^2+1$ on algarv, peab ka $n^2+1$ olema algarv.

\bigskip

\noindent 4. Tõestada, et kahe järjestikuse paaritu algarvu summal on alati vähemalt kolm (mitte tingimata erinevat) algtegurit. 

\bigskip

\noindent 5. T\~oestada ilma Dirichlet' teoreemi kasutamata, et leidub l\~opmata palju algarve, millel on kuju $6k+5$, kus $k\in \N$. 

\bigskip

\noindent 6. Tõestada, et iga naturaalarvu on võimalik üles kirjutada summana, mille liidetavad on kõik {\bf erinevad} ja kas algarvud või arv 1. Summa võib koosneda ka ühestainsast liidetavast. 

\bigskip

\noindent 7. Leida kõik algarvukolmikud $(p,q,r)$, mille korral $pq + pr + qr > pqr$. 

\bigskip

\noindent 8. Tõestada, et aritmeetilises jadas vahega $b<2021$ ei saa olla 12 järjestikust algarvu. 

\bigskip

Et aritmeetilises jadas vahega $b$ oleks $n$ järjestikust algarvu, peab kehtima $(b,n)>1$, kuna vastasel juhul kui $(b,n)=1$ ja mingi jada liige $a_i$ on algarv, siis jäägiga jagades saan $a_i=q_1\cdot n+r_1,\ r_1<n$. Kuna $(b,n)=1$, leiduvad $x$ ja $y$ nii, et $xb+yn=1$, korrutan selle $r_1$ga: $xbr_1+ynr_1=r_1$. Teostan veel kord jäägiga jagamist: $xr_1=n\cdot q_2+r_2,\ r_2<n$. Need kokku pannes saab, et $n|r_1-xbr_1+b(xr_1-r_2)=r_1-br_2$ ehk $n|a_i-r_1+r_1-br_2=a_i-br_2=a_{i-r_2}$, samuti $n|a_i-br_2+bn=a_{i-r_2+n}$. Ehk iga algarvulise liikme ümber leiduvad liikmed, mis jaguvad arvuga $n$, mille vahele jääb vaid $n-1$ liiget.\\
\indent Seega selleks, et mingi $b$ korral saaks 12 järjestikust arvu leiduda, peab kehtima $(b,12)>1$, kuid kuna seal peab sisalduma ka 11, 10, .., 2 järjestikust algarvu, peab $b$ suurim ühistegur kõigi arvudega 2, .., 12 olema suurem kui 1. Selleks peab $b$ vähemalt olema sama suur, nagu seal vahemikus paiknevate algarvude korrutis, kuna kui $b$ suurim ühistegur mingi algarvuga on suurem kui 1, peab see olema võrdne selle algarvuga, kuna algarvul pole rohkem tegureid kui 1 ja tema ise. Nende algarvude korrutis on $2\cdot3\cdot5\cdot7\cdot11=2310>2021$, ehk 2021 sammuga aritmeetilises jadas ei saa leiduda 12 järjestikust algarvu.

\end{document}