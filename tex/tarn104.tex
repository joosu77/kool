\documentclass{article}
\usepackage{amsfonts}
\usepackage{amsmath}
\usepackage{mathtools}
\DeclarePairedDelimiter\bfloor{\Bigl\lfloor}{\Bigr\rfloor}
\DeclarePairedDelimiter\floor{\lfloor}{\rfloor}
\begin{document}
\begin{center}
\Large\textbf{T\"arn\"ulesanne nr. 104}\\
\small{Joosep N\"aks}
\end{center}
Olgu funktsioon $f$ pidevalt diferentseeruv vahemikus $(a,b)$. Kas iga $c\in(a,b)$ korral leiduvad $x_1,x_2\in(a,b)$ nii, et $x_1<c<x_2$ ja $\displaystyle\frac{f(x_2)-f(x_1)}{x_2-x_1}=f'(c)$?\\
\textbf{Lahendus:}\\
Vaatlen funktsiooni $f(x)=x^3$. Tuletise definitsiooni j\"argi on selle funktsiooni tuletis:
\begin{equation*}
f'(x)=\lim_{x\to a}\frac{f(x)-f(a)}{x-a}=\lim_{x\to a}\frac{x^3-a^3}{x-a}=\lim_{x\to a}\frac{(x-a)(x^2+xa+a^2)}{x-a}=\lim_{x\to a}x^2+xa+a^2=3x^2
\end{equation*}
See on pidev reaalarvude hulgas seega on $f$ diferentseeruv reaalarvude hulgas. V\~otan $c=0$, sel juhul $f'(c)=3*0^2=0$. Selleks et leida $x_1$ ja $x_2$, mille puhul kehtiks $\displaystyle\frac{f(x_2)-f(x_1)}{x_2-x_1}=f'(c)=0$, peab kehtima $f(x_1)-f(x_2)=0$ ehk $f(x_1)=f(x_2)$. Kuid kui $x_1<c=0$, siis ka $x_1^3<0$ ehk $f(x_1)<0$ ning kui $x_2>c=0$ siis ka $x_2^3>0$ ehk $f(x_2)>0$. Seega ei saa $f(x_1)=f(x_2)$ kunagi kehtida.\\
Kokkuv\~ottes ei, iga $c$ korral ei saa vastavaid $x_1$ ja $x_2$ v\"a\"artuseid leida.
\end{document}