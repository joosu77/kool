\documentclass[a4paper, 10pt]{article}
\usepackage[estonian]{babel}
\usepackage{t1enc}
\usepackage{amsthm}
\usepackage{amscd}
\usepackage{amssymb}
\usepackage{lscape}
\usepackage{amsfonts}
\usepackage{amsmath}
\usepackage{diagbox}
\usepackage[official]{eurosym}
\usepackage{mathtools}
\usepackage{systeme}
\usepackage{polynom}
\usepackage{xcolor}
\usepackage[shortlabels]{enumitem}
\usepackage[a4paper,margin=1in,footskip=0.25in]{geometry}
\usepackage{pgffor}
\everymath{\displaystyle}
\DeclarePairedDelimiter\ceil{\lceil}{\rceil}
\newcommand{\p}[1]{\frac{\partial}{\partial #1}}
\newcommand{\Z}{\mathbb{Z}}
\newcommand{\N}{\mathbb{N}}
\newcommand{\B}{\mathbb{P}}
\newcommand{\w}{\overline}
\newcommand{\ind}{\mathrm{ind}}
\newcommand{\db}[2]{\slashbox{#1}{#2}}
\topmargin-3em
\oddsidemargin0cm
\textwidth16cm
%\textheight27cm
\evensidemargin-2cm
\begin{document}
\begin{center}
\Large\textbf{Kodutöö nr. 12}\\
\small{Joosep Näks ja Uku Hannes Arismaa}
\end{center}

\bigskip

\bigskip

\noindent 1. Leida arvu $9$ indeks \emph{kõigil} võimalikel alustel mooduli 22 järgi. 

\bigskip 
Märkame, et 7 on algjuur, kuna $7^2\equiv5\pmod{22}$ ning $7^5\equiv -1\pmod{22}$. Leiame, et $\ind_79=8$. Teoreemist 7.32 saame, et $\ind_\alpha9\equiv\ind_79\cdot\ind_\alpha7\pmod{\varphi(22)=10}$ ning $\ind_7\alpha\cdot\ind_\alpha7\equiv1\pmod{10}$, seega $\ind_\alpha9\equiv\ind_79\cdot(\ind_7\alpha)^{-1}\pmod{10}$. $\ind_7\alpha$ väärtused, mille korral $\alpha$ on algjuur on 3, 7, 9, millele vastavad $\alpha$ väätused 13, 17, 19. Nende arvude pöördelemendid on vastavalt 7, 3, 9. 

$\ind_{13}9\equiv\ind_79\cdot7\equiv 6\pmod{10}$

$\ind_{17}9\equiv\ind_79\cdot3\equiv 4\pmod{10}$

$\ind_{19}9\equiv\ind_79\cdot9\equiv 2\pmod{10}$
\bigskip

\noindent 2. Koostada indeksite tabel alusel 3 mooduli 25 järgi. 

\bigskip
Kontrollin esiteks, et 3 on algjuur. Kuna $\varphi(25)=20=2^2\cdot5$, on vaja kontrollida, et $3^4$ ja $3^10$ ei oleks kongruentsed ühega mooduli 25 järgi ning esimene on kongruentne arvuga 6 ja teine arvuga $-1$ ehk tõepoolest on tegu algjuurega. Tabeli moodustamiseks leian väärtused $3^1$ kuni $3^{20}$:\\\\
\begin{tabular}{c|ccccccccccccccccccccc}
k&1&2&3&4&5&6&7&8&9&10&11&12&13&14&15&16&17&18&19&20\\
\hline
$\w{3^k}$&3&9&2&6&18&4&12&11&8&24&22&16&23&19&7&21&13&14&17&1
\end{tabular}\\\\
Ning selle põhjal saab indeksite tabeli:\\\\
\begin{tabular}{|c|cccccccccc|}
\hline
&0&1&2&3&4&5&6&7&8&9\\
\hline
0&&20&3&1&6&&4&15&9&2\\
1&&8&7&17&18&&12&19&5&14\\
2&&16&11&13&10&&&&&\\
\hline
\end{tabular}
\bigskip

\noindent 3. Leida  $\mathrm{ind}_{2}3$ mooduli 25 järgi. Kasutades saadud indeksit, koostada indeksite tabel alusel 2 mooduli 25 järgi.

\bigskip
$\mathrm{ind}_{2}3=7$

Teoreemi 7.32 põhjal saame, et $\ind_2b\equiv\ind_3b\cdot \ind_23\pmod{20}$. Siit saame uue indeksite tabeli

\begin{tabular}{|c|cccccccccc|}
\hline
&0&1&2&3&4&5&6&7&8&9\\
\hline
0&&20&1&7&2&&8&5&3&14\\
1&&16&9&19&6&&4&13&15&18\\
2&&12&17&11&10&&&&&\\
\hline
\end{tabular}
\bigskip

\noindent 4. Leida kõigi rühma $U(\Z_{37})$ elementide järgud ja kõik algjuured mooduli 37 järgi indeksite tabeli abil. Kontrollida vastust ilma astendamist kasutamata. 

\bigskip
Leian kõigepealt algjuure mooduli 37 järgi. Kuna $\varphi(37)=36=6^2$, peab vaid kontrollima, kas $a^6$ on kongruentne arvuga 1, et teada saada, kas $a$ on algjuur. Proovin $a=2$, sel juhul $a^6\equiv-10\not\equiv1\pmod{37}$ ehk 2 on algjuur. Et saada indeksid, leian väärtused $2^1$ kuni $2^{18}$ (teine pool astmetest on samade arvude vastandarvud ehk neid pole vaja välja kirjutada):\\
\begin{tabular}{c|ccccccccccccccccccccc}
k&1&2&3&4&5&6&7&8&9&10&11&12&13&14&15&16&17&18\\
\hline
$\w{2^k}$&2&4&8&16&32&27&17&34&31&25&13&26&15&30&23&9&18&36\\
\end{tabular}\\\\
Arvestades, et teoreemi 7.36 põhjal on elemendi $b$ järk $\frac{36}{(\ind_2 b,36)}$ ehk saan moodustada leitud astmete põhjal indeksite ja järkude tabeli:\\\\
\begin{tabular}{|c|c|c|c|c|c|c|c|c|c|c|}
\hline
\db{ind}{järk}&0&1&2&3&4&5&6&7&8&9\\
\hline
0&&\db{36}{1}&\db{1}{36}&\db{26}{18}&\db{2}{1}&\db{23}{36}&\db{27}{4}&\db{32}{9}&\db{3}{12}&\db{16}{9}\\
\hline
1&\db{24}{3}&\db{30}{6}&\db{28}{9}&\db{11}{36}&\db{33}{12}&\db{13}{36}&\db{4}{9}&\db{7}{36}&\db{17}{36}&\db{35}{36}\\
\hline
2&\db{25}{36}&\db{22}{18}&\db{31}{36}&\db{15}{12}&\db{29}{36}&\db{10}{18}&\db{12}{3}&\db{6}{6}&\db{34}{18}&\db{21}{12}\\
\hline
3&\db{14}{18}&\db{9}{4}&\db{5}{36}&\db{20}{9}&\db{8}{9}&\db{19}{36}&\db{18}{2}&&&\\
\hline
\end{tabular}\\\\
Algjuured on arvud, mille järk on 36 ehk 2, 5, 13, 15, 17, 18, 19, 20, 22, 24, 32 ja 35.

Paneme tähele, et igat järku arve on niipalju, kui neid peab olema.
\bigskip

\noindent 5. Lahendada kongruents $2020\cdot x^{2022}\equiv 2033\pmod{37}$ indeksite tabeli abil.  

\bigskip
Saame kongruentsi viia kujule $22x^{2022}\equiv35\pmod{37}$. Indekiste tabelist leiame 22 pöördelemendi 32, saades kongruentsi $x^{2022}\equiv10\pmod{37}$. 

Teeme kontrolli, kas see on lahenduv: $10^{36/(2022,36)}=10^6\equiv 1\pmod{37}$. Siit saame muuhulgas, et peaksime leidme 6 lahendit.

Lähme üle indeks kujule: $2022\ind_2x\equiv6\ind_2x\equiv\ind_210=24\pmod{36}$ Taandades kõiki osi 6ga, saame $\ind_2x\equiv4\pmod{6}$, seega mooduli 36 järgi sobivad astmeteks 4, 10, 16, 22, 28 ning 34. Indeksite tebalist saame esialgseteks x väärtusteks 16, 25, 9, 21, 12 ning 28.



\bigskip

\noindent 6. Milline kongruentsidest $12^{x^3}\equiv 3^2\pmod{25}$ ja $13^{x^2}\equiv 2^4\pmod{25}$ on lahen\-duv? Leida selle üldlahend.  

\bigskip
Vaatlen esimest võrrandit. Teisest ülesandest näen, et 3 on algjuur mooduli 25 järgi. Seega lause 7.30 järgi saan võrrandi mõlemast poolest võtta indeksi aluse 3 järgi ning tulemuseks on samaväärne võrrand mooduli $\varphi(25)=20$ järgi: $\ind_{3}12^{x^3}\equiv\ind_{3}3^2\pmod{20}$. Indeksi omaduste järgi saab astme indeksi ette tuua: $x^3\cdot\ind_{3}12\equiv2\cdot\ind_{3}3\pmod{20}$. Teise ülesande tabelist näen, et arvu 12 indeks on 7 ja arvu 3 indeks on 1 ehk võrrandist jääb alles $7x^3\equiv2\pmod{20}$. Kuna $4\mid 20$, on kõik lahendid ka lahendid mooduli 4 järgi, kuid proovides läbi kõik $x$ väärtused mooduli 4 järgi on näha, et ükski lahend ei sobi, ehk ka algsel võrrandil puuduvad lahendid.

Vaatlen teist võrrandit. Võtan jällegi mõlemast poolest indeksi alusel 3: $\ind_{3}13^{x^2}\equiv\ind_{3}2^4\pmod{20}$ ehk kasutades indeksi omadusi ja teise ülesande tabelit saab $x^2\cdot 17\equiv4\cdot3\pmod{20}$. Korrutan mõlemad pooled läbi arvuga $-7$: $x^2\equiv-4\pmod{20}$. Tegurdades saan $20=2^2\cdot5$ ehk saan võrrandi jagada võrrandisüsteemiks $$\begin{cases}x^2\equiv-4\pmod{4}\\x^2\equiv-4\pmod{5}\end{cases}$$ Läbi proovides saan esimese võrrandi lahenditeks $x\equiv0\pmod4$ ja $x\equiv2\pmod4$ ehk $x\equiv0\pmod2$ ning teise võrrandi lahenditeks $x\equiv1\pmod5$ ja $x\equiv-1\pmod5$. Seega on HJT põhjal esialgse võrrandi lahendid $x\equiv6\pmod{10}$ ja $x\equiv 4\pmod{10}$.
\bigskip

\noindent 7. Leida, milliste arvude $1\leq a\leq 27$ korral on kongruents $x^{15}\equiv a\pmod{p}$ lahenduv korraga \emph{kõigi} moodulite $p=7,13,27$ järgi.

\bigskip
Vaatame kõigepealt, milliste $a$ väärtuste korral on asi lahenduv, kui $p=7$. Ilmselgelt sobivad 7, 14, 21, millele sobib lahendiks $x\equiv0\pmod{7}$, muude arvude puhul on oluline, et $a^{6/(15,6)}=a^2\equiv1\pmod{7}$. Teame, et selliseid arve on 2 ning need on 1 ja -1. Seega mooduli 7 järgi on olukord lahenduv $a$ väärtuste 6, 7, 8, 13, 14, 15, 20, 21, 22 järgi. 

Nüüd $p=13$. Eelmisest on triviaalne sobivus ainult 13 korral. Järeldusest 7.34 saame seekord $a^{12/(15,12)}=a^4\equiv1\pmod{13}$, mille lehanditeks on 1, -1, 5, -5. Nüüd jäävad alles $a$ väärtused 8, 13, 14, 21.

$p=27$. 8, 13 ning 14 puhul saame kasutada sama võtet, mis enne. $a^{18/(15,18)}=a^6\equiv{1}\pmod{27}$ Arvutades saame, et see väide kehtib ainult $a=8$ korral. 21 puhul saame kongruentsi $x^{15}\equiv21\Leftrightarrow x^{15}-6\equiv0\pmod{27}$. Saame, et $x\equiv 0 \pmod{3}$. Järgmise 3 astme järgi sobivate kandidaatide leidmiseks peame lahendama kongruentsi $\frac{-6}{3}\equiv0\pmod{3}$, mis pole lahenduv, seega pole seda ka esialgne kongruents. 

Seega sobib ainult $a=8$.
\bigskip

\noindent 8. Olgu $p>2$ algarv ja $a\in\Z$. Leida kongruentsi $x^{12}=a \pmod{p}$ kõik võimalikud lahendite arvud ja tuua iga juhu kohta näide, mis seda realiseerib. 

\bigskip
Juhul, kui $a=0$, on alati 1 lahend, milleks on $x=0$, kuna $\Z_p$ kõik elemendid peale $\w0$ on pööratavad ning pööratavate elementide korrutamisel saab tulemuseks alati pööratava elemendi. Muudel juhtudel pole $x=0$ kunagi lahend, kuna see annab iga mooduli järgi tulemuseks 0, seega saame edasi vaadata vaid pööratavaid $x$ ja $a$ väärtuseid.

Kui $x_0$ on mingi $x^{12}\equiv a\pmod p$ lahend, on $x_0^{-1}$ võrrandi $x^{12}\equiv a^{-1}\pmod p$ lahend. Kui võrrandil $x^{12}\equiv a\pmod p$ leidub veel lahendeid, saab iga sellise lahendi $x'$ kohta ühe võrrandi $x^{12}\equiv1\pmod p$ lahendi kuna kui vastavatesse võrranditesse lahendid $x'$ ja $x_0^{-1}$ sisse panna ning võrrandite pooled kokku korrutada saab $(x')^{12}\cdot{(x_0^{-1})^{12}}\equiv a\cdot a^{-1}\pmod p$ ehk $(x'\cdot x_0^{-1})^{12}\equiv1\pmod p$. Seega on iga lahendi $x'$ kohta $a=1$ võrrandi lahend $x'\cdot x_0^{-1}$ ($x'$ saab ka olla sama mis $x_0$, mis puhul tuleb lahend $x_0\cdot x_0^{-1}=1$). Sellised saadud lahendid on kõik erinevad kuna kui leidub kaks $x'$ väärtust $x_1'$ ja $x_2'$, mis annavad sama $a=1$ võrrandi lahendi, siis kehtib $x_1'\cdot x_0^{-1}=x_2'\cdot x_0^{-1}$ ning kui mõlemad pooled arvuga $x_0$ läbi korrutada, saab $x_1'=x_2'$ ehk need on samad arvud. Seega on võrrandi $x^{12}\equiv1\pmod p$ lahendite hulk vähemalt samasuur nagu ühegi võrrandi $x^{12}\equiv a\pmod p$ lahendite hulk.

Väide töötab ka vastupidi, kui võrrandil $x^{12}\equiv a\pmod p$ leidub mingi lahend $x_0$, saab iga võrrandi $x^{12}\equiv1\pmod p$ lahendi $x'$ kohta ühe $x^{12}\equiv a\pmod p$ lahendi kujul $x'\cdot x_0$ kuna lahendid sisse pannes ja võrrandite pooled kokku korrutades saab $(x')^{12}\cdot (x_0)^{12}\equiv 1\cdot a\pmod p$ ehk $(x'\cdot x_0)^{12}\equiv a\pmod p$. Seega kui võrrandil $x^{12}\equiv a\pmod p$ leidub lahendeid, on neid täpselt samapalju nagu võrrandil $x^{12}\equiv 1\pmod p$.

On ka võimalus et võrrandil $x^{12}\equiv a\pmod p$ ei ole lahendeid, näiteks võrrandil $x^{12}\equiv2\pmod 3$ saab väärtuseid läbi proovides et lahendid puuduvad.

Uurin võrrandi $x^{12}\equiv1\pmod p$ lahendite kogust. Teoreemi 7.6 põhjal on selle võrrandi lahendid arvud, mille järk on arvu 12 tegur ehk 1, 2, 3, 4, 6 või 12. Kui mingi tegur $d$ ei jaga arvu $p-1$, ei leidu Lagrange'i teoreemi tõttu ühtegi seda järku elementi. Kui aga leidub, seda järku elemendid teoreemis 7.12 tõestatud võrduse 25 põhjal $\{c^k\mid 1\leq k\leq d, (k,d)=1\}$, kus $c$ on mingi algjuur, ning selliste elementide kogus on $\varphi(d)$.

Seega suurim võimalik lahendite kogus on võrrandil $x^{12}\equiv1\pmod p$, kus $12\mid p-1$ ehk näiteks $p=13$ ning lahendite kogus sel juhul on teoreemis 7.12 tõestatud võrduse 24 põhjal $\sum_{d|12}\varphi(d)=12$.

Teised võimalikud lahendite kogused sellisel kujul võrrandite puhul on kõigepealt moodulid $p$, mille puhul $p-1$ jagub arvuga 6 kuid mitte 12 ehk näiteks $p=7$, mille korral on lahendite kogus $\sum_{d|6}\varphi(d)=6$.

Veel on sellised moodulid, mille korral $p-1$ jagub arvuga 4 kuid mitte arvuga 12 ehk näiteks $p=5$, mille korral on lahendite kogus 4.

Oleksid võimalikud ka sellised moodulid, mille korral $p-1$ jagub arvuga 3 kuid mitte arvuga 6, kuid sel juhul peaks $p=3k+1$, kus $k$ on paaritu arv, kuid kuna 3 ja $k$ on paaritud, on ka nende korrutis paaritu ehk sellele 1 liites saab paarisarvu ning ainus paaris algarv on 2, mis ei sobi $p$ väärtuseks.

On ka sellised moodulid, mille korral $p-1$ jagub arvuga 2 kuid mitte arvuga 4 ega 3 ehk näiteks $p=11$, mis puhul on lahendite kogus 2.

Viimaks oleksid ka moodulid, mille korral $p-1$ ei jagu ei 2 ega 3ga, kuid selleks jällegi $p$ olema paarisarv.

Kokkuvõttes on võimalikud lahendite kogused 0, 1, 2, 4, 6 ja 12.
\bigskip


\end{document}
12.kodutooTex.txt
13 KB