\documentclass{article}
\usepackage{amsfonts}
\usepackage{amsmath}
\usepackage{mathtools}
\DeclarePairedDelimiter\ceil{\lceil}{\rceil}
\addtolength{\oddsidemargin}{-1in}
\addtolength{\evensidemargin}{-1in}
\addtolength{\topmargin}{-0.4in}
\addtolength{\textheight}{1in}
\addtolength{\textwidth}{1.75in}
\begin{document}
\begin{center}
\Large\textbf{Kontrollt\"o\"o nr. 1}\\
10. variant\\
\small{Joosep N\"aks}
\end{center}
\textbf{1.} Olgu F korpus ning olgu $a,b\in F,a\neq 0$. Tuginedes ainult korpuse aksioomidele, t\~oestage, et
\begin{equation*}
(-a^{-1})b=-(a^{-1}b).
\end{equation*}
\textbf{T\~oestus:}
\begin{equation*}
\begin{aligned}
\text{Vaatlen nullelemendi korrutist korpuse elemendiga, millele liidan A3 j\"argi nullelemendi:}\\
0b=0b+0\\
\text{A4 j\"argi saan v\~otta, et nullelement on sama, mis }b^2\text{ ja selle vastandelemendi summa:}\\
0b+0=0b+(b+(-b))\\
\text{A2 j\"argi muudan liitmisj\"arjekorda:}\\
0b+0=(0b+b)+(-b)\\
\text{M3 j\"argi on saan elemendi b l\"abi korrutada \"uhikelemendiga:}\\
(0b+b)+(-b)=(0b+1b)+(-b)\\
\text{Distributiivsuse j\"argi v\~otan b sulgude ette:}\\
(0b+1b)+(-b)=(0+1)b+(-b)\\
\text{A3 j\"argi saab nullelemendi liitmise \"ara kaotada:}\\
(0+1)b+(-b)=1b+(-b)\\
\text{M3 j\"argi saab \"uhikelemendiga korrutamise \"ara kaotada:}\\
1b+(-b)=b+(-b)\\
\text{A4 j\"argi on see v\~ordne 0ga:}\\
b+(-b)=0\\
\text{Seega olen n\"aidanud, et korpuse elemendi ja nullelemendi korrutis on v\~ordne nullelemendiga.}\\
\text{Liidan A3 j\"argi t\~oestatava v\~orrandi paremale poolele nullelemendi:}\\
-(a^{-1}b)=-(a^{-1}b)+0\\
\text{Varem n\"aidatud omaduse p\~ohjal korrutan nullelemendi l\"abi elemendiga }b\\
-(a^{-1}b)+0=-(a^{-1}b)+0b\\
\text{A4 j\"argi kehtib }0=a^{-1}+(-a^{-1})\\
a^{-1}\text{ on korpuse element, kuna a on korpuse element ning M4 kohaselt on sel}\\
\text{juhul ka tema p\"o\"ordelement korpuse element. Asendan selle v\~orrandisse.}\\
-(a^{-1}b)+0b=-(a^{-1}b)+(a^{-1}+(-a^{-1}))b\\
\text{Distributiivsuse j\"argi teen sulud lahti:}\\
-(a^{-1}b)+(a^{-1}+(-a^{-1}))b=-(a^{-1}b)+(a^{-1}b+(-a^{-1})b)\\
\text{A2 j\"argi muudan liitmise j\"arjekorda ning A4 j\"argi kaovad element ja vastandelement}\\
-(a^{-1}b)+(a^{-1}b+(-a^{-1})b=0+(-a^{-1})b\\
\text{A3 j\"argi saab nullelemendi eemaldada:}\\
0+(-a^{-1})b=(-a^{-1})b\\
\text{Sellega olen n\"aidanud, et algse v\~orrandi pooled on v\~ordsed.}\\
\end{aligned}
\end{equation*}
\pagebreak\\
\textbf{2.} Olgu $X$ ja $Y$ sellised mittet\"uhjad reaalarvude hulgad, et iga $x\in X$ ja $y\in Y$ korral $y\geq x$. T\~oestage, et $X$ on \"ulalt ja $Y$ alt t\~okestatud $\inf Y \geq \sup X$.\\
\textbf{Lahendus:}\\
Hulk X on \"ulalt t\~okestatud parajasti siis, kui leidub selline naturaalarv $a$, et iga $x\in X$ korral kehtib $x\leq a$. Kuna Y on mittet\"uhi reaalarvude hulk, saab v\~otta sealt suvalise elemendi $y_0\in Y$ ning kuna iga $Y$ elemendi puhul kehtib $y\geq x$, kehtib ka $y_0\geq x$. Seega on $y_0$ hulga $X$ \"ulemine t\~oke.\\
Hulk Y on alt t\~okestatud parajasti siis, kui leidub selline naturaalarv $a$, et iga $y\in Y$ korral kehtib $y\geq a$. Kuna X on mittet\"uhi reaalarvude hulk, saab v\~otta sealt suvalise elemendi $x_0\in X$ ning kuna iga $X$ elemendi puhul kehtib $y\geq x$, kehtib ka $y\geq x_0$. Seega on $x_0$ hulga $Y$ alumine t\~oke.\\
T\~oestamaks, et kehtib $\inf Y \geq \sup X$ v\"aidan vastu\"aiteliselt, et $\inf Y < \sup X\Leftrightarrow 0 < \sup X-\inf Y$. Defineerin $\varepsilon_0 :=\sup X-\inf Y$. Infiinumi definitsiooni kohaselt iga positiivse $\varepsilon\in\mathbb{R}$ puhul leidub selline $y_0\in Y$, et $y_0<\inf Y +\varepsilon$. Kui v\~otta $\varepsilon=\varepsilon_0$, saab: $y_0<\inf Y +\varepsilon_0\Leftrightarrow y_0<\inf Y -\inf Y+\sup X\Leftrightarrow y_0<\sup X$. Supreenumi definitsiooni kohaselt iga $c\in\mathbb{R}$ korral, mis rahuldab v\~orratust $c<\sup X$, leidub $x_0\in X$, et $c<x_0$. V\~otan $c=y_0$, varem on n\"aidatud, et kehtib $y_0<\sup X$, seega peab kehtima ka $y_0<x_0$. Kuid \"ulesande p\"ustituse j\"argi iga $x\in X$ ja $y\in Y$ korral $y\geq x$, seega tekkis vastuolu ning $\inf Y < \sup X$ ei saa kehtida ehk kehtib $\inf Y \geq \sup X$.\\\\
\textbf{3.} Olgu $(x_n)$ ja $(y_n)$ t\~okestatud arvjadad. T\~oestage, et
\begin{equation*}
\lim_{\overline{n\to\infty}}(\min \{x_n,y_n\})=\min\{\lim_{\overline{n\to\infty}} x_n, \lim_{\overline{n\to\infty}} y_n \}
\end{equation*}
1) N\"aitan, et kehtib $\displaystyle \lim_{\overline{n\to\infty}}(\min \{x_n,y_n\})\leq\min\{\lim_{\overline{n\to\infty}} x_n, \lim_{\overline{n\to\infty}} y_n \}$:\\\\
Olgu jada $(\min\{x_n,y_n\})$ osajada $(\min\{x_{n_k},y_{n_k}\})_{k=1}^\infty$ selline, et $\displaystyle\lim_{k\to\infty}(\min\{x_{n_k},y_{n_k}\})= \lim_{\overline{n\to\infty}}(\min \{x_n,y_n\})$. Sellise osajada olemasolu tuleneb loengukonspekti teoreemist 2.22, mis \"utleb, et iga jada osapiirv\"a\"artuste hulgas on olemas v\"ahim, mis on v\~ordne selle jada alumise piirv\"a\"artusega. Kuna jada $(x_{n_k})$ on t\~okestatud, leidub  tal Bolzano-Weierstrassi teoreemi kohaselt koonduv osajada $(x_{n_{k_j}})_{j=1}^\infty$. Kuna $(x_{n_{k_j}})_{j=1}^\infty$ ja $(y_{n_{k_j}})_{j=1}^\infty$ on koonduvad jadad, on nende miinumumi piirv\"a\"artus v\~ordne nende piirv\"a\"artuste miinumumidega. Seega kehtib:\\
\begin{equation*}
\lim_{\overline{n\to\infty}}(\min \{x_n,y_n\})=\lim_{{j\to\infty}}(\min \{x_{n_{k_j}},y_{n_{k_j}}\})=\min \{\lim_{{j\to\infty}}x_{n_{k_j}},\lim_{{j\to\infty}}y_{n_{k_j}}\}\geq\min \{\lim_{{\overline{n\to\infty}}}x_n,\lim_{\overline{n\to\infty}}y_n\}
\end{equation*}\\
2) N\"aitan, et kehtib $\displaystyle \lim_{\overline{n\to\infty}}(\min \{x_n,y_n\})\geq\min\{\lim_{\overline{n\to\infty}} x_n, \lim_{\overline{n\to\infty}} y_n \}$:\\\\
Kuna $(x_n)$ ja $(y_n)$ on t\~okestatud, on ka $\min\{x_n,y_n\}$ t\~okestatud, seega on jadade alumised piirv\"a\"artused reaalarvud. Defineerin $u_n:=\displaystyle\inf_{k\geq n} x_k$ ja $v_n:=\displaystyle\inf_{k\geq n} y_k$. Siis kehtib iga $n\in\mathbb{N}$ korral v\~orratus:
\begin{equation*}
\min\{u_n,v_n\}\geq\inf\{\min \{x_k,y_k\}:k\geq n\}
\end{equation*}
Piirv\"a\"artuse monotoonsuse kohsalt saab v\~otta m\~olemast poolest piirv\"a\"artuse:
\begin{gather*}
\min\{\lim_{\overline{n\to\infty}}x_n,\lim_{\overline{n\to\infty}}y_n\}=\min\{\lim_{n\to\infty}u_n,\lim_{n\to\infty}v_n\}=\lim_{n\to\infty}\min\{u_n,v_n\}\geq\\
\geq\lim_{n\to\infty}\inf\{\min \{x_k,y_k\}:k\geq n\}=\inf_{\overline{n\to\infty}}\{\min \{x_k,y_k\}\}
\end{gather*}
\pagebreak\\
\textbf{4.} L\"ahtudes funktsiooni piirv\"a\"artuse $\varepsilon$-$\delta$-definitsioonist t\~oestage, et
\begin{equation*}
\lim_{x\to\infty} \frac{x^2}{4-x}=-\infty
\end{equation*}
\textbf{Lahendus:} Funktsiooni piirv\"a\"artuse $\varepsilon$-$\delta$-definitsiooni j\"argi kehtib $\displaystyle\lim_{x\to\infty} f(x)=-\infty$ parajasti siis, kui kehtib
\begin{equation*}
\forall M<0\ \exists N>0:[x\in \mathbb{R}, x>N]\Rightarrow f(x)<M
\end{equation*}
\begin{gather*}
M<0\Leftrightarrow 4M<0\\
x>N>0\Leftrightarrow x>0\\
\text{Valin } N=\max\{-M,4\}\Leftrightarrow N\geq-M \text{ ja }N\geq 4\\
x>N\geq-M\Leftrightarrow x+M>0\\
x(x+M)>0>4M \Leftrightarrow x^2>4M-xM\Leftrightarrow x^2>M(4-x)\\
x>N\geq4\Leftrightarrow 4-x<0\\
\frac{x^2}{4-x}<M
\end{gather*}
Seega kehtib $\displaystyle\lim_{x\to\infty} \frac{x^2}{4-x}=-\infty$.\\\pagebreak\\
\textbf{5.} Leida piirv\"a\"artus $\displaystyle\lim_{|x|\to\infty} \bigg( \frac{2x+3}{2x+1}\bigg)^{x+1+\sin 1}$.\\
\begin{gather*}
\lim_{|x|\to\infty} \bigg( \frac{2x+3}{2x+1}\bigg)^{x+1+\sin 1}=\lim_{|x|\to\infty} \bigg( \frac{2x+1+2}{2x+1}\bigg)^{x+1+\sin 1}=\lim_{|x|\to\infty} \bigg(1+ \frac{2}{2x+1}\bigg)^{x+1+\sin 1}\\
\text{Teen muutujavahetuse } u:=\frac{2x+1}{2}\text{, kui kehtib }|x|\to\infty\text{ siis kehtib ka ka }|u|\to\infty\\
x=u-\frac{1}{2}\\
\begin{aligned}
\lim_{|x|\to\infty} \bigg(1+ \frac{2}{2x+1}\bigg)^{x+1+\sin 1}&=\lim_{|u|\to\infty} \bigg(1+ \frac{1}{u}\bigg)^{u-\frac{1}{2}+1+\sin 1}\\
\lim_{|u|\to\infty} \bigg(1+ \frac{1}{u}\bigg)^{u-\frac{1}{2}+1+\sin 1}&=\lim_{|u|\to\infty} \Bigg(\bigg(1+ \frac{1}{u}\bigg)^u\bigg(1+ \frac{1}{u}\bigg)^\frac{1}{2}\bigg(1+ \frac{1}{u}\bigg)^{\sin1}\Bigg)\\
\end{aligned}\\\\
\lim_{|u|\to\infty}f(u)=A\text{ kehtib parajasti siis, kui kehtivad } \lim_{u\to\infty}f(u)=A \text{ ja } \lim_{-u\to\infty}f(u)=A\\
\shoveleft{\text{Vaatlen juhtu }u\to\infty:}\\
\begin{aligned}
\lim_{u\to\infty} \Bigg(\bigg(1+ \frac{1}{u}\bigg)^u\bigg(1+ \frac{1}{u}\bigg)^\frac{1}{2}\bigg(1+ \frac{1}{u}\bigg)^{\sin1}\Bigg)&=\lim_{u\to\infty} \bigg(1+ \frac{1}{u}\bigg)^u\lim_{u\to\infty}\bigg(1+ \frac{1}{u}\bigg)^\frac{1}{2}\lim_{u\to\infty}\bigg(1+ \frac{1}{u}\bigg)^{\sin1}\\
&=e\bigg(\lim_{u\to\infty}(1+ \frac{1}{u})\bigg)^\frac{1}{2}\bigg(\lim_{u\to\infty}(1+ \frac{1}{u})\bigg)^{\sin1}\\
&=e(1)^\frac{1}{2}(1)^{\sin1}\\
&=e
\end{aligned}\\
\text{Vaatlen juhtu }-u\to\infty\text{, teen muutujavahetuse }t=-u\\
\begin{aligned}
\lim_{t\to\infty} \Bigg(\bigg(1+ \frac{1}{-t}\bigg)^{-t}\bigg(1+ \frac{1}{-t}\bigg)^\frac{1}{2}\bigg(1+ \frac{1}{-t}\bigg)^{\sin1}\Bigg)&=\lim_{t\to\infty} \bigg(1+ \frac{1}{-t}\bigg)^{-t}\lim_{t\to\infty}\bigg(1+ \frac{1}{-t}\bigg)^\frac{1}{2}\lim_{t\to\infty}\bigg(1+ \frac{1}{-t}\bigg)^{\sin1}\\
&=e^{(-1)(-1)}\bigg(\lim_{t\to\infty}(1+ \frac{1}{-t})\bigg)^\frac{1}{2}\bigg(\lim_{t\to\infty}(1+ \frac{1}{-t})\bigg)^{\sin1}\\
&=e(1)^\frac{1}{2}(1)^{\sin1}\\
&=e
\end{aligned}
\end{gather*}
Kuna $\displaystyle\lim_{x\to\infty} \bigg( \frac{2x+3}{2x+1}\bigg)^{x+1+\sin 1}= \displaystyle\lim_{x\to-\infty} \bigg( \frac{2x+3}{2x+1}\bigg)^{x+1+\sin 1}=e$, siis ka $\displaystyle\lim_{|x|\to\infty} \bigg( \frac{2x+3}{2x+1}\bigg)^{x+1+\sin 1}=e$.
\pagebreak\\
\textbf{6.} Olgu $f:\mathbb{R}\setminus \{0\}\to \mathbb{R}$. T\~oestage, et $\displaystyle\lim_{x\to0}f(x)=A$ parajasti siis, kui $\displaystyle\lim_{x\to0} f(\tan x)=A$.\\
\textbf{Lahendus:}\\
1) N\"aitan, et kehtib $\displaystyle\lim_{x\to0} f(\tan x)=A\Rightarrow\displaystyle\lim_{x\to0}f(x)=A$\\
\begin{gather*}
\text{Eeldasin, et kehtib:}\\
\forall \varepsilon>0\ \exists \delta_0>0:[x\in \mathbb{R}, 0<|x|<\delta_0]\Rightarrow |f(\tan x)-A|<\varepsilon\\
\text{Fikseerin epsiloni v\"a\"artuse: }\varepsilon_0=\varepsilon\text{. Tean, et kehtib:}\\
\forall \varepsilon>0\ \exists \delta>0:[x\in \mathbb{R}, 0<|x|<\delta]\Rightarrow |\tan x|<\varepsilon\\
\text{Valin }\varepsilon = \varepsilon_0\\
\exists \delta_1>0:[x\in \mathbb{R}, 0<|x|<\delta_1]\Rightarrow |\tan x|<\varepsilon_0\\
\text{V\~otan }\delta_m=\min\{\delta_0,\delta_1\}\\
\text{Sel juhul kehtib:}\\
\forall \varepsilon_0>0\ \exists \delta_m>0:[x\in \mathbb{R}, 0<|x|<\delta_m]\Rightarrow |f(\tan x)-A|<\varepsilon_0 \text{ ja }|\tan x|<\varepsilon_0\quad(1)\\
\text{Kui eeldada, et }f(x)\text{ piirv\"a\"artus ei ole A, t\"ahendab see, et}\\
\exists \varepsilon>0\ \forall \delta>0\ \exists x\in \mathbb{R}: 0<|x|<\delta \text{ ja }|f(x)-A|\geq\varepsilon\quad (2)\\
\text{Kuna v\"aide (1) kehtib iga epsiloni korral, saab v\~otta }\varepsilon_0=\varepsilon\text{, ning kuna (2) kehtib iga delta puhul, saab v\~otta }\delta=\delta_m.\\
\text{Kuna epsilon ja delta v\"ahenevad ning }|\tan x|<\varepsilon \text { ja }|x|<\delta\text { ning }f(\tan x)\text { koondub, kuid } f(x)\\
\text{ ei koondu, oleme j\~oudnud vastuoluni. Seega peab kehtima } \lim_{x\to0}f(x)=A.\\
\end{gather*}

2) N\"aitan, et kehtib $\displaystyle\lim_{x\to0} f(x)=A\Rightarrow\displaystyle\lim_{x\to0}f(\tan x)=A$\\ $\varepsilon$-$\delta$-definitsiooni j\"argi kehtib $\displaystyle\lim_{x\to0} f(x)=A$ parajasti siis, kui kehtib
\begin{gather*}
\forall \varepsilon>0\ \exists \delta>0:[x\in \mathbb{R}, 0<|x|<\delta]\Rightarrow |f(x)-A|<\varepsilon\\
\text{On teada ka et kehtib }\lim_{x\to0} \tan x=0\text{ ehk}\\
\forall \varepsilon_1>0\ \exists \delta_1>0:[x\in \mathbb{R}, 0<|x|<\delta_1]\Rightarrow |\tan x|<\varepsilon_1\\
\text{Kuna viimane kehtib iga }\varepsilon_1\text{ korral, saab v\~otta } \varepsilon_1 = \delta:\\
\exists \delta_1>0:[x\in \mathbb{R}, 0<|x|<\delta_1]\Rightarrow |\tan x|<\delta\\
\text{Seega eelneva tingimuse p\~ohjal kehtib}\\
\forall \varepsilon>0\ \exists \delta_1:[x\in \mathbb{R}, 0<|x|<\delta_1]\Rightarrow |f(\tan x)-A|<\varepsilon
\end{gather*}
\begin{center}
Seega kehtib $\displaystyle\lim_{x\to0}f(\tan x)=A$.
\end{center}
\end{document}