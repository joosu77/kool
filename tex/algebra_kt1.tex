\documentclass{article}
\usepackage{amsfonts}
\usepackage{amsmath}
\usepackage{mathtools}
\usepackage{systeme}
\usepackage{polynom}
\usepackage{pgfplots}
\usepackage[shortlabels]{enumitem}
\everymath{\displaystyle}
\DeclarePairedDelimiter\ceil{\lceil}{\rceil}
\newcommand{\p}[1]{\frac{\partial}{\partial #1}}
\begin{document}
\begin{center}
\Large\textbf{Kontrolltöö nr. 1 järeltöö}\\
\small{Joosep Näks}
\end{center}
\textbf{1. } Et $(X, \circ)$ oleks rühm, (1) peab ta olema assiotsiatiivne, (2) temal peab leiduma ühikelement ning (3) iga tema element peab olema pööratav. Kontrollin neid:
\begin{enumerate}
\item Olgu $f,g,h\in X$ ning $x\in\mathbb{R}$, siis $((f\circ g)\circ h) (x)=(f\circ g)(5^{a_h}x)=5^{a_f}(5^{a_g}(5^{a_h}x))=5^{a_f}(5^{a_g}(5^{a_h}x))=f(( g\circ h)(x))=(f\circ (g\circ h)) (x)$\\
\item Ühikelemendiks sobib $e(x)=5^0x=x$ kuna ta on ka ühikelement kõigi teisenduste hulga peal.
\item Iga elemendi $f\in X, f(x)=5^ax$ saab valida pöördelemendiks $f^{-1}(x)=5^{-a}x$, kuna nende kompositsioon on: $(f\circ f^{-1})(x)=5^a(5^{-a}x)=x=5^{-a}(5^ax)=(f^{-1}\circ f)(x)$
\end{enumerate}
Seega on $(X, \circ)$ rühm. et ta oleks ka Abeli rühm, peab ta olema ka kommutiivne. Kontrollin seda:\\
Olgu $f,g\in X, x\in \mathbb{R}$, siis $(f\circ g)(x) = 5^{a_f}(5^{a_g}x)=5^{a_g}(5^{a_f}x)=(g\circ f)(x)$\\
Ehk see on kommutiivne ja on ka Abeli rühm.
\pagebreak\\
\textbf{2.} Kui viia kompleksarv $z$ eksponentsiaalkujule, saab võrrandi vasaku poole esitada nii: $\overline{z^3}^2=\overline{(re^{i(\phi+2n\pi)})^3}^2=\overline{(r^3e^{3i(\phi+2n\pi)})}^2=(r^3e^{-3i(\phi+2n\pi)})^2=r^6e^{-6i(\phi+2n\pi)}$ ning parema poole saab esitada: $(re^{i(\phi+2k\pi)})^5=r^5e^{i5(\phi+2k\pi)}$. Et kompleksarvud võrdsed oleksid, peavad olema võrdsed nii nende moodulid kui ka argumendid, ehk peavad kehtima võrrandid $r^6=r^5$ ja $-6(\phi+2n\pi)=5(\phi+2k\pi)$. Esimesest võrrandist on näha, et sobilike kompleksarvude moodul peab olema 0 või 1. Teisest võrrandist saab teisendades $\phi=\frac{2\pi}{11}(5k+6k)$.\\
Seega on võrrandil 12 lahendit: 0 ja kõik kompleksarvud kujul $e^{in\frac{2\pi}{11}},\ n\in\mathbb{Z}$
\pagebreak\\
\textbf{3.} Leian alamringi $R$. Et $R$ oleks $\mathbb{C}$ alamring, peab ta sisaldama $\mathbb{C}$ ühikelementi 1 ja nullelementi 0. $R$ peab olema liitmise suhtes kinnine, seega kõik naturaalarvud sisalduvad ringis $R$, kuna need on kõik mingi koguse ühikelemendi endale juurde liitmisel saadud arvud. $R$ peab olema kinnine ka vastandarvu võtmise suhtes, seega peavad ka kõigi naturaalarvude vastandarvud sisalduma ringis $R$, ehk kõik täisarvud sisalduvad ringis $R$. On teada, et $\mathbb{Z}$ on ring, seega $\mathbb{Z}$ on korpuse $\mathbb{R}$ vähim alamring.\\
Lisan ringi $R$ ka elemendi $6i-2$. Et ring on liitmise suhtes kinnine, peavad sisalduma seal ka iga imaginaararvu $a+ib,\ a,b\in\mathbb{Z}$ puhul kõik arvud kujul $a+ib+c,\ c\in\mathbb{Z}$ ehk $ib+d,\ \forall d\in\mathbb{Z}$. Et ring on korrutamise suhtes kinnine, peavad seal leiduma kõik 6 kordse imaginaarliikme kordajaga imaginaararvud, kuna neid saab arvu $6i$ läbi korrutamisel erinevate täisarvudega ning seejärel mingi täisarvu juurde liitmisel.\\
Saadud imaginaararvude liitmisel saab ikkagi vaid 6 kordse imaginaarliikmega arve, mis on juba ringis. Saadud imaginaararvude korrutamisel saab arve kujul: $(6ai+b)(6ci+d)=-36ac+6i(d+b)+db$, kus imaginaarliikme kordaja on ikkagi 6 kordne.\\
Seega on vähim sobiv alamring $R=\mathbb{Z}\cup\{6ai+b\ |\ a,b\in\mathbb{Z}\}$.
\pagebreak\\
\textbf{4.} Definitsiooni kohaselt on maatriksi $A\in Mat_n(K)$, kus $A=(a_{ij})$ determinant $|A|=\sum_{\sigma\in S_n}\text{sign}(\sigma)\cdot a_{1\sigma(1)}\cdot a_{2\sigma(2)}\cdot ...\cdot a_{n\sigma(n)}$\\
Kui vahetada ära read $(a_{ki})_{i=1..n}$ ja $(a_{ti})_{i=1..n}$, on see sama, mis substitutsioonides ära vahetada nendele ridadele vastavad teisendused ning jätta alles algne maatriks. Ehk kui algses maatriksis mingi substitutsioon on 
\begin{gather*}
\sigma = \begin{pmatrix}
\begin{aligned}
1 && 2 && .. && k && .. && t && .. && n\\
\sigma(1) && \sigma(2) && .. && \sigma(k) && .. && \sigma(t) && .. && \sigma(n)\\
\end{aligned}
\end{pmatrix}
\end{gather*}
siis see sama substitsioon uues maatriksis oleks
\begin{gather*}
\sigma' = \begin{pmatrix}
\begin{aligned}
1 && 2 && .. && k && .. && t && .. && n\\
\sigma(1) && \sigma(2) && .. && \sigma(t) && .. && \sigma(k) && .. && \sigma(n)\\
\end{aligned}
\end{pmatrix}
\end{gather*}
Kuna substitutsiooni alumises reas toimus transpositsioon, muutus substitutsiooni paarsus.\\
Seega on uue maatriksi determinant \begin{gather*}
\begin{aligned}
|A'|=&\sum_{\sigma\in S_n}\text{sign}(\sigma')\cdot a_{1\sigma'(1)}\cdot ...\cdot a_{k\sigma'(k)}\cdot...\cdot a_{t\sigma'(t)}\cdot ...\cdot a_{n\sigma'(n)}\\
=&\sum_{\sigma\in S_n}-\text{sign}(\sigma)\cdot a_{1\sigma(1)}\cdot ...\cdot a_{k\sigma(t)}\cdot...\cdot a_{t\sigma(k)}\cdot ...\cdot a_{n\sigma(n)}\\
=&-|A|
\end{aligned}
\end{gather*}
\pagebreak\\
\textbf{5.} Kontrollin kas vektorruumi tingimused jäävad püsima:
\begin{enumerate}
\item Liitmise assotsiatiivsus jääb püsima, kuna liitmist et muudetud
\item Nullelemendi olemasolu jääb püsima, kuna liitmist ei muudetud
\item Vastandelemendi olemasolu jääb püsima, kuna liitmist ei muudetud.
\item liitmise kommutatiivsus jääb püsima, kuna liitmist et muudetud.
\item skalaariga vektorite summa korrutamise distributiivsus jääb püsima, kuna $\lambda\circ (a+b)=\overline{\lambda}\cdot (a+b)$ on sama, nagu see, kui valida skalaariks $\lambda$ asemele $\overline{\lambda}$ ning teha esialgselt defineeritud skalaariga korrutamist.
\item vektoriga skalaaride summa korrutamise distributiivsus jääb püsima, kuna kompleksarvude kaaskompleksarvude summa on sama, nagu nende kompleksarvude summa kaaskompleks
\item skalaaride korrutamise assotsiatiivsus jääb püsima, kuna kompleksarvude kaaskomplekside korrutis on sama mis nende kompleksarvude korrutise kaaskompleks
\end{enumerate}
\end{document}