\documentclass{article}
\usepackage{amsfonts}
\usepackage{amsmath}
\usepackage{mathtools}
\DeclarePairedDelimiter\bfloor{\Bigl\lfloor}{\Bigr\rfloor}
\DeclarePairedDelimiter\floor{\lfloor}{\rfloor}
\begin{document}
\begin{center}
\Large\textbf{T\"arn\"ulesanne nr. 154}\\
\small{Joosep N\"aks}
\end{center}
Olgu $f$ reaalteljel pidev funktsioon. Leidke piirv\"a\"artus
\begin{gather*}
\lim_{h\to0}\frac{1}{h}\int_a^b(f(x+h)-f(x))dx
\end{gather*}
\textbf{Lahendus:} Olgu $F(x)$ funktsiooni $f(x)$ algfunktsioon ehk kehtib $F'(x)=f(x)$. Sel juhul saab integraali lahti kirjutada:
\begin{equation*}
\begin{aligned}
\lim_{h\to0}\frac{1}{h}\int_a^b(f(x+h)-f(x))dx&=\lim_{h\to0}\frac{1}{h}(F(b+h)-F(b))-(F(a+h)-F(a))\\
&=\lim_{h\to0}\frac{F(b+h)-F(b)}{h}-\frac{F(a+h)-F(a)}{h}
\end{aligned}
\end{equation*}
Tuletise definitsiooni kohaselt on see:
\begin{equation*}
\begin{aligned}
\lim_{h\to0}\frac{F(b+h)-F(b)}{h}-\frac{F(a+h)-F(a)}{h}&=F'(b)-F'(a)\\
&=f(b)-f(a)
\end{aligned}
\end{equation*}
Seega on selle piirv\"a\"artuse v\"a\"artus $f(b)-f(a)$.
\end{document}