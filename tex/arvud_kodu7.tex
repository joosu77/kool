\documentclass[a4paper, 10pt]{article}
\usepackage[estonian]{babel}
\usepackage{t1enc}
\usepackage{amsthm}
\usepackage{amscd}
\usepackage{amssymb}
\usepackage{lscape}
\usepackage{amsfonts}
\usepackage{amsmath}
\usepackage{mathtools}
\usepackage{systeme}
\usepackage{polynom}
\usepackage[shortlabels]{enumitem}
\usepackage[a4paper,margin=1in,footskip=0.25in]{geometry}
\usepackage{pgffor}
\everymath{\displaystyle}
\DeclarePairedDelimiter\ceil{\lceil}{\rceil}
\newcommand{\p}[1]{\frac{\partial}{\partial #1}}
\newcommand{\Z}{\mathbb{Z}}
\newcommand{\N}{\mathbb{N}}
\newcommand{\w}{\overline}
\topmargin-3em
\oddsidemargin0cm
\textwidth16cm
%\textheight27cm
\evensidemargin-2cm
\begin{document}
\begin{center}
\Large\textbf{Kodutöö nr. 7}\\
\small{Joosep Näks ja Uku Hannes Arismaa}
\end{center}

\noindent 1. Lahendada lineaarkongruents $2021x+6590\equiv 2022 \pmod{2020}.$

\bigskip
$2021x+6590\equiv 2022 \pmod{2020}\Leftrightarrow x\equiv 1492 \pmod{2020} $
\bigskip

\noindent 2. Lahendada lineaarkongruentside s\"usteem 
\[
\left\{
\begin{array}{ll}
3x\equiv 7 & \pmod{25}\\
7x\equiv 8 & \pmod{20}\\
11x\equiv 10 & \pmod{15}.
\end{array}
\right.
\]

\bigskip
Et kasutada Hiina jäägiteoreemi, tahan viia süsteemi kujule, kus võrrandite moodulite vähim ühistegur oleks paarikaupa 1. Et lahendini jõuda, tahan ka, et moodulitesse jääksid alles kõik algtegurid, mis seal on, nende maksimaalsetes astmetes. Seega kuna $25=5^2$, $20=2^2\cdot5$ ja $15=3\cdot5$, muudan teise mooduli 4ks ja kolmanda mooduli 3ks. Mooduleid saab muuta kuna kui $20|8-7x$ ja $4|20$ siis transitiivsuse tõttu ka $4|8-7x$ ehk $7x\equiv8\pmod4$ ning analoogselt saab näidata et kolmas võrrand kehtib mooduli 3 järgi.

Eemaldan nüüd tundmatute eest kordajad. On näha, et $25\cdot2+1=51$ jagub kolmega, täpsemalt $51=3\cdot17$ ehk ringis $\Z_{25}$ on $\w3^{-1}=\w{17}=\w{-8}$ ehk esimeses võrrandis saab mõlemad pooled arvuga -8 läbi korrutada ja saab $-8\cdot3x\equiv-8\cdot7\pmod{25}$ ehk $x\equiv-6\pmod{25}$. Teises võrrandis kõigepealt saab uue mooduli tõttu teisendada võrrandi uuele kujule $-x\equiv0\pmod4$ ehk $x\equiv0\pmod4$. Kolmandas võrrandis teisendan võrrandi jällegi uue mooduli järgi ning saan $2x\equiv1\pmod3$, võtan pöördelemendi: $\w2^{-1}=\w2$ ehk $x\equiv2\pmod3$. Seega olen jõudnud uue võrrandisüsteemini:
\[
\left\{
\begin{array}{ll}
x\equiv -6 & \pmod{25}\\
x\equiv 0 & \pmod{4}\\
x\equiv 2 & \pmod{3}.
\end{array}
\right.
\]
Ning selle peal saab rakendada Hiina jäägiteoreemi. Leian esiteks vajalikud moodulite korrutiste pöördelemendid.\\ $m_1=4\cdot3=12$ ehk kuna $-2\cdot12=-24\equiv1\pmod{25}$, saan $\w{k_1}=\w{m_1^{-1}}=\w{-2}$\\ Teise võrrandi kordajaid pole mõtet leida kuna $a_2$ on 0.\\
$m_3=25\cdot4=100$ ehk $\w{k_3}=\w{m_3}^{-1}=\w{1}^{-1}=\w{1}$\\
Ning seega on lahend $x=-6\cdot(-2)\cdot12+0+2\cdot100\cdot1=344$ kõigi moodulite korrutise mooduli järgi ehk kokkuvõttes on süsteemi lahend $x\equiv344\equiv44\pmod{300}$
\bigskip

\noindent 3. Lahendada lineaarkongruentside s\"usteem 
$$
\left\{
\begin{array}{ll}
3x\equiv 7 & \pmod{25}\\
7x\equiv 8 & \pmod{20}\\
11x\equiv 9 & \pmod{15}.
\end{array}
\right.
$$

\bigskip
Selle jaoks on eelmises ülesandes juba eeltöö tehtud, ainus asi mis muutus on kolmanda võrrandi vabaliige, seega kui kolmas võrrand uue vabaliikmega teisendada mooduli kolm järgi saab $2x\equiv0\pmod3$ ehk $x\equiv0\pmod3$. Ning olengi saavutanud võrrandi, mille peal Hiina jäägiteoreemi kasutada:
\[
\left\{
\begin{array}{ll}
x\equiv -6 & \pmod{25}\\
x\equiv 0 & \pmod{4}\\
x\equiv 0 & \pmod{3}.
\end{array}
\right.
\]
Erinevus eelmisest ülesandest on vaid see, et nüüd on ka lahendi summa kolmas liige 0 ehk lahendiks on $x=144+0+0$ ehk $x\equiv144\pmod{300}$.
\bigskip

\pagebreak

\noindent 4. Lahendada kongruents $2022^{\left(2021^{2020}\right)} \pmod{1995}$.

\bigskip
Vaatame arvu $x:=2022^{\left(2021^{2020}\right)}$ moodulite 3,5,7,19 järgi. Saame järgmised tulemused:

$x\equiv 0 \pmod{3}$

$2^4 \equiv 1 \pmod{5}\Rightarrow x\equiv 2^{1^{2020}}\equiv 2 \pmod{5}$

$x\equiv -1^{2021^{2020}}\equiv -1 \pmod{7}$

$8^6 \equiv 1 \pmod{19} \Rightarrow x\equiv 8^{-1^{2020}}\equiv 8 \pmod{19}$

Siit saame HJT järgi, et $x \equiv 2\cdot399\cdot (-1) -285\cdot3+8\cdot105\cdot2=27$
\bigskip

\noindent 5. Leida suuruselt 2019-s selline naturaalarv $n$, mille korral nii $n$ kui $n^2$ annavad arvuga 891 jagades ühe ja sama jäägi. 

\bigskip
Märkame, et lahendada $n^2\equiv n \pmod{891}$ on ekvivalentne ülesandele
\[
\left\{
\begin{array}{ll}
n(n-1)\equiv 0 & \pmod{81}\\
n(n-1)\equiv 0 & \pmod{11}\
\end{array}
\right.
\]
Mõlema võrrandi puhul on ainsad lahendid 1 ja 0, kuna kõrvuti olevad arvud ei saa mõlemad jagada sama algarvu.
Märkame, et $11\cdot -22\equiv 1 \pmod {81}$ ning $81\cdot 3 \equiv 1 \pmod{11}$, seega saame HJTst võimalikeks lahenditeks $x\equiv 0, x\equiv 649, x\equiv 243$ ning $x\equiv 1 \pmod{891}$. Iga suuruselt 4. selline naturaalarv on 891 kordne, seega 2020. arv oleks $891\cdot 505$, seega 2019. on $891\cdot 504+649=449713$.
\bigskip

\noindent 6. \mbox{Tõestada, et kahe järjestikuse ruuduvaba arvu vahe võib olla kuitahes suur.}

\bigskip
Kui tahta leida sellist arvu $x$, millest järgmised $n$ tükki kõik sisaldavad oma tegurite hulgas ruute, saab koostada võrrandisüsteemi kujul $x+i\equiv 0\pmod {p_i^2}$ ehk $i$ ümber tõstes $x\equiv -i\pmod {p_i^2}$ kus $i=1,..,n$ ja $p_1,..,p_n$ on vabalt valitud erinevad algarvud. Kuna $p_i$ on kõik erinevad algarvud, pole neil paarikaupa ühistegurit ehk Hiina jäägiteoreemi põhjal leidub süsteemil lahend. Seega saab valida kuitahes suure arvu $n$ ning leidub selline arv $x$, millest järgmine ruuduvaba arv on vähemalt $n$ arvu kaugusel.
\bigskip

\noindent 7. Leida vähim \emph{naturaal}arv, mis on korraga kahekordne täisruut, kolmekordne täiskuup ja 1999-kordne 1999-s aste. 

\bigskip

Arv, mis me valime peab olema 2 3 ja 1999 kordne, seega kujul $2^i3^j1999^kx$. Selleks, et arv oleks täisaste, peavad kõik algtegurid eraldi olema selles astmes, seega kui $x>1$, siis see ei mõjuta arvu esimest poole täisaste olemist, seega kui leida vähimad $i,j,k$ nii, et arvu esimene pool vastab tingimustele, pole mõtet seda enam $x>1$ läbi korrutada, kuna see teeks arvu asjatult suuremaks.

Et arv $n$ oleks kahekordne kuup, peab kehtima $n=2w^2=2\cdot2^{i-1}3^j1999^k$ ehk $j$ ja $k$ peavad kahega jaguma ning $i$ peab jäägiks andma 1. Sama saab arvugeda 3 ja 1999 teha ning saame tingimused $i\equiv 1,j\equiv0,k\equiv0 \pmod{2}$, $i\equiv 0,j\equiv1,k\equiv0 \pmod{3}$, $i\equiv 0,j\equiv0,k\equiv1 \pmod{1999}$. HJTst saame $i\equiv 1\cdot 3\cdot 1999\cdot 1=5997$, $j \equiv 1\cdot 2\cdot 1999\cdot 2=7996$, $k \equiv 2\cdot 3\cdot 1000\cdot 1333\equiv 9996 \pmod{2\cdot3\cdot 1999}$. Asendades need algsesse kujusse saamegi vähima sellise arvu: $2^{5997}\cdot3^{7996}\cdot1999^{9996}$.

\bigskip
\pagebreak

\noindent 8. Tõestada, et iga paarisarvu $m=2k\in\Z$ ja naturaalarvu $n$ korral leiduvad $a,b\in\Z$ nii, et $m=a-b$ ja $(a,n) = (b,n) = 1$. 

\bigskip
Tegurdan $m$ ja $n$ algteguriteks nii, et $p_i$ on tegurid, mis jagavad nii $n$ kui ka $m$, $q_i$ jagavad vaid arvu $m$ ning $r_i$ vaid arvu $n$. Saan $m=\prod_ip_i^{u_i}\cdot\prod_iq_i^{v_i}$ ja $n=\prod_ip_i^{u_i}\cdot\prod_ir_i^{z_i}$. 

Leian iga $i$ jaoks saab leida sellise $a_i$ väärtuse, et $a_i\not\equiv0\pmod{r_i}$ ja $a_i\not\equiv-\prod_ip_i^{u_i}\pmod{r_i}$. Sellised $a_i$ väärtused leiduvad, kuna $r_i$ on alati suurem kui 2 ehk jäägiklassiringis $\Z_{r_i}$ on vähemalt 3 liiget (kui $n$ jagub 2ga, on 2 $p_i$ väärtuste hulgas, kuna on teada et $m$ on paarisarv). Koostan võrrandisüsteemi võrranditega $x\equiv a_i\pmod{r_i}$ ja $x\equiv1\pmod{p_i}$. Hiina jäägiteoreemi põhjal leidub sellel lahend kuna $r_i$ ja $p_i$ on paarikaupa ühistegurita.

Võtan nüüd $b=x\cdot\prod_iq_i^{v_i}$ ja $a=m+b$. Arv $x$ ei oma arvuga $n$ ühistegurit, kuna võrrandid, mille järgi $x$ leitud sai, ütlevad et ükski arvu $n$ tegur ei jaga arvu $x$ ning ükski $q_i$ ei jaga arvu $n$ seega $n$ ja $b$ ei oma ühistegurit. Arv $a$ ei jagu ühegi algarvuga $p_i$, kuna $a= m+x\cdot\prod_iq_i^{v_i}\equiv 0+a_i\cdot\prod_iq_i^{v_i}\not\equiv0\pmod{p_i}$. Viimane kehtib seepärast, et $q_i$ ja $p_i$ on alati erinevad algarvud ehk nende suurim ühistegur on 1. Ta ei jagu ka ühegi algarvuga $r_i$, kuna $a=\prod_ip_i^{u_i}\cdot\prod_iq_i^{v_i}+x\cdot\prod_iq_i^{v_i}=\prod_iq_i^{v_i}(\prod_ip_i^{u_i}+x)$ ning on teada et arvud $q_i$ korrutis ei jagu arvudega $r_i$ ning eelduste kohaselt $\prod_ip_i^{u_i}+x\not\equiv\prod_ip_i^{u_i}-\prod_ip_i^{u_i}=0\pmod{r_i}$ ehk ka korrutise teine pool ei saa jaguda ühegi arvuga $r_i$.

\bigskip

\noindent 9. Koostada tekst\"ulesanne, mida saab lahendada Hiina \mbox{j\"a\"agiteoreemi} abil. \\(\"Ulesande tekst ja 
selle {\bf lahendus} tuleb esitada kirjalikult. Põhimõtteliselt vale lahendusega ülesanne annab 0 punkti. K\~oige originaalsema üles\-ande
ja kõige raskema \"ulesande koostajad saavad kumbki 3-5 punkti sõltuvalt ülesande tasemest. Kloonitud \"ulesannete 
esitajad saavad 0 punkti.)

\bigskip
Ülesanne: Rabamatkal märkad, et teepeale jäävad jõhvikad on 10m vahedega ning murakad 19m vahedega. Olles kõndinud 87m viimasest kohast, kus murakas ja jõhvikas olid kõrvuti, märkad, et nüüd, kus raba hakkab otsa lõppema, on iga jõhvikas 12m kaugusel eelmisest ning iga murakas 23m kaugusel eelmisest. Kui mitme meetri pärast sa jälle murakat ja jõhvikat lähestikku näed?\\

Lahendus: Leian kõigepealt viimaste jõhvika ja muraka asukohad enne rabast väljumist, $87\equiv7\pmod{10}$ ja $87\equiv11\pmod{19}$ ehk viimasest jõhvikast on möödas 7m ja viimasest murakast 11m. Nüüd uute vahedega saan võrrandid $x\equiv-7\pmod{12}$ ja $x\equiv-11\pmod{23}$, kus $x$ on läbitud tee alates tihedama raba lõpust. Süsteemi lahendades saan $x=23\cdot(-1)\cdot(-7)+12\cdot(-11)\cdot2=-103\equiv173\pmod{12\cdot23}$. Ehk järgmine kord oli murakat ja jõhvikat korraga näha 87+173=260m peale algust.


\end{document}