\documentclass{article}
\usepackage{amsmath}
\usepackage{amsfonts}
\begin{document}

\begin{center}
\Large \textbf {Tärnülesanne 8}
\end{center}
Olgu meil defineeritud kompleksarvude hulk:
\begin{equation*}
C= \left \{\textup{a + b}\cdot \textup{i}\mid \textup{a, b} \in R \right \}
\end{equation*}
Ning sellel defineeritud liitmine ja korrutamine kui:
\begin {equation*}
\begin {aligned}
z_1+z_2&=(a_1+b_1*i)+(a_2+b_2*i)\\
&=(a_1+a_2)+(b_1+b_2)*i\\
z_1*z_2&=(a_1+b_1*i)*(a_2+b_2*i)\\
&=(a_1*a_2-b_1*b-2)+(a_1*b_2+a_2*b_1)*i
\end {aligned}
\end{equation*}
Näitan, et kehtivad liitmise ja korrutamise aksioomid:\\
\textbf{A1:} Liitmise kommutatiivsus:
\begin {equation*}
\begin {aligned}
z_1+z_2&=(a_1+a_2)+(b_1+b_2)*i\\
&=(a_2+a_1)+(b_2+b_1)*i\\
&=z_2+z_1
\end {aligned}
\end {equation*}
\textbf{A2:} Liitmise assotsiatiivsus:
\begin {equation*}
\begin {aligned}
(z_1+z_2)+z_3&=((a_1+a_2)+(b_1+b_2)*i)+(a_3+b_3*i)\\
&=((a_1+a_2)+a_3)+((b_1+b_2)+b_3)*i\\
&=(a_1+(a_2+a_3)+(b_1+(b_2+b_3))*i\\
&=(a_1+b_1*i)+((a_2+a_3)+(b_2+b_3)*i\\
&=z_1+(z_2+z_3)\\
\end {aligned}
\end {equation*}
\textbf{A3:} Nullelemendi olemasolu:\\
Võtan nullelemendiks arvu $0+0*i$
\begin {equation*}
\begin {aligned}
z_1+0&=(a_1+b_1*i)+(0+0*i)\\
&=(a_1+0)+(b_1+0)*i\\
&=a_1+b_1*i\\
&=z_1
\end {aligned}
\end {equation*}
\textbf{A4:} Vastandelemendi olemasolu:\\
Võtan elemendi $a+b*i$ vastandelemendiks elemendi $(-a)+(-b)*i$
\begin {equation*}
\begin {aligned}
z+(-z)&=(a+b*i)+((-a)+(-b)*i)\\
&=(a+(-a))+(b+(-b))*i\\
&=0+0*i\\
&=0
\end {aligned}
\end {equation*}
\textbf{M1:} Korrutamise kommutatiivsus:
\begin {equation*}
\begin {aligned}
z_1*z_2&=(a_1+b_1*i)(a_2+b_2*i)\\
&=(a_1*a_2-b_1*b_2)+(a_1*b_2+a_2*b_1)*i\\
&=(a_2*a_1-b_2*b_1)+(a_2*b_1+a_1*b_2)*i\\
&=(a_2+b_2*i)(a_1+b_1*i)\\
&=z_2*z_1
\end {aligned}
\end {equation*}
\textbf{M2:} Korrutamise assotsiatiivsus:
\begin {equation*}
\begin {aligned}
(z_1*z_2)z_3=&((a_1+b_1*i)(a_2+b_2*i))(a_3+b_3*i)\\
=&((a_1*a_2-b_1*b_2)+(a_1*b_2+a_2*b_1)*i)(a_3+b_3*i)\\
=&((a_1*a_2-b_1*b_2)a_3-(a_1*b_2+a_2*b_1)*b_3)\\&+((a_1*a_2-b_1*b_2)b_3+(a_1*b_2+a_2*b_1)*a_3)*i\\
=&(a_1*a_2*a_3-b_1*b_2*a_3-a_1*b_2*b_3+a_2*b_1*b_3)\\&+(a_1*a_2*b_3-b_1*b_2*b_3+a_1*b_2*a_3+a_2*b_1*a_3)*i\\
=&((a_2*a_3-b_2*b_3)a_1-(a_2*b_3+a_3*b_2)*b_1)\\&+((a_2*a_3-b_2*b_3)b_1+(a_2*b_3+a_3*b_2)*a_1)*i\\
=&(a_1+b_1*i)((a_2*a_3-b_2*b_3)+(a_2*b_3+a_3*b_2)*i)\\
=&(a_1+b_1*i)((a_2+b_2*i)(a_3+b_3*i))\\
=&z_1(z_2*z_3)
\end {aligned}
\end {equation*}
\textbf{M3:} Ühikelemendi olemasolu:
Võtan ühikelemendiks $1+0*i$
\begin {equation*}
\begin {aligned}
z_1*1&=(a+b*i)*(1+0*i)\\
&=(a*1-b*0)+(a*0+1*b)*i\\
&=a+b*i\\
&=z
\end {aligned}
\end {equation*}
\textbf{M4:} Vastandelemendi olemasolu:\\
Võtan elemendi $a+b*i$ vastandelemendiks elemendi $(\frac{a}{a^2+b^2})+(\frac{-b}{a^2+b^2})*i$
\begin {equation*}
\begin {aligned}
z*z^{-1}&=(a+b*i)((\frac{a}{a^2+b^2})+(\frac{-b}{a^2+b^2})*i)\\
&=(\frac{a^2}{a^2+b^2}-\frac{-b^2}{a^2+b^2})+(\frac{-b*a}{a^2+b^2}+\frac{a*b}{a^2+b^2})*i\\
&=(\frac{a^2+b^2}{a^2+b^2})+(\frac{a*b-a*b}{a^2+b^2})*i\\
&=1+0*i\\
&=1
\end {aligned}
\end {equation*}
\textbf{D:} Distributiivsus:
\begin {equation*}
\begin {aligned}
(z_1+z_2)z_3&=((a_1+a_2)+(b_1+b_2)*i)(a_3+b_3*i)\\
&=((a_2+a_1)a_3-(b_1+b_2)b_3)+((a_1+a_2)b_3+(b_1+b_2)a_3)*i\\
&=((a_2*a_3-b_1*b_3)+(a_1*a_3-b_2*b_3))+((a_1*b_3+b_1*a_3)+(a_2*b_3+b_2*a_3))*i\\
&=[((a_2*a_3-b_1*b_3)+(a_1*b_3+b_1*a_3)*i)+((a_1*a_3-b_2*b_3)+(a_2*b_3+b_2*a_3)*i)\\
&=z_1*z_3+z_2*z_3
\end {aligned}
\end {equation*}
Kõik korpuse aksioomid kehtivad seega kompleksarvude hulk valitud liitmise ja korrutamisega on korpus. Eeldan nüüd et kompleksarvude korpus on ka järjestatud. Järjestatud korpuses iga nullist erineva elemendi ruut on suurem kui null:
\begin {equation*}
\begin {aligned}
&(1)\ z\in\mathbb{C},\ z<0\Rightarrow0<-z\Rightarrow0<(-z)(-z)\Rightarrow0<z^2\\
&(2)\ z\in\mathbb{C},\ z>0\Rightarrow0<z*z\Rightarrow0<z^2
\end {aligned}
\end {equation*}
Võtan kompleksarvu $z=0+1*i$ ruudu:
\begin {equation*}
\begin {aligned}
z^2=(0+1*i)^2=0*0-1*1+(0*1+1*0)*i=-1>0\Rightarrow0>1
\end {aligned}
\end {equation*}
Võtan nüüd suvalised elemendid $a,b\in\mathbb{C},\ a>b\Rightarrow a-b>0$:
\begin {equation*}
\begin {aligned}
0>1\Rightarrow 0*(a-b)>1*(a-b)\Rightarrow 0>a-b\Rightarrow b>a
\end {aligned}
\end {equation*}
See aga on vastuolus eeldusega $a<b$, seega kompleksarvude korpus ei saa olla järjestatud.
\end{document}