\documentclass[12pt]{article}
\usepackage{longtable}
\usepackage{multicol}
\usepackage{amssymb}
\usepackage[margin=1in]{geometry}
\newcommand{\q}[1]{\textbf{#1}}
\newcommand{\w}[1]{$\bar{\mbox{#1}}$}
\newcommand\kcol{12}
\newenvironment{tiitel}
	{\begin{center}
	\bfseries
	\large
	}{
	\end{center}
	}
\newcommand{\cm}[1]{\ignorespaces}
\begin{document}
\begin{tiitel}
Ladina keele iseseisev töö
\end{tiitel}
\begin{center}
Joosep Näks
\end{center}

\textbf{1. Tõlge:}\\
\begin{minipage}[t]{.5\textwidth}
\raggedright
1. Thetis \cm{Thetis - merejumalanna} Nereis \cm{Nereis - Nereiid, pmst nümf} cum sciret \cm{3. pööre imperf act konj, teadma} Achillem filium suum, quem ex Peleo \cm{Peleus - Tassaalia kuningas, abl vormis} habebat\cm{3. pööre imperf ind, omama}, si ad Troiam expugnandam \cm{acc fut pass part sõnast expugno ehk ründama, vallutama} isset\cm{3. pööre plusquam perf konj eo, minema}, periturum, commendavit \cm{3. perf, usaldama} eum in insulam Scyron ad Lycomedem regem, quem ille inter virgines filias habitu \cm{harjumus abl} feminino \cm{naiselik abl} servabat \cm{imperf ind päästma, kaitsma} nomine \cm{abl} mutato\cm{dat/abl s perf pass part muutma}, nam virgines Pyrrham nominarunt\cm{3.pl perf}, quoniam capillis \cm{dat/abl juuksed} flavis \cm{dat/abl blond} fuit \cm{3.s perf sum} et Graece rufum pyrrhon dicitur\cm{3.s pass}.\newline

2. Achivi autem cum rescissent \cm{plusquam perf konj, teada saama} ibi eum occultari\cm{pass inf peitma}, ad regem Lycomedem oratores \cm{nom/acc/voc pl kõneleja} miserunt\cm{3. pööre pl perf saatma, viskama}, qui rogarent\cm{3.pl imperf konj küsima}, ut eum adiutorem Danais \cm{dat/abl danaoslased} mitteret\cm{3.s imperf konj saatma}. Rex cum negaret \cm{3.s imperf konj tagasi lükkama} apud se esse, potestatem \cm{acc võim} eis fecit\cm{3.s perf tegema}, ut in regia \cm{abl palee} quaererent\cm{3.pl imperf konj otsima}.\newline

3. Qui cum intellegere \cm{inf aru saama} non possent\cm{3.pl imperf konj}, quis esset \cm{3.s imperf konj olema} eorum, Ulixes \cm{Ulixes - Odüsseus} in regio vestibulo munera feminea posuit\cm{3.s perf panema, asetama}, in quibus clipeum et hastam, et subito tibicinem iussit \cm{3.s perf} canere armorumque crepitum et clamorem fieri \cm{inf fio - tekkima, juhtuma / pass inf facio - tegema, sooritama} iussit.\newline

4. Achilles hostem arbitrans \cm{pres part} adesse \cm{inf} vestem muliebrem dilaniavit \cm{3.s perf} atque clipeum et hastam arripuit\cm{3.pl perf}. Ex hoc est cognitus \cm{perf pass part} suasque operas Achivis promisit \cm{3.s perf} et milites Myrmidones.
\end{minipage}%
\quad\vline\quad%
\begin{minipage}[t]{.5\textwidth}
\raggedright
1. Kui Nereiid Thetis teada sai et tema poeg Achilleus, kelle isa oli Peleus, sureks, kui ta läheks Troojat vallutama, usaldas ta Achilleuse Scyrose saarele kuningas Lycomedese, kus teda kasvatati neitsi tütarde hulgas teise nime all. Ta nimetati Pyrrhaks, kuna tema juuksed olid blondid ja kreeklased ütlesid punapeade kohta "pyrrhon".\newline

2. Ahhaialased see-eest teada saades tema peitmisest, saatsid saadikud kuningas Lycomedese juurde, kes paluksid, et Achilleus danaoslastele appi saadetaks. Kui kuningas lükkas tagasi, et Achilleus on nende hulgas, kuid andis neile loa otsida teda paleest.\newline

3. Kui nad ei suutnud teda leida, pani Odüsseus palee siseõue naiste asju, mille hulka peitis vaskkilpe ja odasid, ja käskis järsku vilepuhujal signaali anda ning sõjariistade tärinat ja kisa tekitada.\newline

4. Achilleus arvates et vaenlane tuli, rebis oma naiste riided puruks ning võttis vaskkilbi ja oda kätte. Sellest saadi aru et see on tema ja ta lubas aidata ahhaialasi ja mürmidooni sõdureid.
\end{minipage}\\\\

\textbf{2. Konjuktiivi vormid aegadega:}\\
\begin{tabular}{c|c}
\textbf{Sõna}&\textbf{Aeg}\\
\hline
sciret&imperfekt\\
isset&plusquam perfekt\\
\hline
rescissent&plusquam perfekt\\
rogarent&imperfekt\\
mitteret&imperfekt\\
negaret&imperfekt\\
quaererent&imperfekt\\
\hline
possent&imperfekt\\
esset&imperfekt\\
\end{tabular}

\newpage


\textbf{3. Sõnad:}
\itemize
\item nominarunt - tegu on mitmuse perfekti kolmanda pöördega, kuid kui see vorm niisama moodustada siis perfekti tüvi on nominav- ning pöördelõpp on -erunt ehk peaks saama nominaverunt, aga siin on -ve- vahelt ära jäetud
\item fieri - tegu on sõna facio passiivi infinitiivi vormiga, kuid kui tavaliselt 3. pöördkonna sõna passiivi infinitiivi moodustamiseks tuleks võtta preesensi tüvi -i lõpuga ehk tuleks faci, siis sõna facio on ebareeglipärane ning passiivi infinitiivi vorm on fieri

\textbf{4. Sõnad:}
\item expugnandam - akkusatiivi ainsuse vorm sõna expugno gerundiivist, preesensi tüvi on expugna-, sellele lisandub gerundiivi tunnus -nd- ning lõpp -us, saades expugnandus, mida käänates saab expugnandam
\item periturum - sõna pereo kesksoost futuurumi partitsiip, sõna pereo perfekti passiivi partitsiibi tüvi on perit- ning sellele lisandub futuurumi partitsiibi lõpp -urum, saades periturum

\textbf{5. AcI konstruktsioonid:}\\
\item \S2 alguses, verb, millest konstruktsioon lähtub on rescissent, akusatiiv on eum ja infinitiiv occultari.
\item \S2 keskel, verb, millest konstruktsioon lähtub on negaret, akusatiiv on se ja infinitiiv esse.
\end{document}

% aci (accusativus cum infinitivo) - lk 162
% sidesõnad - lk 254
% cum kõrvallaused - lk 259
% ut ja si kõrvallaused - lk 260