\documentclass{article}
\usepackage{amsfonts}
\usepackage{amsmath}
\usepackage{mathtools}
\usepackage{systeme}
\usepackage{polynom}
\usepackage{pgfplots}
\everymath{\displaystyle}
\DeclarePairedDelimiter\ceil{\lceil}{\rceil}
\begin{document}
\begin{center}
\Large\textbf{Kodutöö nr. 2}\\
7. variant\\
\small{Joosep Näks}
\end{center}
\textbf{1. } Tõestada, et kui $z$ ja $w$ on kompleksarvud ja $n$ on paaritu naturaalarv, siis
\begin {equation*}
\sqrt[n]{-z^nw}=-z\sqrt[n]{w}
\end{equation*}
\textbf{Lahendus:}\\
Kirjutan võrrandi mõlemad pooled kompleksarvu juure definitsiooni järgi hulkadena lahti:
\begin{gather*}
\sqrt[n]{-z^nw}=\{q\ |\ q,z,w\in\mathbb{C}, n=2k+1, k\in\mathbb{N}:\ q^n=-z^nw\}\\
-z\sqrt[n]{w}=\{-zq\ |\ q,z,w\in\mathbb{C}, n=2k+1, k\in\mathbb{N}:\ q^n=w\}\\
\end{gather*}
Kui parema poole hulga definitsiooni võrduses mõlemad pooled $(-z)^n$-ga läbi korrutada (kuna $n$ on paaritu naturaalarv, kehtib $(-z)^n=-z^n$), saab
\begin{gather*}
-z\sqrt[n]{w}=\{-zq\ |\ q,z,w\in\mathbb{C}, n=2k+1, k\in\mathbb{N}:\ (-zq)^n=-z^nw\}\\
\end{gather*}
Teen asenduse $p=-zq$:
\begin{gather*}
-z\sqrt[n]{w}=\{p\ |\ p,z,w\in\mathbb{C}, n=2k+1, k\in\mathbb{N}:\ p^n=-z^nw\}\\
\end{gather*}
Ning see on sama hulk nagu algse võrrandi vasaku poole hulk, seega kehtib $\sqrt[n]{-z^nw}=-z\sqrt[n]{w}$.
\pagebreak\\
\textbf{2.} Tõestada, et hulk
\begin {gather*}
A=\{z\in\mathbb{C}:|z|=1\}
\end{gather*}
on rühma $(\mathbb{C}\setminus\{0\},\cdot)$ alamrühm.\\
\textbf{Lahendus:}\\
Et hulk $A$ oleks alamrühm, peavad kehtima tingimused
\begin{gather}
\forall x,y\in A\quad x\cdot y\in A\\
\forall x\in A\ \exists y\in A:\ x+y=y+x=1
\end{gather}
(kuna 1 on kompleksarvude ühikelement). Suvalist elementi hulgas $A$ saab esitada kui $e^{ia}, a\in\mathbb{R}$, kuna suvalist kompleksarvu saab esitada kui $re^{ia}$, kus $r$ on tema moodul, kuid hulgas $A$ on kõigi elementide moodul 1. Näitan tingimuse (1) kehtivust:
\begin{gather*}
\forall e^{ia},e^{ib}\in A\quad e^{ia}\cdot e^{ib}=e^{i(a+b)}
\end{gather*}
Ning kuna $a+b\in\mathbb{R}$, on $e^{i(a+b)}$ hulga $A$ liige.\\
Tingimuse 2 korral saab iga $e^{ia}$ jaoks valida pöördelemendi $e^{i(-a)}$. $-a\in\mathbb{R}$, seega see element on hulgas $A$ ja kehtib
\begin{gather*}
e^{ia}\cdot e^{i(-a)}=e^{i(-a)}\cdot e^{ia}=e^{i(a-a)}=e^0=1
\end{gather*}
Seega on $A$ rühma $(\mathbb{C}\setminus\{0\},\cdot)$ alamrühm.
\end{document}