\documentclass{article}
\usepackage{amsfonts}
\usepackage{amsmath}
\usepackage{mathtools}
\usepackage{systeme}
\usepackage{polynom}
\usepackage{pgfplots}
\everymath{\displaystyle}
\DeclarePairedDelimiter\ceil{\lceil}{\rceil}
\begin{document}
\begin{center}
\Large\textbf{Kodutöö nr. 2}\\
7. variant\\
\small{Joosep Näks}
\end{center}
\textbf{1. } Tõestada, et kui $z$ ja $w$ on kompleksarvud ja $n$ on paaritu naturaalarv, siis
\begin {equation*}
\sqrt[n]{-z^nw}=-z\sqrt[n]{w}
\end{equation*}
\textbf{Lahendus:}\\
Kirjutan võrrandi mõlemad pooled kompleksarvu juure definitsiooni järgi hulkadena lahti:
\begin{gather*}
\sqrt[n]{-z^nw}=\{q\in\mathbb{C}\ |\ q^n=-z^nw\}\\
-z\sqrt[n]{w}=\{-zq\in\mathbb{C}\ |\ q^n=w\}\\
\end{gather*}
\end{document}