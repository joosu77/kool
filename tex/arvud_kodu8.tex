\documentclass[a4paper, 10pt]{article}
\usepackage[estonian]{babel}
\usepackage{t1enc}
\usepackage{amsthm}
\usepackage{amscd}
\usepackage{amssymb}
\usepackage{lscape}
\usepackage{amsfonts}
\usepackage{amsmath}
\usepackage{mathtools}
\usepackage{systeme}
\usepackage{polynom}
\usepackage[shortlabels]{enumitem}
\usepackage[a4paper,margin=1in,footskip=0.25in]{geometry}
\usepackage{pgffor}
\everymath{\displaystyle}
\DeclarePairedDelimiter\ceil{\lceil}{\rceil}
\newcommand{\p}[1]{\frac{\partial}{\partial #1}}
\newcommand{\Z}{\mathbb{Z}}
\newcommand{\N}{\mathbb{N}}
\newcommand{\w}{\overline}
\topmargin-3em
\oddsidemargin0cm
\textwidth16cm
%\textheight27cm
\evensidemargin-2cm
\begin{document}
\begin{center}
\Large\textbf{Kodutöö nr. 8}\\
\small{Joosep Näks ja Uku Hannes Arismaa}
\end{center}

\bigskip

\noindent 1. Lahendada kongruents 
\[
3x^4 + 5 x^3 - x^2 - x +1\equiv 0\pmod{7}.
\]

\smallskip
Kasutame Horneri skeemi ja proovimis meetodit.

\begin{equation*}
\begin{array}{c| c c c c c }
&\w{3}&\w{-2}&\w{-1}&\w{-1}&\w{1}\\
\hline
\w{0}&\w{3}&\w{-2}&\w{-1}&\w{-1}&\w{1}\\
\w{1}&\w{3}&\w{1}&\w{0}&\w{-1}&\w{0}\\
\w{2}&\w{3}&\w{-3}&\w{0}&\w{-1}&\w{-1}\\
\w{3}&\w{3}&\w{0}&\w{-1}&\w{3}&\w{3}\\
\w{-3}&\w{3}&\w{3}&\w{-3}&\w{1}&\w{-2}\\
\w{-2}&\w{3}&\w{-1}&\w{1}&\w{-3}&\w{2}\\
\w{-1}&\w{3}&\w{2}&\w{-3}&\w{2}&\w{-1}\\

\end{array}
\end{equation*}

Seega leidsime, et ainuke lahend on $\w{1}$.
\bigskip


\noindent 2. Tegurdada pol\"unoom  
\[
f(x)=2x^5+6x^4+5x^3-3x^2-3x+3
\] 
mooduli 5 j\"argi, s.t. \"ule korpuse $\Z_{5}$.

\bigskip
Kasutame Horneri skeemi ja proovimis meetodit.

\begin{equation*}
\begin{array}{c| c c c c c c}
&\w{2}&\w{1}&\w{0}&\w{2}&\w{2}&\w{-2}\\
\hline
\w{0}&\w{2}&\w{1}&\w{0}&\w{2}&\w{2}&\w{-2}\\
\w{1}&\w{2}&\w{-2}&\w{-2}&\w{0}&\w{2}&\w{0}\\

\end{array}
\end{equation*}

\bigskip
Leidsime esimese teguri $(x-\w{1})$
\begin{equation*}
\begin{array}{c| c c c c c }
&\w{2}&\w{-2}&\w{-2}&\w{0}&\w{2}\\
\hline
\w{1}&\w{2}&\w{0}&\w{-2}&\w{-2}&\w{0}\\

\end{array}
\end{equation*}

Seega on $(x-\w{1})$ kahekordne tegur.

\begin{equation*}
\begin{array}{c| c c c c }
&\w{2}&\w{0}&\w{-2}&\w{-2}\\
\hline
\w{1}&\w{2}&\w{2}&\w{0}&\w{-2}\\
\w{2}&\w{2}&\w{-1}&\w{1}&\w{0}\\

\end{array}
\end{equation*}

\begin{equation*}
\begin{array}{c| c c c }
&\w{2}&\w{-1}&\w{1}\\
\hline
\w{2}&\w{2}&\w{-2}&\w{2}\\
\w{-2}&\w{2}&\w{0}&\w{1}\\
\w{-1}&\w{2}&\w{2}&\w{-1}\\

\end{array}
\end{equation*}
Seega saime, et $f(x)\equiv (x-1)^2(x-2)(2x^2-x+1)\pmod{5}$
\bigskip

\pagebreak
\noindent 3. Milliste $x$ täisarvuliste väärtuste korral on arvu $2x^4+x^3-2x^2+x-2$ mõlemad viimased kümnendnumbrid 2?

\bigskip
Ülesanne taandub kongruentsi $2x^4+x^3-2x^2+x-2\equiv 22 \pmod{100}$ lahendamisele. Selleks peame lahendama kongruentsid $2x^4+x^3-2x^2+x-2\equiv 22 \pmod{4}$, $2x^4+x^3-2x^2+x-2\equiv 22 \pmod{25}\Leftrightarrow 2x^4+x^3-2x^2+x-24\equiv 0 \pmod{25}$. Esimese lahendiks on ainult $x\equiv 0 \pmod{4}$. Mooduli 5 järgi on teise lahenditeks 2 ja 3. 

$(2x^4+x^3-2x^2+x-24)'=8x^3+3x^2-4x+1$

$(8x^3+3x^2-4x+1)(2)=69\equiv -1 \pmod{5} $

$(2x^4+x^3-2x^2+x-24)(2)=10$

Seega peame näite põhjal nüüd leidma lahenduse kongruentsile $-y+2\equiv0 \pmod{5}$, saame $y\equiv2$ ehk üks lahend on $2+2\cdot 5=12$.

$(8x^3+3x^2-4x+1)(3)=232\equiv 2 \pmod{5} $

$(2x^4+x^3-2x^2+x-24)(3)=150$

Seega peame näite põhjal nüüd leidma lahenduse kongruentsile $2y+0\equiv0 \pmod{5}$, saame $y\equiv0$ ehk teine lahend on $3+0+\cdot 5=3$.

HJT järgi peaks leiduma 2 lahendit. Terava silmaga näeb ära, et need on 12 ja 28.
\bigskip


\noindent 4. Lahendada kongruents 
\[
x^4 + 4 x^3 + 2 x^2 + 2 x -38\equiv 0\pmod{125}.
\]

\smallskip
Mooduli 5 järgi saame, et ainus lahend on $x\equiv 3$. 

$f'(x)\equiv-x^3+2x^2+-x+2\pmod{5}$ $f'(3)\equiv3+3-3+2\equiv0 \pmod{5}$

$f(3)=175$, $\frac{175}{5}=35\equiv 0 \pmod{5}$

Seega saame, et lahendis kujul $x=3+5y$ oleva $y$ puhul peab kehtima, et $0y+0\equiv 0 \pmod{5}$, mis kehtib iga $y\in\{0,1,2,3,4\}$ korral, seega ülesande lahendamiseks, peame uurima $x$ kujul $a+25b$, $a\in\{3,8,13,18,23\}$. Iga sellise kuju korral peaksime leidma $b$ valemist $f'(a)b+\frac{f(a)}{5^2}\equiv 0 \pmod{5}$. Kuna $f'(a)\equiv f'(3)=0\pmod{5}$, siis lahenduvus ei sõltu $ b$-st ning peame kontrollima, kas mõne $a$ puhul $\frac{f(a)}{5^2}\equiv 0 \pmod{5}$ ehk teisisõnu, kas $125\mid f(a)$. $f(3)=175$, $f(8)=6250$, $f(13)=37673$, $f(18)=128950$, $f(23)=329575$ Neist jagub 125-ga ainult 6250, seega on lahenditeks $8,8+25,8+50,8+75,8+100$.


\bigskip

\noindent 5. Lahendada kongruents 
\[
x^4 + 4 x^3 + 2 x^2 + 2 x +12\equiv 0\pmod{1925}.
\]

\smallskip
Kuna 1925=$25\cdot 7\cdot 11$, saame polünoomi lahendada igaühe nende järgi eraldi.
Kuna $-38\equiv 12 \pmod {25}$, teame eelmisest ülesandest, et lahendid on 3, 8, 13, -7, -2 (tähistame $a$). 
Proovides 7 järgi, saame, et lahenditeks on 1 ja 5 (tähistame $b$).
11 järgi on ainult 9.

HJT-st saame, et kõik vastused saame kujul $77\cdot 13a+275\cdot 4b+175\cdot -1\cdot 9$, seega need on 218, 383, 603, 768, 988, 1153, 1373, 1538, 1758, 1923.
\bigskip

\pagebreak

\noindent 6. Lahendada m\~oistatus $\ddot{U}KS\times \ddot{U}KS=2\ast\ast\,\,2\,\,1$. (Iga täht t\"ahistab \"uhte konkreetset numbrit ja $\ast$ tähistab suvalist, võib-olla erinevat numbrit.)

\bigskip
Leian alustuseks lahendid võrrandile $x^2\equiv21\pmod{100}$ ehk $x^2-21\equiv0\pmod{100}$ ning leian hiljem nende hulgast arvud, mis sobivad ülejäänud tingimustega kokku.\\
Mooduli saab lahti tegurdada $100=2^2\cdot5^2$, seega leian 4 ja 5 järgi lahendid: $x^2-21\equiv x^2-1\equiv0\pmod{4}$, mille lahenditeks saab läbiproovimisel $x=1$ ja $x=3$, ning $x^2-21\equiv x^2-1\equiv0\pmod5$, mille lahenditeks saab  $x=1$ ja $x=4$.\\
Leian nüüd mooduli 25 järgi lahendeid kujul $x=1+5y$. Arvestades et $(x^2-21)'=2x$, saame $x=1$ korral $f(1)=-20$ ja $f'(1)=2$. Nendest saab võrrandi $2y+\frac{-20}{5}\equiv2y+1\equiv0\pmod5$, mille ainsaks lahendiks on $y=2$. Seega mooduli 25 järgi on lahendid $y=2+5z,\text{ kus }z\in\Z$ ehk algse kongruentsi lahenditeks saab $1+5(2+5x)=1+10+25x\equiv11\pmod{25}$ ehk $x\equiv11\pmod{25}$.\\
Teiseks leian lahendid kujul $x=4+5y$. Saan $f(4)=-5$ ja $f'(4)=8\equiv3\pmod5$, millest tuleb võrrand $3y+\frac{-5}{5}$, mille ainsaks lahendiks on $y=2$. Seega mooduli 25 järgi on lahendid $y=2+5z$ ehk algse kongruentsi lahenditeks saab $4+5(2+5z)=4+10+25z\equiv-11\pmod{25}$ ehk $x\equiv-11\pmod{25}$.\\
Kokkuvõttes on olemas neli süsteemi, 
\begin{gather*}
\left\{
\begin{aligned}
&x\equiv a_1\pmod4\\
&x\equiv a_2\pmod{25}
\end{aligned}
\right.
\end{gather*}
Kus $a_1\in\{1,3\}$ ja $a_2\in\{11,-11\}$. On lihtne näha, et nendega saab lahendid $x_1\equiv61\pmod{100}$, $x_2\equiv89\pmod{100}$, $x_3\equiv11\pmod{100}$ ja $x_4\equiv39\pmod{100}$. Seega $\ddot{U}KS=100z+x_i$, kusjuures $\ddot{U}=z$ ehk $0<z<10$. Esimeste $z$ väärtustega saab $\ddot{U}KS\times \ddot{U}KS$ tulemuseks\\
\begin{tabular}{c|c|c|c|c}
&$x_1$&$x_2$&$x_3$&$x_4$\\
\hline
$z=1$&25921&35721&12321&19321\\
$z=2$&68121&83521&44521&57121
\end{tabular}\\
Nendest ainult variant $\ddot{U}KS=161$ on numbriga 2 algav viiekohaline arv, kusjuures $z=2$ puhul on kõik tulemused suuremad kui võimalik tulemus olla saaks ning kui $z$ suurendada, muutuvad korrutised suuremaks ehk rohkem lahendeid ei saa leiduda. Seega on ainus lahend $\ddot{U}KS=161$.
\bigskip

\noindent 7. Olgu $a$ juhuslik täisarv vahemikust $[1,17]$ ja $b$ samuti juhuslik täisarv vahemikust $[1,18]$. Milline on tõenäosus, et kongruentsil $ax\equiv b\pmod{18}$ on vähemalt üks lahend? Täpselt üks lahend?

\bigskip
Lause 6.2 põhjal on antud kongruents lahenduv parajasti siis, kui $(a,18)\mid b$. Seega kui $a=9$, peab $b$ olema 9 kordne, milleks on $\left\lfloor\frac{18}{9}\right\rfloor=2$ võimalust. Kui $a$ on 6 kordne, milleks on $\left\lfloor\frac{17}{6}\right\rfloor=2$ võimalust, peab ka $b$ olema 6 kordne, milleks on $\left\lfloor\frac{18}{6}\right\rfloor=3$ võimalust. Kui $a$ on 3 kordne, kuid mitte 6 ega 9 kordne, on selleks võimalusi $\left\lfloor\frac{17}{3}\right\rfloor-1-2=2$ ning $b$ peab olema 3 kordne, selleks on $\left\lfloor\frac{18}{3}\right\rfloor=6$ võimalust. Kui $a$ on 2 kordne kuid mitte 6, on selleks $\left\lfloor\frac{17}{2}\right\rfloor-2=6$ võimalust, ning $b$ peab siis olema 2 kordne, milleks on $\left\lfloor\frac{18}{2}\right\rfloor=9$ võimalust. Ülejäänud $a$ väärtuste puhul $(a,18)=1$ ehk sobivad kõik $b$ väärtused, neid $a$ väärtuseid on $17-1-2-2-6=6$. Seega on kokku tõenäosus et lahendeid leidub $\frac{1}{17}\frac{2}{18}+\frac{2}{17}\frac{3}{18}+\frac{2}{17}\frac{6}{18}+\frac{6}{17}\frac{9}{18}+\frac{6}{17}=\frac{182}{17\cdot18}=\frac{91}{153}$.\\
Et lahendeid oleks täpselt 1, peab kehtima $(a,n)=1$. Selle jaoks on 6 $a$ väärtust ehk tõenäosus on $\frac{6}{17}$.
\bigskip

\pagebreak
\noindent 8. T\~oestada, et kongruentsil $x^2\equiv 1 \pmod{2^k}$ on \"uks lahend, kui $k=1$, kaks lahendit, 
kui $k=2$, ning neli lahendit, kui $k\geq 3$.

\bigskip
Lihtsal läbivaatlusel on näha, et $k=1$ puhul on ainus lahend 1, $k=2$ puhul lahendid 1 ja 3 ning $k=3$ puhul 1, 3, 5 ja 7. Pakun nüüd, et iga $k\geq3$ puhul on 4 lahendit $1,2^{k-1}-1,2^{k-1}+1,-1$ ning tõestan seda induktsiooniga. Baas on üleeelmises lauses antud. Sammuks eeldan, et $k$ puhul leiduvad lahendid $L=\{1,2^{k-1}-1,2^{k-1}+1,-1\}$ ning leian lahendid $k+1$ jaoks.\\
Arvutades välja tuletise $(x^2-1)'=2x\equiv0\pmod2$ on näha, et kui leida lahendit võttes aluseks ühe võrra väiksema mooduli astmega lahend $a\in L$, siis uueks lahendiks on $x=a+2^ky$, kus $y$ on järgneva võrrandi lahend: $f'(a)y+\frac{f(a)}{2^k}\equiv0\pmod{2}$, kuid kuna $f'(a)\equiv0\pmod2$, jääb sellest võrrandist alles $\frac{f(a)}{2^k}\equiv0\pmod{2}$ ehk $\frac{a^2-1}{2^k}\equiv0\pmod{2}$. Kontrollin selle kõigi $L$ liikmete puhul läbi.\\
Kui $a=1$, siis $\frac{1-1}{2^k}\equiv0\pmod{2}$ kehtib ehk saame lahendid mõlema võimaliku $y$ väärtuse jaoks: $x_1=1+0\cdot2^k$ ja $x_2=1+1\cdot2^k$.\\
Kui $a=2^{k-1}-1$, siis $\frac{(2^{k-1}-1)^2-1}{2^k}\equiv\frac{2^{2k-2}-2^k}{2^k}\equiv-1\not\equiv0\pmod{2}$ ehk siit lahendeid ei tule.\\
Kui $a=2^{k-1}+1$, siis $\frac{(2^{k-1}+1)^2-1}{2^k}\equiv\frac{2^{2k-2}+2^k}{2^k}\equiv1\not\equiv0\pmod{2}$ ehk siit lahendeid ei tule.\\
Kui $a=-1$, siis $\frac{(-1)^2-1}{2^k}\equiv0\pmod{2}$ kehtib ehk saame lahendid mõlema võimaliku $y$ väärtuse jaoks: $x_3=-1+0\cdot2^k$ ja $x_4=-1+1\cdot2^k$.\\
Seega on saadud lahendite hulk $1,2^k-1,2^k+1,-1$, mida oligi vaja näidata.

\end{document}