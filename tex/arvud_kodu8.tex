\documentclass[a4paper, 10pt]{article}
\usepackage[estonian]{babel}
\usepackage{t1enc}
\usepackage{amsthm}
\usepackage{amscd}
\usepackage{amssymb}
\usepackage{lscape}
\usepackage{amsfonts}
\usepackage{amsmath}
\usepackage{mathtools}
\usepackage{systeme}
\usepackage{polynom}
\usepackage[shortlabels]{enumitem}
\usepackage[a4paper,margin=1in,footskip=0.25in]{geometry}
\usepackage{pgffor}
\everymath{\displaystyle}
\DeclarePairedDelimiter\ceil{\lceil}{\rceil}
\newcommand{\p}[1]{\frac{\partial}{\partial #1}}
\newcommand{\Z}{\mathbb{Z}}
\newcommand{\N}{\mathbb{N}}
\newcommand{\w}{\overline}
\topmargin-3em
\oddsidemargin0cm
\textwidth16cm
%\textheight27cm
\evensidemargin-2cm
\begin{document}
\begin{center}
\Large\textbf{Kodutöö nr. 7}\\
\small{Joosep Näks ja Uku Hannes Arismaa}
\end{center}

\bigskip

\noindent 1. Lahendada kongruents 
\[
3x^4 + 5 x^3 - x^2 - x +1\equiv 0\pmod{7}.
\]

\smallskip

\noindent 2. Tegurdada pol\"unoom  
\[
f(x)=2x^5+6x^4+5x^3-3x^2-3x+3
\] 
mooduli 5 j\"argi, s.t. \"ule korpuse $\Z_{5}$.

\bigskip

\noindent 3. Milliste $x$ täisarvuliste väärtuste korral on arvu $2x^4+x^3-2x^2+x-2$ mõlemad viimased kümnendnumbrid 2?

\bigskip

\noindent 4. Lahendada kongruents 
\[
x^4 + 4 x^3 + 2 x^2 + 2 x -38\equiv 0\pmod{125}.
\]

\smallskip

\noindent 5. Lahendada kongruents 
\[
x^4 + 4 x^3 + 2 x^2 + 2 x +12\equiv 0\pmod{1925}.
\]

\smallskip

\noindent 6. Lahendada m\~oistatus $\ddot{U}KS\times \ddot{U}KS=2\ast\ast\,\,2\,\,1$. (Iga täht t\"ahistab \"uhte konkreetset numbrit ja $\ast$ tähistab suvalist, võib-olla erinevat numbrit.)

\bigskip
Leian alustuseks lahendid võrrandile $x^2\equiv21\pmod{100}$ ehk $x^2-21\equiv0\pmod{100}$ ning leian hiljem nende hulgast arvud, mis sobivad ülejäänud tingimustega kokku.\\
Mooduli saab lahti tegurdada $100=2^2\cdot5^2$, seega leian 4 ja 5 järgi lahendid: $x^2-21\equiv x^2-1\equiv0\pmod{4}$, mille lahenditeks saab läbiproovimisel $x=1$ ja $x=3$, ning $x^2-21\equiv x^2-1\equiv0\pmod5$, mille lahenditeks saab  $x=1$ ja $x=4$.\\
Leian nüüd mooduli 25 järgi lahendeid kujul $x=1+5y$. Arvestades et $(x^2-21)'=2x$, saame $x=1$ korral $f(1)=-20$ ja $f'(1)=2$. Nendest saab võrrandi $2y+\frac{-20}{5}\equiv2y+1\equiv0\pmod5$, mille ainsaks lahendiks on $y=2$. Seega mooduli 25 järgi on lahendid $y=2+5z,\text{ kus }z\in\Z$ ehk algse kongruentsi lahenditeks saab $1+5(2+5x)=1+10+25x\equiv11\pmod{25}$ ehk $x\equiv11\pmod{25}$.\\
Teiseks leian lahendid kujul $x=4+5y$. Saan $f(4)=-5$ ja $f'(4)=8\equiv3\pmod5$, millest tuleb võrrand $3y+\frac{-5}{5}$, mille ainsaks lahendiks on $y=2$. Seega mooduli 25 järgi on lahendid $y=2+5z$ ehk algse kongruentsi lahenditeks saab $4+5(2+5z)=4+10+25z\equiv-11\pmod{25}$ ehk $x\equiv-11\pmod{25}$.\\
Kokkuvõttes on olemas neli süsteemi, 
\begin{gather*}
\left\{
\begin{aligned}
&x\equiv a_1\pmod4\\
&x\equiv a_2\pmod{25}
\end{aligned}
\right.
\end{gather*}
Kus $a_1\in\{1,3\}$ ja $a_2\in\{11,-11\}$. On lihtne näha, et nendega saab lahendid $x_1\equiv61\pmod{100}$, $x_2\equiv89\pmod{100}$, $x_3\equiv11\pmod{100}$ ja $x_4\equiv39\pmod{100}$. Seega $\ddot{U}KS=100z+x_i$, kusjuures $\ddot{U}=z$ ehk $0<z<10$. Esimeste $z$ väärtustega saab $\ddot{U}KS\times \ddot{U}KS$ tulemuseks\\
\begin{tabular}{c|c|c|c|c}
&$x_1$&$x_2$&$x_3$&$x_4$\\
\hline
$z=1$&25921&35721&12321&19321\\
$z=2$&68121&83521&44521&57121
\end{tabular}\\
Nendest ainult variant $\ddot{U}KS=161$ on numbriga 2 algav viiekohaline arv, kusjuures $z=2$ puhul on kõik tulemused suuremad kui võimalik tulemus olla saaks ning kui $z$ suurendada, muutuvad korrutised suuremaks ehk rohkem lahendeid ei saa leiduda. Seega on ainus lahend $\ddot{U}KS=161$.
\bigskip

\noindent 7. Olgu $a$ juhuslik täisarv vahemikust $[1,17]$ ja $b$ samuti juhuslik täisarv vahemikust $[1,18]$. Milline on tõenäosus, et kongruentsil $ax\equiv b\pmod{18}$ on vähemalt üks lahend? Täpselt üks lahend?

\bigskip
Lause 6.2 põhjal on antud kongruents lahenduv parajasti siis, kui $(a,18)\mid b$. Seega kui $a=9$, peab $b$ olema 9 kordne, milleks on $\left\lfloor\frac{18}{9}\right\rfloor=2$ võimalust. Kui $a$ on 6 kordne, milleks on $\left\lfloor\frac{17}{6}\right\rfloor=2$ võimalust, peab ka $b$ olema 6 kordne, milleks on $\left\lfloor\frac{18}{6}\right\rfloor=3$ võimalust. Kui $a$ on 3 kordne, kuid mitte 6 ega 9 kordne, on selleks võimalusi $\left\lfloor\frac{17}{3}\right\rfloor-1-2=2$ ning $b$ peab olema 3 kordne, selleks on $\left\lfloor\frac{18}{3}\right\rfloor=6$ võimalust. Kui $a$ on 2 kordne kuid mitte 6, on selleks $\left\lfloor\frac{17}{2}\right\rfloor-2=6$ võimalust, ning $b$ peab siis olema 2 kordne, milleks on $\left\lfloor\frac{18}{2}\right\rfloor=9$ võimalust. Ülejäänud $a$ väärtuste puhul $(a,18)=1$ ehk sobivad kõik $b$ väärtused, neid $a$ väärtuseid on $17-1-2-2-6=6$. Seega on kokku tõenäosus et lahendeid leidub $\frac{1}{17}\frac{2}{18}+\frac{2}{17}\frac{3}{18}+\frac{2}{17}\frac{6}{18}+\frac{6}{17}\frac{9}{18}+\frac{6}{17}=\frac{182}{17\cdot18}=\frac{91}{153}$.\\
Et lahendeid oleks täpselt 1, peab kehtima $(a,n)=1$. Selle jaoks on 6 $a$ väärtust ehk tõenäosus on $\frac{6}{17}$.
\bigskip

\noindent 8. T\~oestada, et kongruentsil $x^2\equiv 1 \pmod{2^k}$ on \"uks lahend, kui $k=1$, kaks lahendit, 
kui $k=2$, ning neli lahendit, kui $k\geq 3$.

\bigskip
Lihtsal läbivaatlusel on näha, et $k=1$ puhul on ainus lahend 1, $k=2$ puhul lahendid 1 ja 3 ning $k=3$ puhul 
\bigskip

\end{document}