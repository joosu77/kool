\documentclass[a4paper, 10pt]{article}
\usepackage[estonian]{babel}
\usepackage{t1enc}
\usepackage{amsthm}
\usepackage{amscd}
\usepackage{amssymb}
\usepackage{lscape}
\usepackage{amsfonts}
\usepackage{amsmath}
\usepackage{diagbox}
\usepackage[official]{eurosym}
\usepackage{mathtools}
\usepackage{systeme}
\usepackage{polynom}
\usepackage[shortlabels]{enumitem}
\usepackage[a4paper,margin=1in,footskip=0.25in]{geometry}
\usepackage{pgffor}
\everymath{\displaystyle}
\DeclarePairedDelimiter\ceil{\lceil}{\rceil}
\newcommand{\p}[1]{\frac{\partial}{\partial #1}}
\newcommand{\Z}{\mathbb{Z}}
\newcommand{\N}{\mathbb{N}}
\newcommand{\B}{\mathbb{P}}
\newcommand{\w}{\overline}
\topmargin-3em
\oddsidemargin0cm
\textwidth16cm
%\textheight27cm
\evensidemargin-2cm
\begin{document}
\begin{center}
\Large\textbf{Kodutöö nr. 10}\\
\small{Joosep Näks ja Uku Hannes Arismaa}
\end{center}

\bigskip

\noindent 1. Leida elementide $\overline {9}$, $\overline{19}$, $\overline{29}$, $\overline{39}$, $\overline{40}$ ja $\overline{41}$ j\"argud r\"uhmas $U(\Z_{58})$. Kas m\~oni arvudest $9$, $19$, $29$, $39$, $40$ või $41$ on algjuur mooduli $58$ j\"argi?

\bigskip

\noindent 2. Olgu meil 40-st m\"angukaardist koosnev kaardipakk (näiteks \v{s}veitsi regionaalsest kaardimängust \emph{Kaiserspiel}). 
Nummerdame kaardid \"ulemi\-se\-st alumiseni numbritega $1,2,\ldots,40$. 
V\~otame pakist \"ulemise poole ja asetame lauale alumisest 
poolest paremale. Moodustame uue kaardipaki, v\~ottes j\"arjest 
ülemisi kaarte vasakpoolsest ja parempoolsest pakist. Sellisel 
viisil kaardipaki segamist illustreerib j\"argmine tabel:
\begin{center}
\begin{tabular}{|l|c|c|c|c|c|c|c|c|c|c|c|}
\hline
koht vanas pakis & 1 & 2 & 3 & \ldots & 20 & 21 & 22 & 23 & 
24 & \ldots & 40 \\ \hline
koht uues pakis & 2 & 4 & 6 & \ldots & 40 & 1 & 3 & 5 & 
7 & \ldots & 39 \\ \hline
\end{tabular}.
\end{center}
Mitu korda peab pakki niimoodi segama, et kaardid oleksid 
j\"alle esialgses j\"arjekorras? 

\bigskip

\noindent 3. Näidata otse, {\bf kõiki} jäägiklassiringi $\mathbb{Z}_{30}$ elemente järjest astendades, et mooduli $30$ järgi ei leidu algjuuri. 

\bigskip

\noindent 4. Leida kõik algjuured moodulite $8$, $9$, $12$, $14$ ja $18$ järgi. 

\bigskip
Tegurdan moodulid: $8=2^3$, $9=3^2$, $12=2^2\cdot3$, $14=2\cdot7$, $18=2\cdot3^2$. Teoreemi 7.21 järgi leidub algjuuri vaid moodulite järgi, mis avalduvad kujul 2, 4, $p^k$ või $2\cdot p^k$, seega ei leidu moodulite 8 ja 12 järgi ühtegi moodulit.\\
\indent Läbivaatlusel on näha, et mooduli 3 järgi on ainsaks algjuureks $-1$, seega teoreemi 7.14 järgi on 9 järgi vähemalt üks arvudest $-1$ ja $-1+3=2$ algjuur. On teada, et kui $a$ on algjuur, siis $a^k$ on algjuur parajasti siis, kui $(k,\varphi(m))=1$, kus $m$ on moodul. Kuna $\varphi(9)=6$ ja ainsad sobivad arvud, mis on väiksemad kui 6, on 1 ja 5. Seega on algjuured $\w{2^1}=\w2$ ja $\w{2^5}=\w5$.\\
\indent Mooduli 14 jaoks leian kõigepealt mooduli 7 järgi algjuure. Et $\varphi(7)=6=2\cdot3$, on element $a$ algjuur parajasti siis, kui $a^2\not\equiv1\pmod7$ ja $a^3\not\equiv1\pmod7$. Hakkan läbi proovima: $2^2=4$, $2^3\equiv1\pmod7$ ehk 2 ei sobi, $3^2\equiv2\pmod7$ ja $3^3\equiv-1\pmod7$ ehk 3 on algjuur. Mooduli 14 järgi on üks algjuur seega paaritu arv arvudest 3 või $3+7=10$ ehk 3. Kuna jällegi $\varphi(14)=6$, on arvuga 6 suurim ühistegur arvudel 1 ja 5 ehk algjuurteks $\w3$ ja $\w{3^5}=\w5$.\\
\indent Leidsin juba, et mooduli 9 järgi on algjuured 2 ja 5, ehk 18 järgi on üks algjuur $2+9=11$ ja teine 5 (kuna 2 on paarisarv ja 5 paaritu). Kõigi algjuurte kogus on aga $\varphi(\varphi(18))=2$ ehk 5 ja 11 ongi kõik algjuured.
\bigskip

\noindent 5. Olgu $a\in\Z$, $m,n\geq 2$ kõik kolm paarikaupa ühistegurita. Tõestada, et elemendi $\overline{a}$ järk rühmas $U(\Z_{mn})$ on vähim ühiskordne tema järkudest rühmades $U(\Z_{m})$ ja $U(\Z_{n})$. Kas väide jääb kehtima, kui mõni eeldustest on rikutud?

\bigskip

\noindent 6. T\~oestada, et kui $a$ on algjuur mooduli $n$ j\"argi ja $ab\equiv 1\pmod{n}$, siis ka $b$ on algjuur mooduli $n$ j\"argi (s.t. algjuure pöördväärtus on algjuur.)

\bigskip
Kui korrutada võrrandi $ab\equiv1\pmod n$ mõlemad pooled arvuga $(ab)^{k-1}$ läbi, kus $k$ on positiivne täisarv, saame $(ab)^k\equiv1\pmod n$ ehk $a^kb^k\equiv1\pmod n$. Eeldame, et $b$ ei ole algjuur. Sel juhul on tema järk $m$ väiksem kui arvu $a$ järk $t$, ehk $b^m\equiv1\pmod n$. Võttes teisendatud võrrandis $k=m$, saame et $a^m\cdot1\equiv1\pmod n$ ehk $a^m\equiv1\pmod n$, kuid definitsiooni järgi on $t$ vähim naturaalarv, mille puhul kehtib $a^t\equiv1\pmod n$, ehk kuna $m<t$, oleme saanud vastuolu, nii et arvude $a$ ja $b$ järgud peavad samad olema, ehk kui $a$ on algjuur, on ka $b$ algjuur.
\bigskip

\noindent 7. Kasutades fakti, et algarvulise mooduli järgi leidub algjuuri, tõestada \\\emph{Wilsoni teoreem}:
$p\in\N$ on algarv siis ja ainult siis, kui 
$$
(p-1)!\equiv -1 \pmod{p}.
$$

\smallskip

\noindent 8. Olgu $p$ algarv kujul $4k+3$ ja $a\in\Z$. Tõestada, et $a$ on algjuur mooduli $p$ järgi parajasti siis, kui $\overline{-a}$ järk rühmas  $U(\Z_p)$ on $\frac{p-1}{2}$. 

\bigskip
Olgu $a$ algjuur mooduli $p$ järgi. Siis tema järk on $\varphi(p)=p-1$. Kuna tegu on algarvuga kujul $4k+3$, on tegu paaritu arvuga, ehk $p-1$ on paarisarv. Tean, et $-1$ on $\Z_p$ pööratav element, kuna $(p-1,p-2)=1$, sest tegu on järjestike arvudega ning nende suurim ühistegur peaks jagama nende vahet ja ainus arv mis jagab arvu 1 on 1. Kuna $a$ on algtegur, on arvud $a^1,a^2,..,a^{p-1}$ kõik erinevad ning nende hulka kuulub ka $-1$  $(-1)^2=1$
\bigskip

\noindent 9${^*}$. Leida kõik naturaalarvud $n$, mille korral $\frac{2^n+1}{n^2}$ on täisarv. 

\bigskip

\noindent 10${^{**}}$. Olgu $p$ algarv ja olgu iga naturaalarvu $i$ korral $r_i$ jääk, mis tekib arvu $i^i$ jagamisel arvuga $p$. Tõestada, et jada $(r_i)$ on perioodiline ja leida selle perioodi minimaalne pikkus. 
\end{document}