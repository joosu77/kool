\documentclass{article}
\usepackage{amsfonts}
\usepackage{amsmath}
\usepackage{mathtools}
\DeclarePairedDelimiter\ceil{\lceil}{\rceil}
\begin{document}
\begin{center}
\Large\textbf{Kodutöö nr. 1}\\
5. variant
\end{center}
\textbf{1.} Olgu X reaalarvude hulga mittet\"uhi \"ulalt t\~okestatud alamhulk ning olgu a positiivne arv. T\"ahistame $u := supX$. Toetudes arvhulga rajade definitsioonile v\~oi loengukonspekti lausele 1.3, t\~oestage, et hulk $aX := \{ax: x\in X\}$ on \"ulalt t\~okestatud, kusjuures sup(aX) = au.\\
\textbf{T\~oestus:}\\
Loengukonspekti lause 1.3 kohaselt kehtib $u = supX$ parajasti siis kui kehtivad\\
\begin{equation}\label{t1}
x\leq u\ \forall x\in X
\end{equation}
\begin{equation}\label{t2}
\forall \varepsilon\in \mathbb{R}\textup{, mis rahuldab v\~orratust }\varepsilon>0\quad \exists x_0\in X\quad x_0-\varepsilon<u\,\
\end{equation}
Tingimusest (\ref{t1}) saame v\~orratuse korrutamise monotoonsuse p\~ohjal $ax\leq au$.\\
Tingimusest (\ref{t2}) saame v\~orratuse korrutamise monotoonsuse p\~ohjal\\ $a(x_0-\varepsilon)<au\ \Rightarrow\ ax_0-a\varepsilon<au$. Eelnev kehtib iga positiivse $\varepsilon\in F$ korral ning kui korrutada elemendi $\varepsilon$ elemendiga $a^{-1}$-ga, kus a on positiivne, on ka tulemus positiivne korrutamise monotoonsuse t\~ottu ning kuna korpus on korrutamise suhtes kinnine on tulemus ka korpuse F liige. Samuti saab \"oelda et kui $\varepsilon a^{-1}$ on positiivne korpuse F element siis kui seda a-ga korrutada saame ikka positiivse F elemendi korrutamise monotoonsuse t\~ottu ja selle t\~ottu et korpus on korrutamise suhtes kinnine. Seega on suvaline positiivne element $\varepsilon$ samaväärne suvalise positiivse elemendi ja mingi kindla positiivse elemendi korrutisega $\varepsilon a^{-1}$. Seega $ax_0-a\varepsilon<au \iff ax_0-a\varepsilon a^{-1}<au \iff ax_0-\varepsilon<au$. Seega oleme t\~oestanud et kehtivad $ax\leq au$ ja $ax_0-\varepsilon<au$, mis on loengu konspekti lause 1.3 kohaselt t\~oesed parajasti siis, kui kehtib au = sup(aX).\\
\pagebreak\\
\textbf{2.} Jada ($a_n$) on defineeritud j\"argmiselt:\\
\begin{equation*}
a_1=1\quad  ja\quad a_{n+1}=\frac{a_n}{2}+3,\quad n\in \mathbb{N}
\end{equation*}
N\"aidake, et see jada on \"ulevalt t\~okestatud, kusjuures $a_n\leq 6$ iga $x\in\mathbb{N}$ korral. Kas jada $a_n$ on koonduv? jaatava vastuse korral leidke piirväärtus $\displaystyle\lim_{n\to\infty} a_n$.\\
\textbf{T\~oestus:}\\
V\"aidan et jada liiget saab esitada \"uldkujuga $a_n=\frac{3*2^{n}-5}{2^{n-1}}$.\\
T\~oestan matemaatilise induktsiooni kaudu:\\
\textbf{Induktsiooni baas:}\\
Liige $a_1$ \"uldliikme valemi j\"argi:\\
$a_1 = \frac{3*2^1-5}{2^{1-1}} = 1$\\
Liige $a_1$ on ka definitsiooni kohaselt 1.\\
Liige $a_2$ \"uldliikme valemi j\"argi:\\
$a_2 = \frac{3*2^2-5}{2^{2-1}} = 3.5$\\
Liige $a_2$ rekursiivse valemi j\"argi:\\
$a_2 = \frac{1}{2}+3 = 3.5$\\
\textbf{Induktsiooni samm:}\\
$a_k = \frac{3*2^{k}-5}{2^{k-1}}$\\
$a_{k+1} = \frac{\frac{3*2^{k}-5}{2^{k-1}}}{2}+3 = \frac{3*2^{k}-5+3*2^{k}}{2^{k}} = \frac{3*2^{k+1}-5}{2^{k}}$\\
Ning see on sama, mis tuleb \"uldliikme valemit kasutades, seega \"uldliikme valem kehtib. \\
Seega on meil n\"u\"ud jada esitatud kujul $a_n=\frac{3*2^{n}-5}{2^{n-1}}$.\\
V\"aidan vastuv\"aiteliselt et mingi n korral kehtib\\
\begin{equation*}
\begin{aligned}
\frac{3*2^{n}-5}{2^{n-1}}>6\Rightarrow\\
3*2^n-5>6*2^{n-1}\Rightarrow\\
3*2^n-3*2*2^{n-1}>5\Rightarrow\\
3*(2^n-2^n)>5\Rightarrow\\
3>5\\
\end{aligned}
\end{equation*}
J\~oudsin vastuoluni, seega kehtib alati $a_n\leq 6$.\\
V\"aidan, et jada piirv\"a\"artus u = 6. T\~oestus:\\
Definitsiooni kohaselt on a jada $a_n$ piirv\"a\"artus juhul, kui
\begin{equation*}
\begin{aligned}
\forall\varepsilon<0\quad \exists N\ \forall n\geq N: |a_N-u|\leq \varepsilon\Rightarrow\\
|\frac{3*2^{N}-5}{2^{N-1}}-6|\leq\varepsilon\Rightarrow\\
|\frac{3*2^{N}-5-6*2^{N-1}}{2^{N-1}}|\leq\varepsilon\Rightarrow\\
|\frac{-5}{2^{N-1}}|\leq\varepsilon\Rightarrow\\
\frac{5}{2^{N-1}}\leq\varepsilon\Rightarrow\\
2^{N-1}\leq\frac{5}{\varepsilon}\Rightarrow\\
N\leq1+\log_{2}\frac{5}{\varepsilon}
\end{aligned}
\end{equation*}
Seega saab iga $\varepsilon$ kohta leida N definitsiooni tingimus kehtib, nii et jada $a_n$ piirv\"a\"artus on 6.\\
\pagebreak\\
\textbf{3.} L\"ahtudes jada piirv\"a\"artuse definitsioonist, t\~oestage, et\\
\begin{equation*}
\lim_{n\to\infty}(n^2-(-1)^n)=\infty
\end{equation*}
\textbf{T\~oestus:}\\
Piirv\"a\"artuse definitsiooni kohaselt\\
\begin{equation*}
\begin{aligned}
\lim_{n\to\infty}n=\infty\iff\forall M\quad\exists N\forall n\geq N\quad x_n>M\\
\textup{Alati kehtib }(-1)^n \leq 1\Rightarrow-(-1)^n\geq-1 \Rightarrow n^2-(-1)^n\geq n^2-1\\
\textup{Saan alati v\~otta }N=\ceil{M}+2\ \forall M\\
N^2-(-1)^N\geq N^2-1 = \ceil M +2-1 = \ceil M +1 > M\\
\end{aligned}
\end{equation*}
Seega saan alati v\~otta sellise N et $a_N>M$.\\
Sama kehtib ka iga $n>N$ korral, kuna $a_n$ on kasvav jada:\\
\begin{equation*}
\begin{aligned}
a_n \geq n^2-1\\
(-1)^n \geq -1 \Rightarrow -(-1)^n\leq1\Rightarrow (n-1)^2-(-1)^{n-1}\leq (n-1)^2+1 \Rightarrow\\
a_{n-1} \leq (n-1)^2+1 = n^2-2n+2\\
n>2\Rightarrow n^2-2n+2<n^2-4+2=n^2-2<n^2-1\leq a_n
\end{aligned}
\end{equation*}
Seega iga $n>2$ puhul on $a_n$ kasvav jada.\\
$n=1$ puhul\\
$a_1 = 1^2-(-1)^1=1+1=2$\\
$a_2 = 2^2-(-1)^2=4+1=5$\\
Seega on jada kasvav ka esimestel liikmetel.
Kokkuv\~ottes iga M jaoks leidub N nii et $a_N>M$ ja $a_n$ on kasvav jada seega on jada piirv\"a\"artus l\~opmatus.
\end{document}