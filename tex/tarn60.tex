\documentclass{article}
\usepackage{amsfonts}
\usepackage{amsmath}
\usepackage{mathtools}
\DeclarePairedDelimiter\bfloor{\Bigl\lfloor}{\Bigr\rfloor}
\DeclarePairedDelimiter\floor{\lfloor}{\rfloor}
\begin{document}
\begin{center}
\Large\textbf{T\"arn\"ulesanne nr. 60}\\
\end{center}
Leidke j\"argmised piirv\"a\"artused:
\begin{equation*}
\lim_{x\to0}\frac{x}{a}\bfloor{\frac{b}{x}},\quad\lim_{x\to0}\frac{\floor {x}}{x}.
\end{equation*}
\textbf{Esimese piirv\"a\"artuse lahendus:}\\
P\~orandafunktsiooni definitsiooni kohaselt $\floor{x}=x-d,\quad 0\leq d<1,\quad \floor{x}\in\mathbb{Z}$, seega
\begin{equation*}
\begin{aligned}
\lim_{x\to0}\frac{x}{a}\bfloor{\frac{b}{x}}&=\lim_{x\to0}\frac{x}{a}\Bigl(\frac{b}{x}-d\Bigr{)},\ 0\leq d<1\\
&=\lim_{x\to0}\Bigl(\frac{x}{a}\frac{b}{x}-\frac{x}{a}d\Bigr)\\
&=\lim_{x\to0}\Bigl(\frac{b}{a}-\frac{x}{a}d\Bigr)\\
\lim_{x\to0}\frac{x}{a}d=0\Rightarrow &=\lim_{x\to0}\Bigl(\frac{b}{a}-\frac{x}{a}d\Bigr)=\frac{b}{a}
\end{aligned}
\end{equation*}
Seega $\displaystyle\lim_{x\to0}\frac{x}{a}\bfloor{\frac{b}{x}}=\frac{b}{a}$.\\\\
\textbf{Teise piirv\"a\"artuse lahendus:}\\
Jagan piirv\"a\"artuse kaheks, juhuks $x\to0+$ ja $x\to0-$:
\begin{equation*}
\begin{aligned}
\lim_{x\to0+}\frac{\floor{x}}{x}=\lim_{x\to0+}\frac{0}{x}=0\\
\lim_{x\to0-}\frac{\floor{x}}{x}=\lim_{x\to0-}\frac{-1}{x}=\infty
\end{aligned}
\end{equation*}
Kuna $\lim_{x\to0+}\frac{\floor{x}}{x}\neq\lim_{x\to0-}\frac{\floor{x}}{x}$ pole $\lim_{x\to0}\frac{\floor{x}}{x}$ m\"a\"aratud.
\end{document}