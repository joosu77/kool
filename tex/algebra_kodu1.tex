\documentclass{article}
\usepackage{amsfonts}
\usepackage{amsmath}
\usepackage{mathtools}
\usepackage{systeme}
\usepackage{polynom}
\usepackage{pgfplots}
\everymath{\displaystyle}
\DeclarePairedDelimiter\ceil{\lceil}{\rceil}
\begin{document}
\begin{center}
\Large\textbf{Kodutöö nr. 1}\\
7. variant\\
\small{Joosep Näks}
\end{center}
\textbf{1. } Olgu $A$ mittetühi hulk. Defineerime hulgal $A\times A$ tehte $*$ võrdusega
\begin{equation*}
(a,b)*(c,d)=(a,d).
\end{equation*}
Teha kindlaks, kas see tehe on algebraline tehe, kas ta on assotsiatiivne, kommutatiivne, kas tema suhtes leidub ühikelement, millised elemendid on pööratavad. Millise algebralise struktuuriga on tegemist?\\
\\\textbf{Lahendus:}\\
Tehe on algebraline, kuna elemendid $a$ ja $d$ on hulga $A$ liikmed, seega element $(a,d)$ kuulub hulka $A\times A$.\\
Assotsiatiivsuse kontrollimiseks valin suvalised elemendid $(a,b),(c,d),(e,f)\in A\times A$. Et tehe oleks assotsiatiivne, peab kehtima $((a,b)*(c,d))*(e,f)=(a,b)*((c,d)*(e,f))$.
\begin{gather*}
\begin{aligned}
((a,b)*(c,d))*(e,f)&=(a,d)*(e,f)\\
&=(a,f)\\
&=(a,b)*(c,f)\\
&=(a,b)*((c,d)*(e,f))
\end{aligned}
\end{gather*}
Seega on tehe assotsiatiivne.\\
Kommutatiivsuse jaoks peab suvaliste elementide $(a,b),(c,d)\in A\times A$ korral kehtima $(a,b)*(c,d)=(c,d)*(a,b)$. Kontrollin seda:
\begin{gather*}
\begin{aligned}
(a,b)*(c,d)&=(a,d)\\
(c,d)*(a,b)&=(c,b)\\
(a,d)&\neq(c,b)
\end{aligned}
\end{gather*}
Seega tehe ei ole kommutatiivne.\\
Et leiduks ühikelement, peab leiduma element $(e,f)\in A\times A$, mille korral kehtib $(a,b)*(e,f)=(a,b)=(e,f)*(a,b)$ iga elemendi $(a,b)\in A\times A$ korral. Võrdus $(a,b)*(e,f)=(a,b)$ kehtib aga ainult juhul, kui $f=b$, ehk $(e,f)$ väärtus sõltub alati teisest elemendist, nii et ühikelementi ei leidu.\\
Kuna ühikelementi ei leidu, ei saa ka elemendid olla pööratavad.\\
Seega on tehe algebraline ning sellel on vaid assotsiatiivsuse omadus ehk struktuur $(A\times A,*)$ on poolrühm.\pagebreak\\
\textbf{2.} Olgu $G$ rühm. Tõestada, et kui mistahes $g,h\in G$ korral $gh^2=g$, siis $G$ on Abeli rühm.\\\\
\textbf{Lahendus:}\\
Ainus omadus, mis on Abeli rühmal, kuid ei ole rühmal, on kommutatiivsus, seega üritan näidata, et kui kehtib $gh^2=g$, siis kehtib ka $gh=hg$ suvaliste elementide $g,h\in G$ korral.\\
Rühmas on iga element pööratav, seega võtan elemendi $g$ pöördelemendi $g^{-1}$ ning korrutan sellega võrduse $gh^2=g$ mõlemat poolt vasakult poolt.
\begin{gather*}
\begin{aligned}
g^{-1}gh^2&=g^{-1}g\\
h^2&=1\qquad\quad(1)
\end{aligned}
\end{gather*}
Seega on iga elemendi ruut võrdne ühikelemendiga.\\
Näitan nüüd, et kehtib $gh=hg$.\\
Ühikelemendi definitsiooni järgi asendan korrutises $gh$ elemendi $g$ tema enda ja ühikelemendi korrutisega: $gh=g1h$.\\
Omaduse $(1)$ põhjal asendan ühikelemendi ära elemendi $gh$ ruuduga: $g1h=g(gh)^2h$.\\
Teen ruudu lahti: $g(gh)^2h=gghghh=g^2hgh^2$.\\
Omaduse $(1)$ põhjal asendan elementide $g$ ja $h$ ruudud ühikelemendiga: $g^2hgh^2=1hg1=hg$.\\
Seega jõudsin tulemuseni, et suvaliste elementide korral kehtib $gh=hg$ ehk G on kommutatiivne ehk G on Abeli rühm.
\end{document}