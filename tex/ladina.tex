\documentclass[12pt]{article}
\usepackage{longtable}
\usepackage{amssymb}
\newcommand{\q}[1]{\textbf{#1}}
\newcommand{\w}[1]{$\bar{\mbox{#1}}$}
\newcommand\kcol{12}
\newenvironment{tiitel}
	{\begin{center}
	\bfseries
	\large
	}{
	\end{center}
	}
\begin{document}
\begin{tiitel}
Käänded ja presentsi aktiivi pöörded
\end{tiitel}
\begin{longtable}{*{\kcol}{|c}|}
\hline
\multicolumn{\kcol}{|c|}{Käänded}\\
\hline
&\multicolumn{2}{|c|}{I}&\multicolumn{4}{|c|}{II}&\multicolumn{4}{|c|}{III}\\
\hline
Sugu&\multicolumn{2}{|c|}{f}&m&n&m&n&\multicolumn{2}{|c|}{m/n}&\multicolumn{2}{|c|}{f/n}\\
\hline
Kogus&s&pl&\multicolumn{2}{|c|}{s}&\multicolumn{2}{|c|}{pl}&s&pl&s&pl\\
\hline\hline
Nom.&besti\q{a}&besti\q{ae}&-us, -er&-um&-i&-a&$\varnothing$&-\w{e}s/-a&$\varnothing$&-\w{e}s/-a\\
\hline
Gen.&besti\q{ae}&besti\q{\w{a}rum}&\multicolumn{2}{|c|}{-i}&\multicolumn{2}{|c|}{-\w{o}rum}&-is&-um&-is&-ium\\
\hline
Dat.&besti\q{ae}&besti\q{is}&\multicolumn{2}{|c|}{-\w{o}}&\multicolumn{2}{|c|}{-\w{i}s}&-\w{i}&-ibus&-\w{i}&-ibus\\
\hline
Acc.&besti\q{am}&besti\q{as}&\multicolumn{2}{|c|}{-um}&\multicolumn{2}{|c|}{-\w{o}s}&-em/$\varnothing$&-\w{e}s/-a&-im/$\varnothing$&-\w{e}s/-ia\\
\hline
Abl.&besti\q{\w{a}}&besti\q{is}&\multicolumn{2}{|c|}{-\w{o}}&\multicolumn{2}{|c|}{-\w{i}s}&-e&-ibus&-\w{i}&-ibus\\
\hline
Voc.&besti\q{a}&besti\q{ae}&-e&-um&-\w{i}&-a&\\
\hline
\end{longtable}

Täpsustus III käändkonna kohta:
\begin{itemize}
 \item Teine tulp on -i tüvelised, sinna käivad -e, -al ja -ar lõpuga sõnad ning mõned -is lõpuga sõnad kus gen=nom
 \begin{itemize}
  \item nt. turris, turris; mare, maris
 \end{itemize}
 \item Veel eksisteerivad segatüvelised, kuhu kuuluvad sõnad mille tüve lõpus on konsonantühend, käänduvad nagu konsonanttüvelised aga gen pl on -ium
 \begin{itemize}
  \item nt. hostis, hostis; urbs, urbis
 \end{itemize}
 \item Esimeses tulbas on konsonanttüvelised, sinna lähevad kõik muud III sõnad
 \begin{itemize}
  \item nt. homo, hominis; civitas, civitatis; tempos, temporis
 \end{itemize}
\end{itemize}

\begin{longtable}{*{11}{|c}|}
\hline
\multicolumn{11}{|c|}{Pöörded}\\
\hline
&\multicolumn{2}{|c|}{I}&\multicolumn{2}{|c|}{II}&\multicolumn{4}{|c|}{III}&\multicolumn{2}{|c|}{IV}\\
\hline
I&-o&-amus&-eo&-emus&-o&-imus&-io&-imus&-io&-imus\\
II&-as&-atis&-es&-etis&-is&-itis&-is&-itis&-is&-itis\\
III&-at&-ant&-et&-ent&-it&-unt&-it&-iunt&-it&-iunt\\
\hline
inf&\multicolumn{2}{|c|}{-are}&\multicolumn{2}{|c|}{-\w{e}re}&\multicolumn{4}{|c|}{-ere}&\multicolumn{2}{|c|}{-ire}\\
\hline
\end{longtable}

\pagebreak

\begin{tiitel}
Igasugu muid pöördeid
\end{tiitel}

Erinevaid lõppe:
\begin{longtable}{*{7}{|c}|}
\hline
\multicolumn{3}{|c|}{Perfektilõpud}&\multicolumn{2}{|c|}{imperfekti lõpud}&\multicolumn{2}{|c|}{futuurum II}\\
\hline
&s&pl&s&pl&s&pl\\
\hline
I&-\w{i}&-imus&-m&-mus&-o&-imus\\
\hline
II&-ist\w{i}&-istis&-s&-tis&-is&-itis\\
\hline
III&-it&-\w{e}runt&-t&-nt&-it&-int\\
\hline
\end{longtable}
\begin{itemize}
 \item Verbide põhivormid on oleviku aktiivi indikatiiv, perfekti aktiivi indikatiiv, perfekti passiivne partitsiip (kesksoo vorm) ja oleviku aktiivi infinitiiv
 \item nt. am\w{o}, am\w{a}vi, am\w{a}tum, am\w{a}re
 \item Perfekti aktiivi indikatiiv
 \begin{itemize}
  \item lõpetatud tegevus minevikus, sündmus, tagajärg, liht- või täisminevik
  \item moodustatakse perfektitüvi+lõpp
  \item nt. armastas: amavi, amavisti, amavit
 \end{itemize}
 \item Perfekti passiivne partitsiip
 \begin{itemize}
  \item tud-kesksõna, käitub nagu omadussõna
  \item nt. armastatud: amatus, amata, amatum
 \end{itemize}
 \item Oleviku partitsiip
 \begin{itemize}
  \item v-kesksõna või des-vorm või tegijanimi
  \item moodustamiseks oleviku infinitiivist -re lõpp eemaldada ning I ja II pöördkonna puhul panna -ns, III, IV pöördkonna puhul -ens asemele
  \item -io lõpulistel III pöördkonna verbidel lisandub -ens ette ka i
  \item nt. armastav, armastaja, armastades: am\w{a}ns\\
  rööviv, röövides, röövija: rapiens (inf rapere)
  \item käändub nagu III käändkonna segatüüp, genitiivis tuleb -ns lõpu asemele -ntis
 \end{itemize}
 \item imperfekt
 \begin{itemize}
  \item kestev või korduv tegevus minevikus, lihtminevik
  \item oleviku tüvi + ba(I,II pk)/eba(III,IV pk) + imperfekti pöördelõpp
  \item nt. armastas: am\w{a}bam, am\w{a}b\w{a}s, am\w{a}bat
 \end{itemize}
 \item pluskvamperfekt
 \begin{itemize}
  \item teisele mineviku tegevusele eelnev tegevus, enneminevik
  \item perfekti tüvi + era + imperfekti lõpp
  \item nt. am\w{a}veram, am\w{a}ver\w{a}s, am\w{a}verat
 \end{itemize}
 \item futuurum I
 \begin{itemize}
  \item tegevus tulevikus
  \item oleviku tüvi + b(I,II pk)/e(III,IV pk) + imperfekti pöördelõpp
  \item kui pöördelõpp hakkab konsonandiga, tuleb selle ette i, III pöörde pl puhul u
  \item III ja IV pöördkonnas I pöördes s on -e+o asemel -am
  \item nt. am\w{a}b\w{o}, am\w{a}bis, am\w{a}bit
 \end{itemize}
 \item futuurum II
 \begin{itemize}
  \item lõpetatud tegevus tulevikus, millele järgneb teine tuleviku tegevus
  \item perfekti tüvi + er + futuurum II lõpp
  \item nt. am\w{a}ver\w{o}, am\w{a}veris, am\w{a}verit
 \end{itemize}
\end{itemize}

\pagebreak
\begin{tiitel}
Eessõnad
\end{tiitel}
\begin{itemize}
 \item akkusatiiviga
 \begin{itemize}
  \item ad - juurde, poole, juures, ääres, kuni
  \item ante - enne, varem, ees
  \item apud - juures
  \item circum - ümber, ümbruses, ligidal, juures
  \item contra - vastu
  \item inter - vahel, keskel, seas
  \item per - läbi, mööda, kaudu, jooksul, kestel
  \item post - pärast, peale, taga
  \item trans - üle, sinnapoole, sealpool
 \end{itemize}
 \item ablatiiviga
 \begin{itemize}
  \item a/ab - poolt, juurest, alates
  \item e/ex - seest, -st, -lt, hulgast
  \item cum - koos, ühes, -ga
  \item sine - ilma, -ta
  \item pro - eest, asemel, -ks
  \item prae - ees, tõttu, asemel, -ks
  \item de - pealt, -lt, millegi üle, kohta, -st
 \end{itemize}
 \item akkusatiivi v ablatiiviga
 \begin{itemize}
  \item in (acc) - sisse, -sse, peale, -le
  \item in (abl) - sees, -s, peal, -l
  \item sub (acc) - alla, ligi, juurde
  \item sub (abl) - all, ligi, juures
 \end{itemize}
\end{itemize}
\pagebreak
\begin{tiitel}
Kõneviisid
\end{tiitel}
\begin{itemize}
\item Indikatiiv ehk kindel
\item Imperatiiv ehk käskiv, ainult oleviku teine pööre, harva ka futuurum
\item Konjuktiiv ehk subjunktiiv
\begin{itemize}
\item Soovid, lootused ja kahtlused - nagu tingiv kõneviis (viisakam ja rahulikum kui imperatiiv)
\item käsud ja üleskutsed
\item aktiivi lõpud: -m, -s, -t, -mus, -tis, -nt
\item passiivi lõpud: -r, -ris, -tur, -mur, -mini, -ntur
\item 1. pöördkonnas vahehäälikuks e, teistes a
\item imperfekti moodistamine: imperatiiv + konjuktiivi lõpp 
\item imperfekti olemus: teostamatus v irreaalsus, nt. "Kui ta ometi veel elaks" (tähenduses et ta on kindlalt surnud, oleviku konjuktiivi puhul oleks teadmata), "Milline oleks maailma ilma valguseta?"
\item perfekti aktiivi moodustamine: perfekti tüvi + eri + konjuktiivi lõpp
\item perfekti passiivi moodustamine: perfekti passiivi partitsiip + olema (sim, sis, sit, simus, sitis, sint)
\item perfekti olemus: soov mineviku kohta (nt. "ta ehk tuli"), väite võimalikkus (nt. "kes seda usuks?"), keeld (koos nolitega, nt. "ära kutsu teda")
\item plusquamperfekti aktiivi moodustamine: perfekti infinitiiv + konjuktiivi lõpp
\item plusquamperfekti passiivi moodustamine: perfekti partitsiip + olema konjuktiivi perfekt (essem, esses, esset, essemus, essetis, essent)
\item olemus sama nagu imperfektil aga minevikus, nt "Oleks ta ometi tulnud"
\end{itemize}
\end{itemize}
\end{document}