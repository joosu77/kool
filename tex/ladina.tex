\documentclass[12pt]{article}
\usepackage{longtable}
\usepackage{amssymb}
\newcommand{\q}[1]{\textbf{#1}}
\newcommand{\w}[1]{$\bar{\mbox{#1}}$}
\newcommand\kcol{12}
\begin{document}
\begin{longtable}{*{\kcol}{|c}|}
\hline
\multicolumn{\kcol}{|c|}{Käänded}\\
\hline
&\multicolumn{2}{|c|}{I}&\multicolumn{4}{|c|}{II}&\multicolumn{4}{|c|}{III}\\
\hline
Sugu&\multicolumn{2}{|c|}{f}&m&n&m&n&\multicolumn{2}{|c|}{m/n}&\multicolumn{2}{|c|}{f/n}\\
\hline
Kogus&s&pl&\multicolumn{2}{|c|}{s}&\multicolumn{2}{|c|}{pl}&s&pl&s&pl\\
\hline\hline
Nom.&besti\q{a}&besti\q{ae}&-us, -er&-um&-i&-a&$\varnothing$&-\w{e}s/-a&$\varnothing$&-\w{e}s/-a\\
\hline
Gen.&besti\q{ae}&besti\q{\w{a}rum}&\multicolumn{2}{|c|}{-i}&\multicolumn{2}{|c|}{-\w{o}rum}&-is&-um&-is&-ium\\
\hline
Dat.&besti\q{ae}&besti\q{is}&\multicolumn{2}{|c|}{-\w{o}}&\multicolumn{2}{|c|}{-\w{i}s}&-\w{i}&-ibus&-\w{i}&-ibus\\
\hline
Acc.&besti\q{am}&besti\q{as}&\multicolumn{2}{|c|}{-um}&\multicolumn{2}{|c|}{-\w{o}s}&-em/$\varnothing$&-\w{e}s/-a&-im/$\varnothing$&-\w{e}s/-ia\\
\hline
Abl.&besti\q{\w{a}}&besti\q{is}&\multicolumn{2}{|c|}{-\w{o}}&\multicolumn{2}{|c|}{-\w{i}s}&-e&-ibus&-\w{i}&-ibus\\
\hline
Voc.&besti\q{a}&besti\q{ae}&-e&-um&-\w{i}&-a&\\
\hline
\end{longtable}

Täpsustus III käändkonna kohta:
\begin{itemize}
 \item Teine tulp on -i tüvelised, sinna käivad -e, -al ja -ar lõpuga sõnad ning mõned -is lõpuga sõnad kus gen=nom
 \begin{itemize}
  \item nt. turris, turris; mare, maris
 \end{itemize}
 \item Veel eksisteerivad segatüvelised, kuhu kuuluvad sõnad mille tüve lõpus on konsonantühend, käänduvad nagu konsonanttüvelised aga gen pl on -ium
 \begin{itemize}
  \item nt. hostis, hostis; urbs, urbis
 \end{itemize}
 \item Esimeses tulbas on konsonanttüvelised, sinna lähevad kõik muud III sõnad
 \begin{itemize}
  \item nt. homo, hominis; civitas, civitatis; tempos, temporis
 \end{itemize}
\end{itemize}

\begin{longtable}{*{11}{|c}|}
\hline
\multicolumn{11}{|c|}{Pöörded}\\
\hline
&\multicolumn{2}{|c|}{I}&\multicolumn{2}{|c|}{II}&\multicolumn{4}{|c|}{III}&\multicolumn{2}{|c|}{IV}\\
\hline
I&-o&-amus&-eo&-emus&-o&-imus&-io&-imus&-io&-imus\\
II&-as&-atis&-es&-etis&-is&-itis&-is&-itis&-is&-itis\\
III&-at&-ant&-et&-ent&-it&-unt&-it&-iunt&-it&-iunt\\
\hline
inf&\multicolumn{2}{|c|}{-are}&\multicolumn{2}{|c|}{-\w{e}re}&\multicolumn{4}{|c|}{-ere}&\multicolumn{2}{|c|}{-ire}\\
\hline
\end{longtable}
\pagebreak
Eessõnad:
\begin{itemize}
 \item akkusatiiviga
 \begin{itemize}
  \item ad - juurde, poole, juures, ääres, kuni
  \item ante - enne, varem, ees
  \item apud - juures
  \item circum - ümber, ümbruses, ligidal, juures
  \item contra - vastu
  \item inter - vahel, keskel, seas
  \item per - läbi, mööda, kaudu, jooksul, kestel
  \item post - pärast, peale, taga
  \item trans - üle, sinnapoole, sealpool
 \end{itemize}
 \item ablatiiviga
 \begin{itemize}
  \item a/ab - poolt, juurest, alates
  \item e/ex - seest, -st, -lt, hulgast
  \item cum - koos, ühes, -ga
  \item sine - ilma, -ta
  \item pro - eest, asemel, -ks
  \item prae - ees, tõttu, asemel, -ks
  \item de - pealt, -lt, millegi üle, kohta, -st
 \end{itemize}
 \item akkusatiivi v ablatiiviga
 \begin{itemize}
  \item in (acc) - sisse, -sse, peale, -le
  \item in (abl) - sees, -s, peal, -l
  \item sub (acc) - alla, ligi, juurde
  \item sub (abl) - all, ligi, juures
 \end{itemize}
\end{itemize}
\end{document}