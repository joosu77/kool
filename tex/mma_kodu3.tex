\documentclass{article}
\usepackage{amsfonts}
\usepackage{amsmath}
\usepackage{mathtools}
\usepackage{systeme}
\usepackage{polynom}
\usepackage{pgfplots}
\usepackage[shortlabels]{enumitem}
\everymath{\displaystyle}
\DeclarePairedDelimiter\ceil{\lceil}{\rceil}
\newcommand{\p}[1]{\frac{\partial}{\partial #1}}
\begin{document}
\begin{center}
\Large\textbf{Kodutöö nr. 3}\\
3. variant\\
\small{Joosep Näks}
\end{center}
\textbf{1. }Leida võrrandiga $z^3+z^2xy=1$ punkti $(2,1,-1)$ ümbruses ilmutamata kujul määratud funktsiooni $z=z(x,y)$ osatuletised $z_x'(2,1),\ z_y'(2,1),\ z_{y^2}''(2,1)$.\\\\
\textbf{Lahendus:}\\
Leian funktsiooni, mis võrdub nulliga $F(x,y)=z^3+z^2xy-1=0$.\\
Et sellel funktsioonil leiduks tuletis, peab ta olema punkti $(2,1,-1)$ ümber mingis ristkülikus määratud, osatuletised $x$ ja $y$ järgi peavad selles ristkülikus olema pidevad ning osatuletis $y$ järgi peab olema nullist erinev.\\
Funktsioon on määratud tervel $\mathbb{R}^2$ ruumis, seega on ta määratud ristkülikus punkti $(2,1)$ ümber.\\
Leian funktsiooni osatuletised:
\begin{gather*}
F_z'(x,y)=3z^2+2zxy\\
F_x'(x,y)=3z^2\frac{\partial z}{\partial x}+2zxy\frac{\partial z}{\partial x}+z^2y\\
F_y'(x,y)=3z^2\frac{\partial z}{\partial y}+2zxy\frac{\partial z}{\partial y}+z^2x\\
\end{gather*}
\end{document}