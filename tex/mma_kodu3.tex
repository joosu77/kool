\documentclass{article}
\usepackage{amsfonts}
\usepackage{amsmath}
\usepackage{mathtools}
\usepackage{systeme}
\usepackage{polynom}
\usepackage{pgfplots}
\usepackage[shortlabels]{enumitem}
\everymath{\displaystyle}
\DeclarePairedDelimiter\ceil{\lceil}{\rceil}
\newcommand{\p}[1]{\frac{\partial}{\partial #1}}
\begin{document}
\begin{center}
\Large\textbf{Kodutöö nr. 3}\\
3. variant\\
\small{Joosep Näks}
\end{center}
\textbf{1. }Leida võrrandiga $z^3+z^2xy=1$ punkti $(2,1,-1)$ ümbruses ilmutamata kujul määratud funktsiooni $z=z(x,y)$ osatuletised $z_x'(2,1),\ z_y'(2,1),\ z_{y^2}''(2,1)$.\\\\
\textbf{Lahendus:}\\
Leian funktsiooni, mis võrdub nulliga $F(x,y)=z^3+z^2xy-1=0$.\\
Et leiduksid osatuletised $z_x', z_y', z_{y^2}''$, peavad kehtima tingimused:
\begin {itemize}
\item$F(x,y,z)$ on punkti $(2,1,-1)$ ümber mingis risttahukas määratud
\item funktsioonil $F(x,y,z)$ on pidevad osatuletised $x$ ja $y$ järgi selles risttahukas
\item osatuletis $y$ järgi on punktis $(2,1,-1)$ nullist erinev.
\end{itemize}
Kuna $F(x,y,z)$ on tervel $\mathbb{R}^2$ tasandil määratud elementaarfunktsioonide kompositisoon, on ta määratud terves $\mathbb{R}^3$ ruumis, seega on ta määratud mingis risttahukas $\mathcal{D}$ punkti $(2,1,-1)$ ümber.\\\\
Leian funktsiooni osatuletised:
\begin{gather*}
F_z'(x,y,z)=3z^2+2zxy\\
F_x'(x,y,z)=z^2y\\
F_y'(x,y,z)=z^2x\\
\end{gather*}
$F_x'$, $F_z'$ ja $F_y'$ on tervel reaaltasandil pidevate elementaarfunktisoonide kompostitsioon seega on nad pidevad terves $\mathbb{R}^3$ ruumis ehk nad on ka pidevad selles samas risttahukas $\mathcal{D}$.\\
Kontrollin $F_z'$ väärtust vaadeldavas punktis:\\
$F_z'(2,1,-1)=3(-1)^2+2\cdot(-1)\cdot2\cdot1=3-4=-1\neq0$\\
Seega kõik tingimused kehtivad ehk otsitavad osatuletised leiduvad mingis ümbruses ning avalduvad valemitega $z_x'(x,y)=-\frac{F_x'(x,y,z(x,y))}{F_z'(x,y,z(x,y))}$ ja $z_y'(x,y)=-\frac{F_y'(x,y,z(x,y))}{F_z'(x,y,z(x,y))}$, kusjuures $z(2,1)=-1$.\\
Leian need osatuletised punktis $(2,1,-1)$:
\begin{gather*}
\begin{aligned}
z_x'(2,1)&=-\frac{F_x'(2,1,z(2,1))}{F_z'(2,1,z(2,1))}\\
&=-\frac{(-1)^2\cdot1}{3\cdot(-1)^2+2\cdot(-1)\cdot2\cdot1}\\
&=-\frac{1}{-1}=1\\
z_y'(2,1)&=-\frac{F_y'(2,1,z(2,1))}{F_z'(2,1,z(2,1))}\\
&=-\frac{(-1)^2\cdot2}{3\cdot(-1)^2+2\cdot(-1)\cdot2\cdot1}\\
&=-\frac{2}{-1}=2\\
\end{aligned}
\end{gather*}
Leian $z$ teist järku osatuletise $y$ järgi:
\begin{gather*}
\begin{aligned}
\hspace{-10\parindent}
z_{y^2}''(x,y)&=\p{y}z_y'\\
&=\p{y}\left(-\frac{F_y'(x,y,z(x,y))}{F_z'(x,y,z(x,y))}\right)\\
&=\p{y}\left(-\frac{x\cdot z^2(x,y)}{3z^2(x,y)+2xy\cdot z(x,y)}\right)\\
&=-\frac{2x\cdot z^2(x,y)\cdot z_y'(x,y)\left(3z(x,y)+2xy\right)-2x\cdot z^2(x,y)\left(3z(x,y)z_y'(x,y)+x\cdot z(x,y)+xy\cdot z_y'(x,y)\right)}{(3z^2(x,y)+2xy\cdot z(x,y))^2}\\
\hspace{-10\parindent}
z_{y^2}''(2,1)&=-\frac{2\cdot2\cdot (-1)^2\cdot 2\left(3\cdot(-1)+2\cdot2\cdot1\right)-2\cdot2\cdot (-1)^2\left(3\cdot(-1)\cdot2+2\cdot (-1)+2\cdot1\cdot 2\right)}{(3\cdot(-1)^2+2\cdot2\cdot1\cdot (-1))^2}\\
&=\frac{8-4(-6-2+4)}{1}=24
\end{aligned}
\end{gather*}
Seega $z_x'(2,1)=1$, $z_y'(2,1)=2$ ja $z_{y^2}''(2.1)=24$\pagebreak\\
\textbf{2.} Leida funktsiooni $f(x,y)=e^{-x^2-y^2}(2x^2+3y^2)$ globaalsed ekstreemumid ringis
\begin{gather*}
\mathcal{D}:=\left\{(x,y)\in\mathbb{R}^2:x^2+y^2\leq4\right\}
\end{gather*}
\end{document}