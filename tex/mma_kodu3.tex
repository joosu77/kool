\documentclass{article}
\usepackage{amsfonts}
\usepackage{amsmath}
\usepackage{mathtools}
\usepackage{systeme}
\usepackage{polynom}
\usepackage{pgfplots}
\usepackage[shortlabels]{enumitem}
\everymath{\displaystyle}
\DeclarePairedDelimiter\ceil{\lceil}{\rceil}
\newcommand{\p}[1]{\frac{\partial}{\partial #1}}
\begin{document}
\begin{center}
\Large\textbf{Kodutöö nr. 3}\\
3. variant\\
\small{Joosep Näks}
\end{center}
\textbf{1. }Leida võrrandiga $z^3+z^2xy=1$ punkti $(2,1,-1)$ ümbruses ilmutamata kujul määratud funktsiooni $z=z(x,y)$ osatuletised $z_x'(2,1),\ z_y'(2,1),\ z_{y^2}''(2,1)$.\\\\
\textbf{Lahendus:}\\
Leian funktsiooni, mis võrdub nulliga $F(x,y)=z^3+z^2xy-1=0$.\\
Et leiduksid osatuletised $z_x', z_y', z_{y^2}''$, peavad kehtima tingimused:
\begin {itemize}
\item$F(x,y,z)$ on punkti $(2,1,-1)$ ümber mingis risttahukas määratud
\item funktsioonil $F(x,y,z)$ on pidevad osatuletised $x$ ja $y$ järgi selles risttahukas
\item osatuletis $y$ järgi on punktis $(2,1,-1)$ nullist erinev.
\end{itemize}
Kuna $F(x,y,z)$ on tervel $\mathbb{R}^2$ tasandil määratud elementaarfunktsioonide kompositisoon, on ta määratud terves $\mathbb{R}^3$ ruumis, seega on ta määratud mingis risttahukas $\mathcal{D}$ punkti $(2,1,-1)$ ümber.\\\\
Leian funktsiooni osatuletised:
\begin{gather*}
F_z'(x,y,z)=3z^2+2zxy\\
F_x'(x,y,z)=z^2y\\
F_y'(x,y,z)=z^2x\\
\end{gather*}
$F_x'$, $F_z'$ ja $F_y'$ on tervel reaaltasandil pidevate elementaarfunktisoonide kompostitsioon seega on nad pidevad terves $\mathbb{R}^3$ ruumis ehk nad on ka pidevad selles samas risttahukas $\mathcal{D}$.\\
Kontrollin $F_z'$ väärtust vaadeldavas punktis:\\
$F_z'(2,1,-1)=3(-1)^2+2\cdot(-1)\cdot2\cdot1=3-4=-1\neq0$\\
Seega kõik tingimused kehtivad ehk otsitavad osatuletised leiduvad mingis ümbruses ning avalduvad valemitega $z_x'(x,y)=-\frac{F_x'(x,y,z(x,y))}{F_z'(x,y,z(x,y))}$ ja $z_y'(x,y)=-\frac{F_y'(x,y,z(x,y))}{F_z'(x,y,z(x,y))}$, kusjuures $z(2,1)=-1$.\\
Leian need osatuletised punktis $(2,1,-1)$:
\begin{gather*}
\begin{aligned}
z_x'(2,1)&=-\frac{F_x'(2,1,z(2,1))}{F_z'(2,1,z(2,1))}\\
&=-\frac{(-1)^2\cdot1}{3\cdot(-1)^2+2\cdot(-1)\cdot2\cdot1}\\
&=-\frac{1}{-1}=1\\
z_y'(2,1)&=-\frac{F_y'(2,1,z(2,1))}{F_z'(2,1,z(2,1))}\\
&=-\frac{(-1)^2\cdot2}{3\cdot(-1)^2+2\cdot(-1)\cdot2\cdot1}\\
&=-\frac{2}{-1}=2\\
\end{aligned}
\end{gather*}
Leian $z$ teist järku osatuletise $y$ järgi:
\begin{gather*}
\begin{aligned}
\hspace{-10\parindent}
z_{y^2}''(x,y)&=\p{y}z_y'\\
&=\p{y}\left(-\frac{F_y'(x,y,z(x,y))}{F_z'(x,y,z(x,y))}\right)\\
&=\p{y}\left(-\frac{x\cdot z^2(x,y)}{3z^2(x,y)+2xy\cdot z(x,y)}\right)\\
&=-\frac{2x\cdot z^2(x,y)\cdot z_y'(x,y)\left(3z(x,y)+2xy\right)-2x\cdot z^2(x,y)\left(3z(x,y)z_y'(x,y)+x\cdot z(x,y)+xy\cdot z_y'(x,y)\right)}{(3z^2(x,y)+2xy\cdot z(x,y))^2}\\
\hspace{-10\parindent}
z_{y^2}''(2,1)&=-\frac{2\cdot2\cdot (-1)^2\cdot 2\left(3\cdot(-1)+2\cdot2\cdot1\right)-2\cdot2\cdot (-1)^2\left(3\cdot(-1)\cdot2+2\cdot (-1)+2\cdot1\cdot 2\right)}{(3\cdot(-1)^2+2\cdot2\cdot1\cdot (-1))^2}\\
&=\frac{8-4(-6-2+4)}{1}=24
\end{aligned}
\end{gather*}
Seega $z_x'(2,1)=1$, $z_y'(2,1)=2$ ja $z_{y^2}''(2.1)=24$\pagebreak\\
\textbf{2.} Leida funktsiooni $f(x,y)=e^{-x^2-y^2}(2x^2+3y^2)$ globaalsed ekstreemumid ringis
\begin{gather*}
\mathcal{D}:=\left\{(x,y)\in\mathbb{R}^2:x^2+y^2\leq4\right\}
\end{gather*}
\textbf{Lahendus:}\\
Globaalsed ekstreemumid saavad leiduda funktsiooni (1) kriitilistes punktides ringi $\mathcal{D}$ sisepunktides ja (2) tema võimalikes ekstreemumpunktides ringi $\mathcal{D}$ rajal.\\
\begin{enumerate}
\item $f$ on terves ringis pidev seega kõik tema kriitilised punktid on statsionaarsed punktid. Nende leidmiseks leian $f$ osatuletised:
\begin{gather*}
\begin{aligned}
f_x'(x,y)&=\p{x}\left(e^{-x^2-y^2}(2x^2+3y^2)\right)\\
&=-2xe^{-x^2-y^2}(2x^2+3y^2)+4xe^{-x^2-y^2}\\
&=2xe^{-x^2-y^2}(2-2x^2-3y^2)\\
f_y'(x,y)&=\p{y}\left(e^{-x^2-y^2}(2x^2+3y^2)\right)\\
&=-2ye^{-x^2-y^2}(2x^2+3y^2)+6ye^{-x^2-y^2}\\
&=2ye^{-x^2-y^2}(3-2x^2-3y^2)
\end{aligned}
\end{gather*}
Et punkt $(x_0,y_0)$ oleks statsionaarne punkt, peab kehtima $f_x'(x_0,y_0)=f_y'(x_0,y_0)=0$. Leian selleks osatuletiste nullkohad. $f_x'$ on korrutis seega on ta null parajasti siis kui $x=0$, $e^{-x^2-y^2}=0$ või $2-2x^2-3y^2=0$. $e^{-x^2-y^2}=0$ ei saa kehtida kuna $e^x,\ x\in\mathbb{R}$ saavutab vaid positiivseid väärtuseid. Samuti saab teisest osatuletisest tingimused $y=0$ või $3-2x^2-3y^2=0$. Tingimusi kombineerides saab neli võrrandisüsteemi statsionaarsete punktide leidmiseks:
\begin{gather*}
\begin{aligned}
&\left\{
\begin{aligned}
&x=0\\
&y=0
\end{aligned}
\right.\\
&\left\{
\begin{aligned}
&x=0\\
&3-2x^2-3y^2=0
\end{aligned}
\right.\\
&\left\{
\begin{aligned}
&2-2x^2-3y^2=0\\
&y=0
\end{aligned}
\right.\\
&\left\{
\begin{aligned}
&2-2x^2-3y^2=0\\
&3-2x^2-3y^2=0
\end{aligned}
\right.
\end{aligned}
\end{gather*}
Esimesest süsteemist saab punkti $P_1(0,0)$, teisest punktid $P_2(0,1)$ ja $P_3(0,-1)$, kolmandast punktid $P_4(1,0)$ ja $P_5(-1,0)$. Neljandal võrrandisüsteemil ei leidu ühtegi lahendit. Kõik leitud punktid asuvad ringi sisepunktide hulgas seega on need kõik sobivad statsionaarsed punktid. Leian funktsiooni väärtused nendes punktides: $f(P_1)=0$, $f(P_2)=3e^{-1}$, $f(P_3)=3e^{-1}$, $f(P_4)=2e^{-1}$ ja $f(P_5)=2e^{-1}$. Seega on lokaalne miinimum punktis $P_1$ ja lokaalsed maksimumid punktides $P_2$ ja $P_3$.
\item Ringi rajapunktid esituvad võrrandiga $x^2+y^2=4$. Asendan funktsiooni sisse rajapunktide võrrandi $y^2=4-x^2$ tingimusega $x^2\leq4$ ehk $-2\leq x\leq2$ kuna kui $x^2>4$, siis $y^2=4-x^2<0$ kuid reaalarvu ruut ei saa negatiivne olla.\\
Tingimustest saab otspunktid $P_6(-2,0)$ ja $P_7(2,0)$.\\
Sisse asendades saan ühe muutuja funktsiooni $f(x)=e^{-x^2-(4-x^2)}(2x^2+3(4-x^2))=e^{-4}(12-x^2)$. Võtan sellest tuletise: $f'(x)=2e^{-4}x$. Selle nullkoht on $2e^{-4}x_0=0\Leftrightarrow x_0=0$. $y$ väärtuseks saan $y=\pm\sqrt{4-0}=\pm2$ ehk ekstreemumid on rajapunktides $P_8(0,2)$ ja $P_9(0,-2)$.\\
Leian väärtused nendes punktides: $f(P_6)=8e^{-4}$, $f(P_7)=8e^{-4}$, $f(P_8)=12e^{-4}$ ja $f(P_9)=12e^{-4}$
\end{enumerate}
Seega on ringis $\mathcal{D}$ funktsiooni $f$ globaalne miinimum $f(0,0)=0$ ja globaalsed maksimumid on $f(0,1)=f(0,-1)=3e^{-1}$.\pagebreak\\
\textbf{3.} Kasutades Lagrange'i meetodit, leida funktsiooni $f(x,y,z)=x+4y-2z$ tinglinkud lokaalsed ekstreemumid lisatingimusel
\begin{gather*}
x^2+2y^2+z^2=6,\quad\quad x-2z=0
\end{gather*}
\textbf{Lahendus:}\\
Leian Lagrange'i funktsiooni:
\begin{gather*}
\Phi(x,y,z,\lambda,\mu)=x+4y-2z+\lambda(x^2+2y^2+z^2-6)+\mu(x-2z)
\end{gather*}
Leian selle osatuletised kõigi muutujate järgi:
\begin{gather*}
\begin{aligned}
\Phi_x'(x,y,z,\lambda,\mu)&=1+2x\lambda+\mu\\
\Phi_y'(x,y,z,\lambda,\mu)&=4+4y\lambda\\
\Phi_z'(x,y,z,\lambda,\mu)&=-2+2z\lambda-2\mu\\
\Phi_\lambda'(x,y,z,\lambda,\mu)&=x^2+2y^2+z^2-6\\
\Phi_\mu'(x,y,z,\lambda,\mu)&=x-2z\\
\end{aligned}
\end{gather*}
Kui kõik osatuletised panna nulliga võrduma ning lahendada võrrandisüsteem, saab lahendiks kaks punkti $P_0(0,-\sqrt{3},0,\frac{1}{\sqrt{3}},-1)$ ja $P_1(0,\sqrt{3},0,-\frac{1}{\sqrt{3}},-1)$.\\
Et kontrollida, kas saadud punktides on ekstreemumid, leian $\Phi(x,y,z,0,0)$ hessiaani maatriksi:
\begin{align}
A(P)&=\begin{pmatrix}
\Phi_{xx}''(P) & \Phi_{xy}''(P) & \Phi_{xz}''(P)\\
\Phi_{yx}''(P) & \Phi_{yy}''(P) & \Phi_{yz}''(P)\\
\Phi_{zx}''(P) & \Phi_{zy}''(P) & \Phi_{zz}''(P)\\\notag
\end{pmatrix}\\
&=\begin{pmatrix}
2\lambda & 0 & 0\\
0 & 3\lambda & 0\\
0 & 0 & 2\lambda\notag
\end{pmatrix}
\end{align}
Punktis $P$ on lokaalne miinimum, kui hessiaani maatriks on punktis $P$ positiivselt määratud, ja lokaalne maksimum, kui hessiaani maatriks on negatiivselt määratud. Muul juhul ei asu punktis $P$ lokaalset ekstreemumi.\\
Sylvesteri tunnuse põhjal on hessiaani maatriks positiivselt määratud, kui kõik tema peamiinorid on positiivsed, ning negatiivselt määratud, kui tema esimene peamiinor on negatiivne ja järgmised peamiinorid on vahelduvate märkidega.
Leian peamiinorite väärtused punktis $P_0$:
\begin{align}
A_1(P_0)&= 2\cdot\frac{1}{\sqrt{3}}>0\notag\\
A_2(P_0)&=\begin{vmatrix}
2\cdot\frac{1}{\sqrt{3}} & 0\\
0 & 3\cdot\frac{1}{\sqrt{3}}\notag
\end{vmatrix}=2\cdot\frac{1}{\sqrt{3}}\cdot3\cdot\frac{1}{\sqrt{3}}=2>0\\
A_3(P_0)&=\begin{vmatrix}
2\cdot\frac{1}{\sqrt{3}} & 0 & 0\\
0 & 3\cdot\frac{1}{\sqrt{3}} & 0\\
0 & 0 & 2\cdot\frac{1}{\sqrt{3}}\notag
\end{vmatrix}=2\cdot\frac{1}{\sqrt{3}}\cdot3\cdot\frac{1}{\sqrt{3}}\cdot2\cdot\frac{1}{\sqrt{3}}=\frac{4}{\sqrt{3}}>0
\end{align}
Kõik miinorid on positiivsed ehk punktis $P_0$ on lokaalne miinimum.\\
Leian peamiinorite väärtused punktid $P_1$:
\begin{align}
A_1(P_1)&= 2\left(-\frac{1}{\sqrt{3}}\right)=-\frac{2}{\sqrt{3}}<0\notag\\
A_2(P_1)&=\begin{vmatrix}
2\left(-\frac{1}{\sqrt{3}}\right) & 0\\
0 & 3\left(-\frac{1}{\sqrt{3}}\right)\notag
\end{vmatrix}=2\left(-\frac{1}{\sqrt{3}}\right)\cdot3\left(-\frac{1}{\sqrt{3}}\right)=2>0\\
A_3(P_1)&=\begin{vmatrix}
2\left(-\frac{1}{\sqrt{3}}\right) & 0 & 0\\
0 & 3\left(-\frac{1}{\sqrt{3}}\right) & 0\\
0 & 0 & 2\left(-\frac{1}{\sqrt{3}}\right)\notag
\end{vmatrix}=2\left(-\frac{1}{\sqrt{3}}\right)\cdot3\left(-\frac{1}{\sqrt{3}}\right)\cdot2\left(-\frac{1}{\sqrt{3}}\right)=-\frac{4}{\sqrt{3}}<0
\end{align}
Peamiinorite väärtused on vahelduvate märkidega, seega on hessiaan negatiivselt määratud, ehk punktis $P_1$ on lokaalne maksimum.\\
Seega on ekstreemumid $f(0, -\sqrt{3},0)=-4\sqrt{3}$ ja $f(0, \sqrt{3},0)=4\sqrt{3}$, millest esimene on lokaalne miinimum ja teine lokaalne maksimum.
\end{document}