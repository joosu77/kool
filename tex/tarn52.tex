\documentclass{article}
\usepackage{amsfonts}
\usepackage{amsmath}
\usepackage{mathtools}
\DeclarePairedDelimiter\ceil{\lceil}{\rceil}
\begin{document}
\begin{center}
\Large\textbf{T\"arn\"ulesanne nr. 52}\\
\end{center}
Olgu $(a_n)$ positiivsete liikmetega jada nii, et $\displaystyle\lim_n \frac{a_n}{n}=0$ ja\\ $\displaystyle\overline\lim_n \frac{a_1+a_2+...+a_n}{n}\in\mathbb{R}. Leidke \displaystyle\lim_n \frac{a_1^2+a_2^2+...+a_n^2}{n^2}$\\\\
\textbf{Lahendus:}\\
Teise ja kolmanda limiidi saan lahti kirjutada summadena:
\begin{equation*}
\begin{aligned}
\lim_n \frac{a_1+a_2+...+a_n}{n}=\lim_{c\to\infty} \sum_{n=1}^c \frac{a_n}{c}\\
\lim_n \frac{a_1^2+a_2^2+...+a_n^2}{n^2}=\lim_{c\to\infty} \sum_{n=1}^c \frac{a_n^2}{c^2}
\end{aligned}
\end{equation*}
\end{document}