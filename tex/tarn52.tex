\documentclass{article}
\usepackage{amsfonts}
\usepackage{amsmath}
\usepackage{mathtools}
\DeclarePairedDelimiter\ceil{\lceil}{\rceil}
\begin{document}
\begin{center}
\Large\textbf{T\"arn\"ulesanne nr. 52}\\
\end{center}
Olgu $(a_n)$ positiivsete liikmetega jada nii, et $\displaystyle\lim_n \frac{a_n}{n}=0$ ja\\ $\displaystyle\overline\lim_n \frac{a_1+a_2+...+a_n}{n}\in\mathbb{R}$. Leidke $\displaystyle\lim_n \frac{a_1^2+a_2^2+...+a_n^2}{n^2}$.\\\\
\textbf{Lahendus:}\\
Loon uue jada:
\begin{equation*}
\begin{aligned}
p_n:&=a_n\\
t:&=\lim_n \frac{p_1+p_2+...+p_n}{n}\\
P_n:&=p_1+p_2+...+p_n=\lim_n tn\\
\end{aligned}
\end{equation*}
T\"ahistan kolmanda piirv\"a\"artuse \"umber:
\begin{equation*}
\begin{aligned}
\lim_n \frac{a_1^2+a_2^2+...+a_n^2}{n^2}=\lim_n \frac{a_1*p_1+a_2*p_2+...+a_np_n}{n^2}=\\
\lim_n t\frac{a_1*p_1+a_2*p_2+...+a_np_n}{tn}=\lim_n t\frac{a_1*p_1+a_2*p_2+...+a_np_n}{nP_n}
\end{aligned}
\end{equation*}
Ning loengu konspekti kaalutud keskmiste kohta k\"aiva lause 2.31 kohaselt kui $P_n\rightarrow\infty$:
\begin{equation*}
\begin{aligned}
a_n\rightarrow A\Rightarrow\lim_n \frac{a_1*p_1+a_2*p_2+...+a_np_n}{P_n}= A\iff\\
\lim_n t\frac{a_1*p_1+a_2*p_2+...+a_np_n}{nP_n}= \frac{tA}{n}\\
\end{aligned}
\end{equation*}
\"Ulesandes on \"oeldud et $\displaystyle\lim_n \frac{a_n}{n}=0$ seega $\frac{tA}{n}=t0$ ning kuna t on reaalarv siis saame \"oelda, et $\displaystyle\lim_n \frac{a_1^2+a_2^2+...+a_n^2}{n^2}= 0$. J\"a\"ab alles veel juht kus $P_n$ ei l\"ahene l\~opmatusele, kuid kuna iga $a_n$ on positiivne, peab see t\"ahendama, et $P_n\in\mathbb{R}_+$, ning sel juhul ei saa summa $P_n$ minna l\~opmatusesse ka juhul kui iga liige ruutu v\~otta ning $\displaystyle\lim_n \frac{k}{n^2}=0\quad k\in\mathbb{R}$.\\
Seega on k\"usitud piirv\"a\"artus alati 0.
\end{document}