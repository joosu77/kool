\documentclass{article}
\usepackage{amsfonts}
\usepackage{amsmath}
\usepackage{mathtools}
\DeclarePairedDelimiter\ceil{\lceil}{\rceil}
\begin{document}
\begin{center}
\Large\textbf{T\"arn\"ulesanne nr. 52}\\
\end{center}
Olgu $(a_n)$ positiivsete liikmetega jada nii, et $\displaystyle\lim_n \frac{a_n}{n}=0$ ja\\ $\displaystyle\overline\lim_n \frac{a_1+a_2+...+a_n}{n}\in\mathbb{R}$. Leidke $\displaystyle\lim_n \frac{a_1^2+a_2^2+...+a_n^2}{n^2}$\\\\
\textbf{Lahendus:}\\
Loon uue jada:
\begin{equation*}
\begin{aligned}
p_n:=a_n\iff\\
\lim_n \frac{p_1+p_2+...+p_n}{n}:=t\iff\\
P_n:=p_1+p_2+...+p_n=\lim_n tn\\
\end{aligned}
\end{equation*}
T\"ahistan kolmanda limiidi \"umber:
\begin{equation*}
\begin{aligned}
\lim_n \frac{a_1^2+a_2^2+...+a_n^2}{n^2}=\lim_n \frac{a_1*p_1+a_2*p_2+...+a_np_n}{n^2}=\\
\lim_n t\frac{a_1*p_1+a_2*p_2+...+a_np_n}{tn}=\lim_n t\frac{a_1*p_1+a_2*p_2+...+a_np_n}{nP_n}
\end{aligned}
\end{equation*}
Ning loengu konspekti kaalutud keskmiste kohta k\"aiva lause 2.31 kohaselt:
\begin{equation*}
\begin{aligned}
a_n\rightarrow A\Rightarrow\lim_n \frac{a_1*p_1+a_2*p_2+...+a_np_n}{P_n}\rightarrow A\iff\\
\\
\end{aligned}
\end{equation*}
\end{document}