\documentclass{article}
\usepackage{amsfonts}
\usepackage{amsmath}
\usepackage{mathtools}
\DeclarePairedDelimiter\bfloor{\Bigl\lfloor}{\Bigr\rfloor}
\DeclarePairedDelimiter\floor{\lfloor}{\rfloor}
\begin{document}
\begin{center}
\Large\textbf{T\"arn\"ulesanne diferentssuhted}\\
\small{Joosep N\"aks}
\end{center}
Olgu $f$ n korda pidevalt diferentseeruv. N\"aidake, et iga $k\leq n$ korral\\
\begin {equation*}
\lim_{h\to0} f[x,x+h,...,x+kh]=\frac{1}{k!}f^{(k)}(x)
\end{equation*}
Diferentssuhte definitsioon:
\begin{equation*}
f[x_{i_0},x_{i_1},...,x_{i_k}]=\frac{f[x_{i_1},...,x_{i_k}]-f[x_{i_0},...,x_{i_{k-1}}]}{x_{i_k}-x_{i_0}}
\end{equation*}
\textbf{Lahendus:} Lahendan matemaatilise induktsiooniga:\\
\textbf{Baas:} $k=1$ korral saame v\~orrandi vasakule poolele diferentssuhte definitsioonist tuletise definitsiooni:
\begin{equation*}
\lim_{h\to0} f[x,x+h]=\lim_{h\to0}\frac{f[x]-f[x+h]}{x-(x+h)}=\lim_{h\to0}\frac{f[x+h]-f[x]}{h}=f'(x)=\frac{1}{1!}f^{(1)}(x)
\end{equation*}
Seega $k=1$ puhul v\~orrand kehtib.\\
\textbf{Samm:} Eeldame, et $k=t$ puhul kehtib:
\begin {equation}
\lim_{h\to0} f[x,x+h,...,x+th]=\frac{1}{t!}f^{(t)}(x)
\end{equation}
Sel juhul saab $k=t+1$ lahti kirjutada nii:
\begin {equation*}
\begin{aligned}
\lim_{h\to0} f[x,x+h,...,x+h(t+1)]=&\lim_{h\to0}\frac{f[x+h,...,x+h(t+1)]-f[x,...,x+ht]}{x+h(t+1)-x}\\
=&\lim_{h\to0}\frac{f[x+h,...,x+h(t+1)]-\frac{1}{t!}f^{(t)}(x)}{h(t+1)}
\end{aligned}
\end{equation*}
Vaatleme kuju $\displaystyle\lim_{h\to0}f[x+h,...,x+h(t+1)]$. Kui siin teha muutujavahetus $u=x+h$, saame $\displaystyle\lim_{h\to0}f[u,u+h,...,u+ht]$ ning eelduse (1) kohaselt on see v\~ordne kujuga $\frac{1}{t!}f^{(t)}(u)=\frac{1}{t!}f^{(t)}(x+h)$.
\begin {equation*}
\begin{aligned}
\lim_{h\to0}\frac{f[x+h,...,x+h(t+1)]-\frac{1}{t!}f^{(t)}(x)}{h(t+1)}=&\lim_{h\to0}\frac{\frac{1}{t!}f^{(t)}(x+h)-\frac{1}{t!}f^{(t)}(x)}{h(t+1)}\\
=&\lim_{h\to0}\frac{1}{(t+1)!}\frac{f^{(t)}(x+h)-f^{(t)}(x)}{h}\\
=&\frac{1}{(t+1)!}f^{(t+1)}(x)
\end{aligned}
\end{equation*}
Seega on t\~oestatud, et iga $0<k\leq n$ korral kehtib\\
\begin {equation*}
\lim_{h\to0} f[x,x+h,...,x+kh]=\frac{1}{k!}f^{(k)}(x)
\end{equation*}
\end{document}