\documentclass[a4paper, 10pt]{article}
\usepackage[estonian]{babel}
\usepackage{t1enc}
\usepackage{amsthm}
\usepackage{amscd}
\usepackage{amssymb}
\usepackage{lscape}
\usepackage{amsfonts}
\usepackage{amsmath}
\usepackage{mathtools}
\usepackage{systeme}
\usepackage{polynom}
\usepackage[shortlabels]{enumitem}
\usepackage[a4paper,margin=1in,footskip=0.25in]{geometry}
\usepackage{pgffor}
\everymath{\displaystyle}
\DeclarePairedDelimiter\ceil{\lceil}{\rceil}
\newcommand{\p}[1]{\frac{\partial}{\partial #1}}
\newcommand{\Z}{\mathbb{Z}}
\newcommand{\N}{\mathbb{N}}
\newcommand{\w}{\overline}
\topmargin-3em
\oddsidemargin0cm
\textwidth16cm
%\textheight27cm
\evensidemargin-2cm
\begin{document}
\begin{center}
\Large\textbf{Kodutöö nr. 6 esimene tärnülesanne}\\
\small{Joosep Näks}
\end{center}
Väidan, et antud jadas leidub iga mooduli $n$ jaoks mingi liige $a_k$, millest alates on kõik järgnevad liikmed mooduli $n$ järgi samad, ehk periood on 1. Võtan suvalise liikme $a_t$, $t\geq k$ mooduli $n$ järgi.\\
\indent Kui $n$ on paarisarv, saab seda avaldada kujul $n=2^p\cdot b$, kus $b$ on paaritu ehk $(2^p,b)=1$, seega $\Z_n$ ja $\Z_{2^p}\times\Z_b$ on isomorfsed, ehk saab vaadata eraldi liikme jääki $2^p$ ja $b$ järgi. Liige $a_k$ on valitud sobivalt, et $t\geq k>p$ ehk $a_t=2^{(2^{2..})}=(2^p)^q\equiv0\pmod{2^p}$. Mooduli $b$ jaoks saab kasutada Euleri teoreemi, kuna $(2,b)=1$ ehk $2^{\varphi(b)}\equiv1\pmod b$. Tähistan $a_t$ astendaja: $a_t=2^x$ ning jagan seda arvuga $\varphi(b)$ jäägiga: $x=q\varphi(b)+r$. Nüüd saab vaadeldava liikme avaldada: $a_t=2^x=(2^{\varphi(b)})^q\cdot 2^r\equiv2^r\pmod b$. Ehk kokkuvõttes $a_t$ vastab $\Z_{2^p}\times\Z_b$ liikmele $(0,2^r)$, $r<\varphi(b)$, $r\equiv x\pmod{\varphi(b)}$.\\
\indent Kui $n$ on paaritu, saab kohe kasutada Euleri teoreemi ning analoogselt saab, et $a_t\equiv2^r\pmod n$, kus $r<\varphi(n)$ ja $r\equiv x\pmod{\varphi(n)}$, kus $a_t=2^x$.\\
\indent Nüüd saab seda korrata, võttes $a_t$ asemele $x$ ning $n$ asemele vastavalt $\varphi(b)$ või $\varphi(n)$, olenevalt kas $n$ oli paaris või paaritu. Seda saab nii kaua korrata, kuni moodul, mille järgi jääki võetakse (ehk algselt $n$, hiljem $\varphi(n)$ või $\varphi(b)$ ning järgmisel tasemel $\varphi(\varphi(n))$ või midagi sarnast jne) on mõni 2 aste $2^w$. Sel juhul nagu varem näidatud, kui on $a_k$ sobivalt valitud, on alles jäänud astendajate jääk $2^w$ järgi 0.\\
Moodul jõuab kindlasti lõpliku koguse sammude jooksul mõne 2 astmeni, kuna alati kehtib $\varphi(n)<n$. Seda seetõttu, et $\varphi(n)$ on arvust $n$ väiksemate ja võrdsete arvude kogus, mille suurim ühistegur arvuga $n$ on 1, kuid arvust $n$ väiksemaid arve on $n-1$ tükki ning $(n,n)=1$ kehtib vaid juhul kui $n=1$ ehk kui $n>1$ siis $\varphi(n)<n$. Seega kui vahepeal mõne muu 2 astmeni ei jõua, tuleb lõpuks 2 ise vastu. Arvust 2 ei saa mööda minna kuna $\varphi(n)$ ei saa kunagi olla väiksem kui 1, kuna 1 on alati arvu $n$ jagaja ehk $\varphi(n)$ on vähemalt 1, ja $\varphi(n)$ väärtus saab olla 1 vaid juhul, kui $n=2$, kuna kui $n$ on mõni suurem arv, on $(1,n)=1$ ja $(n,n-1)=1$, sest kui kehtiks $d|n$ ja $d|n-1$, kehtiks ka $d|n-(n-1)=1$, kuid ükski algarv ei jaga arvu 1.\\
\indent Seega kokkuvõttes kui $a_p$ on valitud piisavalt kaugele, on kõik järgnevad liikmed sellega võrdsed mooduli $n$ järgi.
\end{document}