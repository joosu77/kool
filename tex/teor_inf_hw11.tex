\documentclass[a4paper, 10pt]{article}
\usepackage[estonian]{babel}
\usepackage{t1enc}
\usepackage{amsthm}
\usepackage{amscd}
\usepackage{amssymb}
\usepackage{lscape}
\usepackage{amsfonts}
\usepackage{amsmath}
\usepackage{diagbox}
\usepackage[official]{eurosym}
\usepackage{mathtools}
\usepackage{systeme}
\usepackage{polynom}
\usepackage{xcolor}
\usepackage[shortlabels]{enumitem}
\usepackage[a4paper,margin=1in,footskip=0.25in]{geometry}
\usepackage{pgffor}
\everymath{\displaystyle}
\DeclarePairedDelimiter\ceil{\lceil}{\rceil}
\newcommand{\p}[1]{\frac{\partial}{\partial #1}}
\newcommand{\Z}{\mathbb{Z}}
\newcommand{\N}{\mathbb{N}}
\newcommand{\B}{\mathbb{P}}
\newcommand{\w}{\overline}
\newcommand{\ind}{\mathrm{ind}}
\newcommand{\db}[2]{\slashbox{#1}{#2}}
\newcommand{\leg}[2]{\left(\frac{#1}{#2}\right)}
\topmargin-3em
\oddsidemargin0cm
\textwidth16cm
%\textheight27cm
\evensidemargin-2cm
\begin{document}
\begin{center}
\Large\textbf{Homework 11}\\
\small{Joosep Näks}
\end{center}

In all sections for big O notation I will be using the definition that $$f(n)=O(g(n))\Leftrightarrow\limsup_{n\to\infty}\frac{|f(n)|}{g(n)}<\infty$$ and for small o notation I will be using the definition that $$f(n)=o(g(n))\Leftrightarrow\lim_{n\to\infty}\frac{f(n)}{g(n)}=0$$.

a) Take the limit 
\begin{gather*}
\begin{aligned}
\lim_{n\to\infty}\frac{f(n)}{g(n)}=&\lim_{n\to\infty}\frac{n^{2.5}}{3n^3+2n^2+5n\log_2(n)+1}\\
=&\lim_{n\to\infty}\frac{n^{2.5}n^{-2.5}}{(3n^3+2n^2+5n\log_2(n)+1)n^{-2.5}}\\
=&\lim_{n\to\infty}\frac{1}{3n^{0.5}+2n^{-0.5}+5n^{-1.5}\log_2(n)+n^{-2.5}}\\
\end{aligned}
\end{gather*}
In the denominator of the last limit $3n^{0.5}$ goes to infinity as $n$ goes to infinity and the rest of the terms are all positive for any positive $n$ so the denominator goes to infinity while the numerator is finite, meaning that the limit is 0. This means that both $f(n)=O(g(n))$ and $f(n)=o(g(n))$. If we take the inverse of the fraction i nthe limit we get infinity since the numerator goes to infinity and the denominator is finite, meaning that neither $g(n)=O(f(n))$ nor $g(n)=o(f(n))$ hold.

b) Take the limit 
\begin{gather*}
\begin{aligned}
\lim_{n\to\infty}\frac{f(n)}{g(n)}=&\lim_{n\to\infty}\frac{\log_2(n)\cdot\log_2\log_2(n)}{n^{\frac13}}\\
\lim_{n\to\infty}\frac{f(n)}{g(n)}=&\lim_{n\to\infty}\frac{\log_2(n)\cdot\log_2\log_2(n)}{n^{\frac13}}\\
\end{aligned}
\end{gather*}

c) Take the limit 
\begin{gather*}
\begin{aligned}
\lim_{n\to\infty}\frac{f(n)}{g(n)}=&\lim_{n\to\infty}\frac{2^{(n^2)}}{100^{5n}}\leq\lim_{n\to\infty}\frac{2^{(n^2)}}{128^{5n}}\\
=&\lim_{n\to\infty}\frac{2^{(n^2)}}{2^{7\cdot5n}}\\
=&\lim_{n\to\infty}2^{n^2-35n}=\infty\\
\end{aligned}
\end{gather*}
Meaning that neither $f(n)=O(g(n))$ nor $f(n)=o(g(n))$ hold. If we take the inverse however, it is easy to see that the limit goes to 0, meaning that both $g(n)=O(f(n))$ and $g(n)=o(f(n))$ hold.
\end{document}