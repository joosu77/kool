\documentclass{article}
\usepackage{amsfonts}
\usepackage{amsmath}
\usepackage{mathtools}
\usepackage{systeme}
\usepackage{polynom}
\usepackage{pgfplots}
\usepackage[shortlabels]{enumitem}
\usepackage[a4paper,margin=1in,footskip=0.25in]{geometry}
\everymath{\displaystyle}
\DeclarePairedDelimiter\ceil{\lceil}{\rceil}
\newcommand{\p}[1]{\frac{\partial}{\partial #1}}
\begin{document}
\begin{center}
\Large\textbf{Kodutöö nr. 6}\\
1. variant\\
\small{Joosep Näks}
\end{center}
\textbf{1. }Arendada funktsioon$$f(x)=\frac{\pi}4-\frac x2,\ x\in[0,\pi],$$siinusreaks. Uurida rea punktiviisi koonduvust (s.t. kas koondub ja mis väärtuseks) lõigus $[0,\pi]$.\\
Olgu $s_n(x)$ selle siinusrea n-nes osasumma ning olgu $$\sigma_n(x)=\frac{s_0(x)+...+s_n(x)}{n+1}$$ Joonistada lõigus $[-5,5]$ graafik, millel oleks toodud funktsioonid $f,\ s_10,\ s_100$ ning samas lõigus teine graafik, millel oleks toodud funktsioonid $f,\ \sigma_10,\ \sigma_100$.\\\\
\textbf{Lahendus:}\\

\end{document}